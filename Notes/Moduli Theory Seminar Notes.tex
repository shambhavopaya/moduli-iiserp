\documentclass[11pt]{amsart}
\usepackage[margin=1.2in,marginparsep=0.1in,marginparwidth=1in]{geometry}
\usepackage{amssymb,amsmath,amsthm,amstext,amscd,latexsym,graphics,graphicx,bbm,caption}
\usepackage[usenames,dvipsnames,svgnames,table]{xcolor}
\usepackage[plainpages=false,colorlinks=true, linkcolor=blue, citecolor=red, pagebackref]{hyperref}
\usepackage{enumitem}
\usepackage{tikz-cd}
\usepackage{verbatim}


\makeatletter
\def\@settitle{\begin{center}%
		\baselineskip14\p@\relax
		\bfseries
		\uppercasenonmath\@title
		\@title
		\ifx\@subtitle\@empty\else
		\\[1ex]\uppercasenonmath\@subtitle
		\footnotesize\mdseries\@subtitle
		\fi
	\end{center}%
}
\def\subtitle#1{\gdef\@subtitle{#1}}
\def\@subtitle{}
\makeatother


%% LaTeX Definitions
%\newcounter{countup}

\newcommand{\rup}[1]{\lceil{#1}\rceil}
\newcommand{\rdown}[1]{\lfloor{#1}\rfloor}
\newcommand{\ilim}{\mathop{\varprojlim}\limits} % inverse limit
\newcommand{\dlim}{\mathop{\varinjlim}\limits}  % direct limit
\newcommand{\surj}{\twoheadrightarrow}
\newcommand{\inj}{\hookrightarrow}
\newcommand{\tensor}{\otimes}
\newcommand{\ext}{\bigwedge}
\newcommand{\Intersection}{\bigcap}
\newcommand{\Union}{\bigcup}
\newcommand{\intersection}{\cap}
\newcommand{\union}{\cup}

%%%%%%%%%%%%%%%%%%%%%%%%%%%%% new new commands :) %%%%%%%%%%%%%%%%
\newcommand{\supp}{{\rm Supp}}
\newcommand{\Exceptional}{{\rm Ex}}
\newcommand{\del}{\partial}
\newcommand{\delbar}{\overline{\partial}}
\newcommand{\boldphi}{\mbox{\boldmath $\phi$}}

%%%%%%%%%%%%%%%%%%%%%%%%%%%%%%%%%%%%%%%%%%%%%%%%%%%%%%%%%%%%%%%%%%%%%%%%%%%%%%

\newcommand{\udiv}{\underline{\Div}}

%%%%%%%%%%%%%%%%%

\newcommand{\Proj}{{\P roj}}
\newcommand{\sEnd}{{\sE nd}}
\newcommand{\mc}{\mathcal}
\newcommand{\mb}{\mathbb}
\newcommand{\an}{{\rm an}} 
\newcommand{\red}{{\rm red}}
\newcommand{\codim}{{\rm codim}}
\newcommand{\Dim}{{\rm dim}}
\newcommand{\rank}{{\rm rank}}
\newcommand{\Ker}{{\rm Ker  }}
\newcommand{\Pic}{{\rm Pic}}
\newcommand{\per}{{\rm per}}
\newcommand{\ind}{{\rm ind}}
\newcommand{\Div}{{\rm Div}}
\newcommand{\Hom}{{\rm Hom}}
\newcommand{\Aut}{{\rm Aut}}
\newcommand{\im}{{\rm im}}
\newcommand{\Spec}{{\rm Spec \,}}
\newcommand{\Sing}{{\rm Sing}}
\newcommand{\sing}{{\rm sing}}
\newcommand{\reg}{{\rm reg}}
\newcommand{\Char}{{\rm char}}
\newcommand{\Tr}{{\rm Tr}}
\newcommand{\Gal}{{\rm Gal}}
\newcommand{\Min}{{\rm Min \ }}
\newcommand{\Max}{{\rm Max \ }}
\newcommand{\Alb}{{\rm Alb}\,}
\newcommand{\Mat}{{\rm Mat}}
%\newcommand{\GL}{{\rm GL}\,}        % For the general linear group
\newcommand{\GL}{{\G\L}}
\newcommand{\Ho}{{\rm Ho}}
\newcommand{\ie}{{\it i.e.\/},\ }
\renewcommand{\iff}{\mbox{ $\Longleftrightarrow$ }}
\renewcommand{\tilde}{\widetilde}
% Skriptbuchstaben
\newcommand{\sA}{{\mathcal A}}
\newcommand{\sB}{{\mathcal B}}
\newcommand{\sC}{{\mathcal C}}
\newcommand{\sD}{{\mathcal D}}
\newcommand{\sE}{{\mathcal E}}
\newcommand{\sF}{{\mathcal F}}
\newcommand{\sG}{{\mathcal G}}
\newcommand{\sH}{{\mathcal H}}
\newcommand{\sI}{{\mathcal I}}
\newcommand{\sJ}{{\mathcal J}}
\newcommand{\sK}{{\mathcal K}}
\newcommand{\sL}{{\mathcal L}}
\newcommand{\sM}{{\mathcal M}}
\newcommand{\sN}{{\mathcal N}}
\newcommand{\sO}{{\mathcal O}}
\newcommand{\sP}{{\mathcal P}}
\newcommand{\sQ}{{\mathcal Q}}
\newcommand{\sR}{{\mathcal R}}
\newcommand{\sS}{{\mathcal S}}
\newcommand{\sT}{{\mathcal T}}
\newcommand{\sU}{{\mathcal U}}
\newcommand{\sV}{{\mathcal V}}
\newcommand{\sW}{{\mathcal W}}
\newcommand{\sX}{{\mathcal X}}
\newcommand{\sY}{{\mathcal Y}}
\newcommand{\sZ}{{\mathcal Z}}
% Sonderbuchstaben mit Doppellinie
\newcommand{\A}{{\mathbb A}}
\newcommand{\B}{{\mathbb B}}
\newcommand{\C}{{\mathbb C}}
\newcommand{\D}{{\mathbb D}}
\newcommand{\E}{{\mathbb E}}
\newcommand{\F}{{\mathbb F}}
\newcommand{\G}{{\mathbb G}}
\newcommand{\HH}{{\mathbb H}}
\newcommand{\I}{{\mathbb I}}
\newcommand{\J}{{\mathbb J}}
\newcommand{\M}{{\mathbb M}}
\newcommand{\N}{{\mathbb N}}
\renewcommand{\O}{{\mathbb O}}
\renewcommand{\P}{{\mathbb P}}
\newcommand{\Q}{{\mathbb Q}}
\newcommand{\R}{{\mathbb R}}
\newcommand{\T}{{\mathbb T}}
\newcommand{\U}{{\mathbb U}}
\newcommand{\V}{{\mathbb V}}
\newcommand{\W}{{\mathbb W}}
\newcommand{\X}{{\mathbb X}}
\newcommand{\Y}{{\mathbb Y}}
\newcommand{\Z}{{\mathbb Z}}
\newcommand{\Sh}{\sS h}
\newcommand{\deltaop}{\Delta^{op}(\sS h(Sm/\mathbf{k}))}
\newcommand{\pdeltaop}{\Delta^{op}(P\sS h(Sm/\mathbf{k}))}
%\newcommand{\psh}{\pi_0^{\text{\tiny pre}}}
\newcommand{\psh}{\pi_0}
\renewcommand{\k}{\mathbf{k}}

\newcommand{\colim}{{\rm colim \,}}
\newcommand{\DM}[2]{\mathbf{DM}_{#2}^{\mathit{eff}}(#1)}

\theoremstyle{definition}
\newtheorem{theorem}{Theorem}[section]
\newtheorem{lemma}[theorem]{Lemma}
\newtheorem{definition}[theorem]{Definition}
\newtheorem{question}[theorem]{Question}
\newtheorem{example}[theorem]{Example}



\begin{document}
% Document information
\title{Seminar on Moduli Theory}
\subtitle{\today}
\author{Neeraj Deshmukh}
%\date{\today}
%\address[IISERM]{Indian Institute of Science Education and Research, Mohali}

\maketitle


These are notes for the first few lectures. The aim is to gather sufficient algebraic geometry background for discussing moduli theory. A lot of \LaTeX\, code in this document has been shamelessly copied from the stacks project repository on GitHub\footnote{Thank you Aise Johan de Jong \textit{et al.} for TeX-ing all that math!}.

\begin{comment}
\section{Plan}

One way to go to through these sessions is go recall theory of schemes (at level of things in Hartshorne chapter two), by doing a beeline through all the definitions, properties, etc. However, I feel since the point of doing this exercise is to become more comfortable in working with scheme, we will just do lots of examples instead. By this, I mean we will just to prove some things in very concrete situations. This will help you build a concrete picture of the generalities.

List of some things to discuss (just do lots of examples):

\begin{enumerate}
	\item Definition of a scheme.
	\item Say affine communication lemma \textit{stress this!}
	\item examples
	\begin{enumerate}
		\item $\P^n$ and it sheaf theory! This already clarifies the $\sO(n)$'s
	\end{enumerate}
\end{enumerate}


Things to say: finite-generation, valuative criteria for $\P^1$ (using DVR's - how giving a point and its specialisation is the same as giving a map $\Spec R\rightarrow \P^1$).\\
List of examples:
\begin{enumerate}
	\item $\A^1$ with a double point (what are quasi-coherent sheaves on this?)
	\item $\A^2\setminus \lbrace 0,0\rbrace$ (what is the structure sheaf?).
	\item $\P^1$ (its structure sheaf!).
	\item $V_{+}(x^2+y^2+z^2)$ over $\R$ and $\C$. The point is that there is a change of coordintes which 
	\item Blow-up of $\A^2$ at a point. (because everyone should know about blow-ups!)
	\item $\Spec R[x_1,x_2,\ldots]$ as an example of something non-noetherian.
	\item An example of a scheme without a closed point.
\end{enumerate}

List of morphisms:
\begin{enumerate}
	\item $x\mapsto x^2$ (more, generally $x^n$). This covers ramified, finitely presented, flat.
	\item a non-quasi-compact open-immersion. Polynomial ring in infinitely many variables and knock off the origin. Also, the origin of is an example of something 
\end{enumerate}
\end{comment}

\section{Schemes}
For the sake of completeness we begin by reviewing the definition of a locally ringed space.

\begin{definition}
	\label{definition-locally-ringed-space}
	Locally ringed spaces.
	\begin{enumerate}
		\item A {\it locally ringed space $(X, \mathcal{O}_X)$}
		is a pair consisting of a
		topological space $X$ and a sheaf of rings $\mathcal{O}_X$ all of whose stalks
		are local rings.
		\item Given a locally ringed space $(X, \mathcal{O}_X)$ we say that
		$\mathcal{O}_{X, x}$ is the {\it local ring of $X$ at $x$}.
		We denote $\mathfrak{m}_{X, x}$ or simply $\mathfrak{m}_x$
		the maximal ideal of $\mathcal{O}_{X, x}$. Moreover, the
		{\it residue field of $X$ at $x$} is the residue field
		$\kappa(x) = \mathcal{O}_{X, x}/\mathfrak{m}_x$.
		\item A {\it morphism of locally ringed spaces}
		$(f, f^\sharp) : (X, \mathcal{O}_X) \to (Y, \mathcal{O}_Y)$
		is a morphism of ringed spaces such that for all $x \in X$
		the induced ring map $\mathcal{O}_{Y, f(x)} \to \mathcal{O}_{X, x}$ is a
		local ring map.
	\end{enumerate}
\end{definition}


We know that affine schemes are locally ringed spaces: we take $\Spec R$ with the zariski topology and for any principal open set $D(f)$ we assign the ring $R_f$. So, any ring $R$ produces the sheaf $\tilde{R}$ on $\Spec R$. This is called the tilde construction. (sanity check: if you can do this, then you should be able to construct a sheaf on $\Spec R$ for any $R$-module $M$).


\begin{definition}
	A {\it scheme} is a locally ringed space with the property that
	every point has an open neighbourhood which is an affine scheme.
	A {\it morphism of schemes} is a morphism of locally
	ringed spaces. The category of schemes will be denoted
	$Sch$.
\end{definition}

\begin{definition}
	Let $(X,\mathcal{O}_X)$ be a scheme. A sheaf of modules on $X$ is a sheaf $\mathcal{F}$ on $X$ such that for every open set $U$, $\mathcal{F}(U)$ is an $\mathcal{O}_X(U)$-module. We say that a sheaf of modules $\mathcal{F}$ is \textit{quasi-coherent} if for every affine open $U\simeq \Spec(R)$, the sheaf $\mathcal{F}|_U$ on $U$ is of the form $\tilde{M}$ for some $R$-module $M$.
\end{definition}


Make special note of the next lemma. This basically lets us reduce problems about schemes to statement about affine schemes (ergo, ring theory), whenever the problem at hand is of a \textit{local} nature. Ravi Vakil calls this \textit{affine communication lemma}.


\begin{lemma}
	\label{lemma-locally-P}
	Let $X$ be a scheme. Let $P$ be a local property of rings.
	The following are equivalent:
	\begin{enumerate}
		\item The scheme $X$ is locally $P$.
		\item For every affine open $U \subset X$ the property
		$P(\mathcal{O}_X(U))$ holds.
		\item There exists an affine open covering $X = \bigcup U_i$ such that
		each $\mathcal{O}_X(U_i)$ satisfies $P$.
		\item There exists an open covering $X = \bigcup X_j$
		such that each open subscheme $X_j$ is locally $P$.
	\end{enumerate}
	Moreover, if $X$ is locally $P$ then every open subscheme
	is locally $P$.
\end{lemma}

This is how commutative algebra meets geometry. Often, the properties that we want to consider are ``globalised" versions of statements about rings.\footnote{You can also ``globalise" morphisms of rings, but now you have two schemes to work locally on. We'll do this soon.}

\subsection{Two ways of Gluing \texorpdfstring{$\A^1\setminus \lbrace 0\rbrace$}{Pn}}

Take two copies of $\A^1:= \Spec k[x]$\footnote{For simplicity, assume that $k$ is field, but this is not needed.}. Let $U:= \Spec k[x,1/x]$, be the complement of the origin in $\A^1$.

\begin{center}
\begin{tikzcd}
	\A^1 &	\A^1\\
	U\arrow[u, hook]\arrow{r}{\sim} & U\arrow[u, hook]\\
\end{tikzcd}
\end{center}

Giving this information is that same giving a scheme which is looks like $\A^1$ around every point (why?). We consider two possible choices for the identification on $U$:
\begin{align*}
x &\mapsto x\\
x &\mapsto 1/x
\end{align*}
\begin{example}
 The first choice gives us a scheme which is like $\A^1$ everywhere except at the origin where it is now two points instead of one. Notice that the ring of global section of this scheme is $k[x]$ (a global section is same as giving polynomials $f, g\in k[x]$, one for each copy of $\A^1$ which are equal on $U$; conclude form this).
\end{example}

\begin{example}
The second choice gives us the projective line $\P^1$. This is looks like $\A^1$ with ``a point added at infinity". We will now compute its global sections.

Let $f,g\in\A^1$ be two polynomials such that $f(x)=g(1/x)$ in $k[x,1/x]$. Then straightforward algebra shows that this can happen only when $f, g$ are constant, i.e, $\Gamma(\P^1,\sO_{\P^1})=k$.
\end{example}

\subsection{A DVR with double origin.} Similar to $\A^1$ with double origin, we can glue two copies of a DVR. Let $R$ be a discrete valuation ring. Then $\Spec R$ has exactly two point: the generic point (zero ideal) and the closed point (maximal ideal). The generic point is open in $\Spec R$ and is given by $\Spec K$, where $K$ is quotient field of $R$. As for $\A^1$, the ring of global sections of a DVR with double origin is $R$.

Furthermore, to determine any coherent sheaf (a quasi-coherent sheaf which is a finitely generated module on each copy of $R$) it is sufficient to give a pair $(n,T)$ where $n$ is a positive integer and $T\in Gl_n(K)$. Since $R$ is a principal ideal domain, any finitely generated $R$-module $M$ is a direct sum of its free and torsion parts. Thus, if $M, N$ are two finitely generated $R$-modules,there exists an isomorphism (given by a $K$-linear map) $M\otimes K \simeq N\otimes K$ if and only if the rank of their free parts is the same. Of course, this description is not unique. However, if we restrict to locally free sheaves, then we have a correspondence between locally free sheaves on DVR with a double origin and pair $(n,T)$.


\begin{comment}
\begin{tikzcd}[remember picture]
	A \arrow[r] & B\\
	C \arrow[r] & D\\
\end{tikzcd}
\begin{tikzpicture}[overlay,remember picture]
\path (\tikzcdmatrixname-2-1) to node[midway,sloped]{$\subseteq$}
(\tikzcdmatrixname-1-1);
\path (\tikzcdmatrixname-2-2) to node[midway,sloped]{$\subseteq$}
(\tikzcdmatrixname-1-2);
\end{tikzpicture}
\end{comment}

\subsection{\texorpdfstring{$P$}{P} versus locally \texorpdfstring{$P$}{P}}\label{PvlP}
All the above are examples of locally normal (in fact, regular), locally reduced and locally Noetherian scheme\footnote{This is probably not standard notation, but instructive for the current discussion.}. For any property that is locally $P$ (as defined in \ref{lemma-locally-P}), the usual rule of thumb for nomenclature is $P=$ locally $P\,\,+$ quasi-compact: for example, a scheme is Noetherian if it is locally Noetherian and quasi-compact. Not all properties are of this type: for example quasi-compactness, sepratedness, properness, etc. We will come back to this when we discuss morphisms.



\subsection{Line Bundles on \texorpdfstring{$\P^1$}{P1}:}
 Locally on an affine open, this should be a free module of rank one. Let's contruct one such line bundle (non-trivial, of course): There are two open sets, $D(x)$ and $D(y)$, on these our line bundle looks like $k[x]$ and $k[y]$, respectively. Now, how do they glue on $k[x,1/x]\simeq k[y,1/y]$? Let's use the map which sends $\phi(1):f(x)\mapsto f(x)y$, since $y$ is $1/x$ in this ring, we see that the global sections are linear polynomials. You construct such a map $\phi(n)$ for every power of $y$. That will give you degree $n$ monomials. These line bundles are called $\sO(n)$'s. Playing around with the algebra of the maps $\phi(n)$ a little will that these line bundles satisfy relations like $\sO(n)\otimes\sO(m)\simeq \sO(m+n)$, and admit duals which are denote by $\sO(-n)$.\footnote{The line bundle $\sO(1)$ is important. To say that a variety is projective, we need to show that something like $\sO(1)$ lives on it. Actually, some lesser works, but we will come back to this later.}
[Discussed till here as of August 28, 2020]


\subsection{A slightly more involved scheme: \texorpdfstring{$\P^n$}{Pn}.}
Consider 
\[D(x_i):= \Spec k[x_{0/i},x_{1/i},\ldots,\ldots,x_{n/i}]/(x_{i/i}-1).\]
This is basically the same as $\A^n$, but we write it like this for reasons that will become evident soon. If we invert one of the variables, say $x_{j/i}$, we can write an isomorphism $D(x_i)_{x_{j/i}}\cong D(x_j)_{x_{i/j}}$ given by the maps 

\[\phi_{ij}:x_{k/i}\mapsto x_{k/j}/x_{i/j}\;\; \&\;\; \phi_{ji}:x_{k/j}\mapsto x_{k/i}/x_{j/i}\]
Now we just have to check that this agrees on triples. For this, you just have to check that $\phi_{ij}\circ\phi_{jk}=\phi_{ik}$ (what is the (co)domain of these maps?). Note that this construction doesn't really utilise the fact that $k$ is a field.

\subsection{A classical interlude.} Here's a classical definition of $\P^n$ in terms of \textit{homogeneous coordinates}. Let $k$ be a field. Consider $k^{n+1}\setminus{(0,0,\ldots,0)}$. We define $\P^n$ to be:
\[\P^n:=\lbrace (x_0,x_1,\ldots,x_n)\; |\; (x_i)\simeq (y_i)\; \text{if there is a $\lambda\in k^{\times}$ such that for all $i$,}\; x_i=\lambda y_i\rbrace\]

We denote the equivalence class of the tuple $(x_i)$ by $[x_0:x_1:\ldots:x_n]$. These are called homogeneous coordinates. If we assume one of the coordinates to be non-zero, say $x_i$, then we can divide the entire tuple by it. This gives us a set, $D(x_i):=\lbrace [x_0/x_i:x_1/x_i:\ldots:1:\ldots:x_n/x_i]\rbrace$. It is easy to see that this set in bijection with $k^n$. Set $x_k/x_i:=x_{k/i}$\footnote{This notation is meant to be suggestive.}, then tuples in $D(x_i)$ look like $[x_{0/i}:x_{1/i}:\ldots:1:\ldots:x_{n/i}]$. This $D(x_i)$ should be thought of as the complement of the hyperplane defined by $x_i$ (which is actually a projective space of one dimension less). 

If an $x_j$ is non-zero, for a $j$ distinct from $i$, then we can divide by it. This gives us the relation, $x_{k/i}/x_{j/i}=x_{k/j}$. This is the origin of the morphisms $\phi_{ij}$ above\footnote{If you have seen the construction of Grassmannians as smooth manifolds, the same construction also goes through in algebraic geometry.}. Homogenisation and de-homogenisation(?).


\subsection{Proj of a graded ring.} Now that we have defined $\P^n$, I want to motivate the proj construction for graded rings using the example of $\P^n$. In a sense, this construction is not very different from the gluing construction above. However, it gives us more control ove the algebra and sheaf theory of $\P^n$ and its subschemes. For example, every closed subscheme of $\P^n$ comes from a graded ideal (this is a neat analogue of the affine case). Another advantage is that it lets us talk about affine open covers given by the complements of non-linear hypersurfaces. This can be useful when dealing with closed subschemes of $\P^n$, but then we won't need to make any choices about waht affine covers should be. 

Consider the ring $S_{\bullet}:=k[x_0,x_1,\ldots,x_n]$, now thought of as a graded ring with the grading given by degrees of monomials. The degree of a monomial $x^{r_1}_{i_1}\ldots x^{r_m}_{i_m}$ is the integer $r_1+\ldots +r_m$. The constants have degree zero. We can write our ring as $S_{\bullet}=\oplus_{i\geq 0}S_i$, where each $S_i$ is the homogeneous component of degree $i$. Note that $S_{\bullet}$ is generated by the elements $x_i$'s as an $S_0$-algebra (here, $S_0$ is $k$). The ideal generated by the $x_i$'s is just $S_{+}=\oplus_{i>0}S_i$. This is just the ideal $(x_0,\ldots,x_n)$ written by keeping track of the grading. We will call this the irrelevant ideal. Then, proj of the graded ring $S_{\bullet}$, $\Proj(S_{\bullet})$ is the set of those \text{homogeneous} prime ideals which do not contain $S_+$. This inherits a Zariski topology from $\Spec S_{\bullet}$. We can then to check that this is a scheme by producing affine open covers using the homogeneous elements. The resulting scheme is $\P^n$.

This can be a bit tricky to write down when the homogeneous elements have large degrees. But to give you a flavour of what is going on, let's examine what this is in for $\P^1$. We have already seen a description of $\P^1$ above, by gluing $\A^1\setminus 0$ in an ``inverse fashion".  By the above discussion, $\P^1:=\Proj k[x,y]$. Here's a slightly different description of $\P^1$ using degree $2$ hypersurfaces. Consider the homogeneous ideal $(x^2,xy,y^2)$. Localise $k[x,y]$ with respect to $x^2$. This gives us a graded ring $(S_{\bullet})_{x^2}=k[x,y]_{x^2}$, where $1/x^2$ has degree $-2$. Thus, elements here are of the form $f(x,y)/(x^{2})^N$ and can have negative degrees. Look at the zero graded piece of this ring which we will denote by the horrible notation, $((S_{\bullet})_{x^2})_0$. This ring can be described as,
\[((S_{\bullet})_{x^2})_0=k[xy/x^2,y^2/x^2].\]
This can be identified with $k[y/x]$. So, $\Spec ((S_{\bullet})_{x^2})_0=\A^1$. At this point, we have to check that this does give an affine open in $\Proj k[x,y]$. This is true because there is a bijection between prime ideal of $((S_{\bullet})_{x^2})_0$ and the homogeneous prime ideals of $(S_{\bullet})_{x^2}$ (One way is easy. For the other direction, take a prime $\mathfrak{p}$ in $((S_{\bullet})_{x^2})_0$ and show that the \textit{radical} of the homogeneous ideal generated by $\mathfrak{p}$ in $(S_{\bullet})_{x^2}$ in prime\footnote{There is nothing particularly enlightening in doing this exercise for $x^2$, the exact same proof works for any element of positive degree.}). Similarly, inverting the elements $xy$ and $y^2$ in $S_{\bullet}$ and looking at the zeroth graded pieces gives us the polynomial rings,
\begin{align*}
((S_{\bullet})_{y^2})_0 &=k[x^2/y^2,xy/y^2]\longleftrightarrow \A^1\\
((S_{\bullet})_{xy})_0 &=k[x^2/xy,y^2/xy]\longleftrightarrow \A^1\setminus 0.
\end{align*}
The radical of the ideal $(x^2,xy,y^2)$ contains the irrelevant ideal (why?). This implies that every homogeneous prime ideal is contained in one of the above three affine pieces. Thus, this gives us a covering of $\P^1$.

If instead we invert degree one elements $x$ and $y$, the zeroth graded pieces of these localisations look like,
\begin{align*}
((S_{\bullet})_{x})_0 &=k[x/y]\longleftrightarrow\A^1\\
((S_{\bullet})_{y})_0 &=k[y/x]\longleftrightarrow\A^1.
\end{align*}
This will, then, recover our original construction of $\P^1$.

Note that if you just invert $x^2$ and $xy$, then this does not give a cover $\P^1$, since the the radical of $(x^2,xy)$ does not contain the irrelevant ideal. Geometrically speaking, this is because inverting $xy$ corresponds to the affine open of $\P^1$ obtained by removing $0$ and $\infty$. However, the radical of the ideal $(x^2,y^2)$ does, indeed, contain the irrelevant ideal\footnote{This was pointed out to me by Kartik Roy.}. This makes geometric sense because the elements $x$ and $x^2$ have the same vanishing locus.

\subsection{Closed subschemes of \texorpdfstring{$\P^n$}{Pn}} Just as in closed subschemes of an affine scheme $\Spec R$, are given by ideals $I\subset R$, closed subschemes of $\P^n$ correspond to homogeneous ideals of $k[x_0,\ldots,x_n]$.

Let $Y\overset{i}{\hookrightarrow} \P^n$ be a closed subscheme. This is given by a sheaf of ideals $\sI$ such that,

\[0\rightarrow \sI\rightarrow \sO_{\P^n}\rightarrow i_*\sO_Y\rightarrow 0.\]

On $D(x_i)$, this gives us an injection,
\[0\rightarrow \sI_{x_i}\rightarrow k[x_{0/i},x_{1/i},\ldots,\ldots,x_{n/i}]/(x_{i/i}-1). \]

Let $\sI_{x_i}$ be generated by a collection of polynomials $(f_1,\ldots,f_r)$ on $D(x_i)$. If the degree of highest degree term of $f_j$ is $r$, then $x_i^rf_j$ is a homogeneous polynomial in $\lbrace x_0,\ldots,x_n\rbrace$ which we will also call $f_j$, by abuse of notation. Doing this for every $D(x_j)$, gives us a collection of homogeneous elements of $k[x_0,\ldots,x_n]$. Let $I$ be the homogeneous ideal generated by these elements. More precisely,
\[I:=\lbrace (f_{kl})\;|\; \text{the elements}\; \lbrace f_{ki}\rbrace_k\; \text{generate}\; \sI_{x_i}\rbrace\]
We claim that $\tilde{I}=\sI$. To do this, it is sufficient to show that if $f_1,\ldots f_r$ are homogeneous elements of $k[x_0.\ldots,x_n]$ which generate $\sI_{x_i}$, and $g_1,\ldots,g_s$ are homogeneous elements of $k[x_0.\ldots,x_n]$ which generate $\sI_{x_j}$, then $f_l|_{D(x_j)}\in \sI_{x_j}$.

Note that on $D(x_i x_j)$ we have isomorphisms,
\begin{align*}
k[x_{0/i},\ldots,\widehat{x_{i/i}},\ldots,x_{n/i}]_{x_{j/i}} &\rightarrow k[x_{0/j},\ldots,\widehat{x_{j/j}},\ldots,x_{n/j}]_{x_{i/j}}\\
x_{k/i} &\overset{\phi_{ij}}{\longrightarrow} \frac{x_{k/j}}{x_{i/j}}\\
\frac{x_{k/i}}{x_{j/i}} &\overset{\phi_{ji}}{\longleftarrow} x_{k/j}.
\end{align*}
Now by the sheaf property, $\sI_{x_i}|_{D(x_i x_j)}=\sI_{x_j}|_{D(x_i x_j)}$. Thus, any generator $f_l$ of $\sI_{x_i}$ can be expressed in terms of the generators $g_k$'s of $\sI_{x_j}$ on $D(x_i x_j)$, and vice verse.

Write $f_l=f_l(x_{0/i},\ldots,x_{n/i})$, and $g_k=g_k(x_{0/j},\ldots,x_{n/j})$. Then, on $D(x_i x_j)$, we have,
\[f_l(x_{0/i},\ldots,x_{n/i})=f_l(\frac{x_{0/j}}{x_{i/j}},\ldots,\frac{x_{n/j}}{x_{i/j}})=\sum_k \frac{\alpha_k g_k}{x_{i/j}^{r_k}},\]
where $\alpha_k\in k[x_{0/j},\ldots,\widehat{x_{j/j}},\ldots,x_{n/j}]_{x_{i/j}}$. Let degree of the highest degree term of $f_l(x_{0/i},\ldots,x_{n/i})$ be $N$. Then, multiplying throughout by $x_{i/j}^N$, gives us
\[x_{i/j}^N f_l(\frac{x_{0/j}}{x_{i/j}},\ldots,\frac{x_{n/j}}{x_{i/j}})= f_l(x_{0/j},\ldots,x_{n/j})=\sum_k  x_{i/j}^{N-r_k} \alpha_k g_k.\]
Thus, $f_l|_{D(x_j)}\in \sI_{x_j}$. A similar argument shows that $g_k\in \sI_{x_i}$.

\noindent\textbf{Possible alternative approach.} Just as in the case of $\P^1$, we can define the line bundles $\sO(n)$'s whose global sections are degree $n$ polynomials. Therefore, we can write that $k[x_0,\ldots,x_n]=\oplus_m\Gamma (\sO(m),\P^n)$. For the exact sequence
\[0\rightarrow \sI\rightarrow \sO_{\P^n}\rightarrow i_*\sO_Y\rightarrow 0,\]
tensoring by $\sO(m)$ gives,
\[0\rightarrow \sI(m)\rightarrow \sO_{\P^n}(m)\rightarrow i_*\sO_Y(m)\rightarrow 0.\]

Since taking global section is a left exact functor, we have an injection for each $m$,
\[\Gamma (\sI(m),\P^n)\rightarrow \Gamma (\sO_{\P^n}(m)).\]

Let $I=\oplus_m \Gamma (\sI(m),\P^n)\subset \oplus_m\Gamma (\sO(m),\P^n) = k[x_0,\ldots,x_n]$. Is $\tilde{I}=\sI$?\footnote{I suspect that this $I$ is the same as the one constructed earlier by homogenising the generators of the affine opens, since the gluing maps on $\sO(m)$ are just mulplication by the $m$-th power of some $x_i$.}. [Discussed till here as of Spetember 4, 2020]

\subsection{Some more examples.}

\begin{enumerate}
	\item An example of a non-Noetherian scheme is $\Spec R[x_1,x_2,\ldots]$. 
	\item $V_{+}(x^2+y^2+z^2)$ over $\R$ and $\C$. Over $\C$, one have the following linear change of coordinates, $(x,y,z)\mapsto (x+iy,x-iy,iz)$. Then, $(x+iy) (x-iy) -z^2= x^2+y^2+(iz)^2$. So, this is the same as $V_+(uv-z^2)$, which is the (2-fold-)Veronese embedding of $\P^1$ in $\P^2$ given by $[x:y]\mapsto [x^2:xy:y^2]$. Similarly, the $d$-fold Veronese embedding is given by $[x:y]\mapsto [x^d:x^{d-1}y:\ldots:xy^{d-1}:y^d]$.
	\item Blow-up of $\A^2$ at the origin. This is described as 
	\[Y:=\lbrace ((x,y),[X:Y])\subset \A^2\times\P^1\; |\; xY=yX\rbrace\]
	This comes with a projection map $p: Y\rightarrow \A^2$. Observe that outside the origin this is an isomorphism, i.e, $p^{-1}(\A^1\setminus 0)=\A^1\setminus 0$, and $p^{-1}(0)=\P^1$. %(because everyone should know about blow-ups!)
	\item An example of a scheme without a closed point. This one is a involved but idea is as follows. There is a valuation on the polynomial ring (or power series ring?) in infinitely many variable, such that every non-maximal prime ideal is contained in a non-maximal prime ideal. Then, knocking off the maximal ideal gives a scheme without any closed point. This scheme is \textit{not} quasi-compact.
\end{enumerate}

\section{Morphisms}
As mentioned before, many of the properties of morphisms that we are interested in are ``globalised" versions of properties of ring maps. However, we have to first say what it means for morphism of schemes to be a local property. There are three kinds of local properties: local on the source, local on the target, local on the source and target. We will say what this means now:

\begin{definition}\label{locally-P-morphisms}
	Let $\mathcal{P}$ be a property of morphisms of schemes. Let $f:X\rightarrow Y$ be a morphism which satisfies $\mathcal{P}$. Then,
	\begin{enumerate}
		\item We say that $\mathcal{P}$ is \textit{affine-local on the target} if given any affine open cover $\lbrace V_i\rbrace$ of $Y$, $f:X\rightarrow Y$ has $\mathcal{P}$ if and only if the restriction $f: f^{-1}(V_i)\rightarrow V_i$ has $\mathcal{P}$ for each $i$.
		\item We say that $\mathcal{P}$ is \textit{affine-local on the source} if given any affine open cover $\lbrace U_i\rbrace$ of $X$, $X\rightarrow Y$ has $\mathcal{P}$ if and only if the composite $U_i\rightarrow Y$ has $\mathcal{P}$ for each $i$.
	\end{enumerate}
\end{definition}

Using \textit{affine communication lemma} one can then show that it suffices to check the above statements on single affine open cover.

An important maxim of Grothendieck was that instead of considering schemes in isolation, we should look at things relative to each other, i.e, everything should be seen as a propery of morphisms. Then properties of schemes should really be thought of as properties of morphisms $X\rightarrow\Spec \Z$ (or whatever base you are working over. For a lot of people it is the spectrum of a field). This is mostly true: many property of schemes can be turned into properties of morphisms of schemes\footnote{For example, affine opens form a basis for the topology on a scheme. Can this statment be ``relativised" to affine morphisms $X\rightarrow Y$? I don't know the answer.}.

\subsection{To be (or not to be) Noetherian} The discussion in \ref{PvlP} also applies to properties of morphisms, i.e., a morphism is said to be $P$ if it is locally $P$ and quasi-compact: a morphism is finite-type if it is locally finite type and quasi-compact; quasi-finite if it is locally quasi-finite and quasi-compact.

However, this is not true of finite presentation. A morphism is of finite presentation if it is locally of finite presentation, quasi-compact \textit{and quasi-separarted}. This is because. really, finite presentation is a condition to correct for non-Noetherian-ness over arbitrary bases. Note that for Noetherian schemes, a morphism (locally) of finite type is automatically (locally) of finite presentation. Furthermore, quasi-separatedness is automatic for Noetherian scheme. There is a very nice discussion on mathoverflow on this that I encourage you to look up\footnote{\href{https://mathoverflow.net/questions/36737/why-does-finitely-presented-imply-quasi-separated}{https://mathoverflow.net/questions/36737/why-does-finitely-presented-imply-quasi-separated}}. 

Examples:
\begin{enumerate}
	\item $x\mapsto x^2$ (more, generally $x^n$). This morphism is ramified at the origin (but unramified on $\A^1\setminus \lbrace 0\rbrace$), finitely presented, flat.
	\item A non-quasi-compact open immersion. $\Spec k[x_1,x_2,\ldots]\setminus\lbrace (x_1,x_2,\ldots)\rbrace\hookrightarrow \Spec k[x_1,x_2,\ldots]$ Polynomial ring in infinitely many variables and knock off the origin.
	\item A finite morphism. 
	\item A smooth morphism. A non-smooth morphism (nodal curve over $\A^1$).
	\item Open embeddings are locally of finite presentation\footnote{This is not true in perfectoid geometry, which is quite sad.}.
	\item Open embedding is \'{e}tale is fppf is fpqc.
\end{enumerate}


\section{Plan for September 11}

Doing the examples in section 1.9. Can add more examples. Some points to ponder:

\begin{enumerate}
	\item hilbert scheme of hypersurface.
	\item that curve in Hartshorne's deformation theory.
	\item $Hom(-,\P^n)$, $Hom(-,\A^n)$.
\end{enumerate}

Here's a tentative plan:
\begin{enumerate}
	\item Do functor of point of $\A^n$. Do functor of points of $\P^n$.
	\item Do those examples in 1.9.
	\item Then finish morphisms.
\end{enumerate}

Start with Amitsur's lemma next week.

%\bibliography{Mybib.bib}
\bibliographystyle{alpha}


\end{document}



