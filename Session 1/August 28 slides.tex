\documentclass[ignorenonframetext,t]{beamer}
\usepackage{amscd}
\usepackage{verbatim}

\usetheme{Frankfurt}

\setbeamerfont{block body}{size=\small}

\setbeamercolor{mycolor}{fg=white,bg=black}
\defbeamertemplate*{footline}{shadow theme}{%
	\leavevmode%
	\hbox{\begin{beamercolorbox}[wd=.5\paperwidth,ht=2.5ex,dp=1.125ex,leftskip=.3cm plus1fil,rightskip=.3cm]{author in head/foot}%
			\usebeamerfont{author in head/foot}\hfill\insertshorttitle\, - \insertsubtitle
		\end{beamercolorbox}%
		\begin{beamercolorbox}[wd=.4\paperwidth,ht=2.5ex,dp=1.125ex,leftskip=.3cm,rightskip=.3cm plus1fil]{title in head/foot}%
			\usebeamerfont{title in head/foot}\insertshortauthor\hfill%
		\end{beamercolorbox}%
		\begin{beamercolorbox}[wd=.1\paperwidth,ht=2.5ex,dp=1.125ex,leftskip=.3cm,rightskip=.3cm plus1fil]{mycolor}%
			\hfill\insertframenumber\,/\,\inserttotalframenumber
	\end{beamercolorbox}}%
	\vskip0pt%
}


%% LaTeX Definitions
%\newcounter{countup}

\newcommand{\rup}[1]{\lceil{#1}\rceil}
\newcommand{\rdown}[1]{\lfloor{#1}\rfloor}
\newcommand{\ilim}{\mathop{\varprojlim}\limits} % inverse limit
\newcommand{\dlim}{\mathop{\varinjlim}\limits}  % direct limit
\newcommand{\surj}{\twoheadrightarrow}
\newcommand{\inj}{\hookrightarrow}
\newcommand{\tensor}{\otimes}
\newcommand{\ext}{\bigwedge}
\newcommand{\Intersection}{\bigcap}
\newcommand{\Union}{\bigcup}
\newcommand{\intersection}{\cap}
\newcommand{\union}{\cup}

%%%%%%%%%%%%%%%%%%%%%%%%%%%%% new new commands :) %%%%%%%%%%%%%%%%
\newcommand{\supp}{{\rm Supp}}
\newcommand{\Exceptional}{{\rm Ex}}
\newcommand{\del}{\partial}
\newcommand{\delbar}{\overline{\partial}}
\newcommand{\boldphi}{\mbox{\boldmath $\phi$}}

%%%%%%%%%%%%%%%%%%%%%%%%%%%%%%%%%%%%%%%%%%%%%%%%%%%%%%%%%%%%%%%%%%%%%%%%%%%%%%

\newcommand{\udiv}{\underline{\Div}}

%%%%%%%%%%%%%%%%%

\newcommand{\Proj}{{\P roj}}
\newcommand{\sEnd}{{\sE nd}}
\newcommand{\mc}{\mathcal}
\newcommand{\mb}{\mathbb}
\newcommand{\an}{{\rm an}} 
\newcommand{\red}{{\rm red}}
\newcommand{\codim}{{\rm codim}}
\newcommand{\Dim}{{\rm dim}}
\newcommand{\rank}{{\rm rank}}
\newcommand{\Ker}{{\rm Ker  }}
\newcommand{\Pic}{{\rm Pic}}
\newcommand{\per}{{\rm per}}
\newcommand{\ind}{{\rm ind}}
\newcommand{\Div}{{\rm Div}}
\newcommand{\Hom}{{\rm Hom}}
\newcommand{\Aut}{{\rm Aut}}
\newcommand{\im}{{\rm im}}
\newcommand{\Spec}{{\rm Spec \,}}
\newcommand{\Sing}{{\rm Sing}}
\newcommand{\sing}{{\rm sing}}
\newcommand{\reg}{{\rm reg}}
\newcommand{\Char}{{\rm char}}
\newcommand{\Tr}{{\rm Tr}}
\newcommand{\Gal}{{\rm Gal}}
\newcommand{\Min}{{\rm Min \ }}
\newcommand{\Max}{{\rm Max \ }}
\newcommand{\Alb}{{\rm Alb}\,}
\newcommand{\Mat}{{\rm Mat}}
%\newcommand{\GL}{{\rm GL}\,}        % For the general linear group
\newcommand{\GL}{{\G\L}}
\newcommand{\Ho}{{\rm Ho}}
\newcommand{\ie}{{\it i.e.\/},\ }
\renewcommand{\iff}{\mbox{ $\Longleftrightarrow$ }}
\renewcommand{\tilde}{\widetilde}
% Skriptbuchstaben
\newcommand{\sA}{{\mathcal A}}
\newcommand{\sB}{{\mathcal B}}
\newcommand{\sC}{{\mathcal C}}
\newcommand{\sD}{{\mathcal D}}
\newcommand{\sE}{{\mathcal E}}
\newcommand{\sF}{{\mathcal F}}
\newcommand{\sG}{{\mathcal G}}
\newcommand{\sH}{{\mathcal H}}
\newcommand{\sI}{{\mathcal I}}
\newcommand{\sJ}{{\mathcal J}}
\newcommand{\sK}{{\mathcal K}}
\newcommand{\sL}{{\mathcal L}}
\newcommand{\sM}{{\mathcal M}}
\newcommand{\sN}{{\mathcal N}}
\newcommand{\sO}{{\mathcal O}}
\newcommand{\sP}{{\mathcal P}}
\newcommand{\sQ}{{\mathcal Q}}
\newcommand{\sR}{{\mathcal R}}
\newcommand{\sS}{{\mathcal S}}
\newcommand{\sT}{{\mathcal T}}
\newcommand{\sU}{{\mathcal U}}
\newcommand{\sV}{{\mathcal V}}
\newcommand{\sW}{{\mathcal W}}
\newcommand{\sX}{{\mathcal X}}
\newcommand{\sY}{{\mathcal Y}}
\newcommand{\sZ}{{\mathcal Z}}
% Sonderbuchstaben mit Doppellinie
\newcommand{\A}{{\mathbb A}}
\newcommand{\B}{{\mathbb B}}
\newcommand{\C}{{\mathbb C}}
\newcommand{\D}{{\mathbb D}}
\newcommand{\E}{{\mathbb E}}
\newcommand{\F}{{\mathbb F}}
\newcommand{\G}{{\mathbb G}}
\newcommand{\HH}{{\mathbb H}}
\newcommand{\I}{{\mathbb I}}
\newcommand{\J}{{\mathbb J}}
\newcommand{\M}{{\mathbb M}}
\newcommand{\N}{{\mathbb N}}
\renewcommand{\O}{{\mathbb O}}
\renewcommand{\P}{{\mathbb P}}
\newcommand{\Q}{{\mathbb Q}}
\newcommand{\R}{{\mathbb R}}
\newcommand{\T}{{\mathbb T}}
\newcommand{\U}{{\mathbb U}}
\newcommand{\V}{{\mathbb V}}
\newcommand{\W}{{\mathbb W}}
\newcommand{\X}{{\mathbb X}}
\newcommand{\Y}{{\mathbb Y}}
\newcommand{\Z}{{\mathbb Z}}
\newcommand{\Sh}{\sS h}
\newcommand{\deltaop}{\Delta^{op}(\sS h(Sm/\mathbf{k}))}
\newcommand{\pdeltaop}{\Delta^{op}(P\sS h(Sm/\mathbf{k}))}
%\newcommand{\psh}{\pi_0^{\text{\tiny pre}}}
\newcommand{\psh}{\pi_0}
\renewcommand{\k}{\mathbf{k}}

\newcommand{\colim}{{\rm colim \,}}
\newcommand{\DM}[2]{\mathbf{DM}_{#2}^{\mathit{eff}}(#1)}

\theoremstyle{definition}
\newtheorem{question}[theorem]{Question}




% Document information
\title[Moduli@IISERP]{Seminar on Moduli Theory}
\subtitle{Lecture 1}
\author{Neeraj Deshmukh}
\date{August 28, 2020}
%\address[IISERM]{Indian Institute of Science Education and Research, Mohali}

\begin{document}
	
	
\begin{frame}
\titlepage
\end{frame}

\begin{frame}
For the sake of completeness we begin by reviewing the definition of a locally ringed space.

\begin{definition}
	\label{definition-locally-ringed-space}
	Locally ringed spaces.
	\begin{enumerate}
		\item A {\it locally ringed space $(X, \mathcal{O}_X)$}
		is a pair consisting of a
		topological space $X$ and a sheaf of rings $\mathcal{O}_X$ all of whose stalks
		are local rings.
		\item Given a locally ringed space $(X, \mathcal{O}_X)$ we say that
		$\mathcal{O}_{X, x}$ is the {\it local ring of $X$ at $x$}.
		We denote $\mathfrak{m}_{X, x}$ or simply $\mathfrak{m}_x$
		the maximal ideal of $\mathcal{O}_{X, x}$. Moreover, the
		{\it residue field of $X$ at $x$} is the residue field
		$\kappa(x) = \mathcal{O}_{X, x}/\mathfrak{m}_x$.
		\item A {\it morphism of locally ringed spaces}
		$(f, f^\sharp) : (X, \mathcal{O}_X) \to (Y, \mathcal{O}_Y)$
		is a morphism of ringed spaces such that for all $x \in X$
		the induced ring map $\mathcal{O}_{Y, f(x)} \to \mathcal{O}_{X, x}$ is a
		local ring map.
	\end{enumerate}
\end{definition}
\end{frame}

\begin{frame}{Tilde construction}

We know that affine schemes are locally ringed spaces.
\end{frame}
we take $\Spec R$ with the zariski topology and for any principal open set $D(f)$ we assign the ring $R_f$. So, any ring $R$ produces the sheaf $\tilde{R}$ on $\Spec R$. This is called the tilde construction. (sanity check: if you can do this, then you should be able to construct a sheaf on $\Spec R$ for any $R$-module $M$).

\begin{frame}
\begin{definition}
	A {\it scheme} is a locally ringed space with the property that
	every point has an open neighbourhood which is an affine scheme.
	A {\it morphism of schemes} is a morphism of locally
	ringed spaces. The category of schemes will be denoted
	$Sch$.
\end{definition}

\end{frame}

\begin{frame}
\begin{definition}
	Let $(X,\mathcal{O}_X)$ be a scheme. A sheaf of modules on $X$ is a sheaf $\mathcal{F}$ on $X$ such that for every open set $U$, $\mathcal{F}(U)$ is an $\mathcal{O}_X(U)$-module. We say that a sheaf of modules $\mathcal{F}$ is \textit{quasi-coherent} if for every affine open $U\simeq \Spec(R)$, the sheaf $\mathcal{F}|_U$ on $U$ is of the form $\tilde{M}$ for some $R$-module $M$.
\end{definition}

\end{frame}


Make special note of the next lemma. This basically lets us reduce problems about schemes to statement about affine schemes (ergo, ring theory), whenever the problem at hand is of a \textit{local} nature. Ravi Vakil calls this \textit{affine communication lemma}.

\begin{frame}{Affine Communication Lemma}

\begin{lemma}
	\label{lemma-locally-P}
	Let $X$ be a scheme. Let $P$ be a local property of rings.
	The following are equivalent:
	\begin{enumerate}
		\item The scheme $X$ is locally $P$.
		\item For every affine open $U \subset X$ the property
		$P(\mathcal{O}_X(U))$ holds.
		\item There exists an affine open covering $X = \bigcup U_i$ such that
		each $\mathcal{O}_X(U_i)$ satisfies $P$.
		\item There exists an open covering $X = \bigcup X_j$
		such that each open subscheme $X_j$ is locally $P$.
	\end{enumerate}
	Moreover, if $X$ is locally $P$ then every open subscheme
	is locally $P$.
\end{lemma}
\end{frame}

This is how commutative algebra meets geometry. Often, the properties that we want to consider are ``globalised" versions of statements about rings. Some examples, normality, reduced, etc.\footnote{You can also ``globalise" morphisms of rings, but now you have two schemes to work locally on. We'll do this soon.}



Examples:

\begin{frame}{Examples}

Two ways of gluing $\A^1\setminus\lbrace 0\rbrace$: $x\mapsto x$ or $x\mapsto 1/x$.
	\begin{enumerate}
		\item Double origin: What are global sections? what are quasi-coherent sheaves?
		\item $\P^1$: What are global sections?
	\end{enumerate}
\end{frame}

\begin{frame}{More examples}
A normal scheme, a reduced scheme and a Noetherian scheme.
\end{frame}

\begin{frame}{More Examples}
 Something non-noetherian: $\Spec R[x_1,x_2,\ldots]$.
\end{frame}



\begin{frame}{Line Bundles on $\P^1$}
Locally on an affine open, this should be a free module of rank one. Let's contruct one such line bundle (non-trivial, of course).

\end{frame}

 There are two open sets, $D(x)$ and $D(y)$, on these our line bundle looks like $k[x]$ and $k[y]$, respectively. Now, how do they glue on $k[x,1/x]\simeq k[y,1/y]$? Let's use the map which sends $\phi(1):f(x)\mapsto f(x)y$, since $y$ is $1/x$ in this ring, we see that the global sections are linear polynomials. You construct such a map $\phi(n)$ for every power of $y$. That will give you degree $n$ monomials. These line bundles are called $\sO(n)$'s. Playing around with the algebra of the maps $\phi(n)$ a little will that these line bundles satisfy relations like $\sO(n)\otimes\sO(m)\simeq \sO(m+n)$, and admit duals which are denote by $\sO(-n)$.
\footnote{The line bundle $\sO(1)$ is important. To say that a variety is projective, we need to show that something like $\sO(1)$ lives on it. Actually, some lesser works, but we will come back to this later.}

\begin{frame}{A slightly more involved scheme: $\P^n$.}
	Let $D(x_i):= k[x_{0/i},x_{1/i},\ldots,\hat{x_{i/i}},\ldots,x_{n/i}]$. We have a maps $\phi_{ij}:D(x_i)_{x_j}\rightarrow D(x_j)_{x_i}$,\, given by $x_{k/i}\mapsto x_{k/j}$.
\end{frame}
\footnote{If you have seen the construction of Grassmannians as smooth manifolds, the same construction also goes through in algebraic geometry.}

If we are over an algebraically closed field, then closed points of $\P^n$ can be written in \textit{homogeneous coordinates} as $[x_0:x_1:\ldots:x_n]$, where two such coordinates are the same if they differ by scalar multiplication.

\end{frame}












More examples:
\begin{frame}{More examples}
\begin{enumerate}
	\item $V_{+}(x^2+y^2+z^2)$ over $\R$ and $\C$. 
	\item Blow-up of $\A^2$ at the origin.
	\item An example of a scheme without a closed point.
\end{enumerate}
\end{frame}
 Over $\C$, one have the following linear change of coordinates, $(x,y,z)\mapsto (x+iy,x-iy,iz)$. Then, $(x+iy) (x-iy) -z^2= x^2+y^2+(iz)^2$. So, this is the same as $V_+(uv-z^2)$, which is the (2-fold-)Veronese embedding of $\P^1$ in $\P^2$ given by $[x:y]\mapsto [x^2:xy:y^2]$. Similarly, the $d$-fold Veronese embedding is given by $[x:y]\mapsto [x^d:x^{d-1}y:\ldots:xy^{d-1}:y^d]$.
 

%\section{Morphism}
As mentioned before, many of the properties of morphisms that we are interested in are ``globalised" versions of properties of ring maps. However, we have to first say what it means for morphism of schemes to be a local property. There are three kinds of local properties: local on the source, local on the target, local on the source and target. We will say what this means now:

\begin{frame}
\begin{definition}
	Let $\mathcal{P}$ be a property of morphisms of schemes. Let $f:X\rightarrow Y$ be a morphism which satisfies $\mathcal{P}$. Then,
	\begin{enumerate}
		\item We say that $\mathcal{P}$ is \textit{affine-local on the target} if given any affine open cover $\lbrace V_i\rbrace$ of $Y$, $f:X\rightarrow Y$ has $\mathcal{P}$ if and only if the restriction $f: f^{-1}(V_i)\rightarrow V_i$ has $\mathcal{P}$ for each $i$.
		\item We say that $\mathcal{P}$ is \textit{affine-local on the source} if given any affine open cover $\lbrace U_i\rbrace$ of $X$, $X\rightarrow Y$ has $\mathcal{P}$ if and only if the composite $U_i\rightarrow Y$ has $\mathcal{P}$ for each $i$.
	\end{enumerate}
\end{definition}

Using \textit{affine communication lemma} one can then show that it suffices to check the above statements on single affine open cover.

\end{frame}

An important maxim of Grothendieck was that instead of considering schemes in isolation, we should look at things relative to each other, i.e, everything should be seen as a propery of morphisms. This is mostly true: every property of schemes can be turned into a property of morphisms of schemes.

Examples:

\begin{frame}
	Something flat, something finitely presented/finite type, something finite.
\end{frame}

\begin{frame}
	Not all properties are like this. For example, separatedness, properness, quasi-compactness, etc.
\end{frame}

\begin{frame}
	Something ramified, something smooth, something singular.
\end{frame}

\begin{enumerate}
	\item $x\mapsto x^2$ (more, generally $x^n$). This covers ramified, finitely presented, flat.
	\item A non-quasi-compact, open-immersion. Polynomial ring in infinitely many variables and knock off the origin.
	\item A finte morphism. 
	\item A smooth morphism. A non-smooth morphism (nodal curve over $\A^1$).
\end{enumerate}

Open embeddings are locally finite presentation.\footnote{This is not true in perfectoid geometry, which is quite sad.}

Open embedding is \'{e}tale is fppf is fpqc


\bibliography{Mybib.bib}
\bibliographystyle{alpha}

\end{document}



