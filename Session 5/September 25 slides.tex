\documentclass[ignorenonframetext,t]{beamer}
\usepackage{amscd}
\usepackage{verbatim}

\usetheme{Frankfurt}

\setbeamerfont{block body}{size=\small}

\setbeamercolor{mycolor}{fg=white,bg=black}
\defbeamertemplate*{footline}{shadow theme}{%
	\leavevmode%
	\hbox{\begin{beamercolorbox}[wd=.5\paperwidth,ht=2.5ex,dp=1.125ex,leftskip=.3cm plus1fil,rightskip=.3cm]{author in head/foot}%
			\usebeamerfont{author in head/foot}\hfill\insertshorttitle\, - \insertsubtitle
		\end{beamercolorbox}%
		\begin{beamercolorbox}[wd=.4\paperwidth,ht=2.5ex,dp=1.125ex,leftskip=.3cm,rightskip=.3cm plus1fil]{title in head/foot}%
			\usebeamerfont{title in head/foot}\insertshortauthor\hfill%
		\end{beamercolorbox}%
		\begin{beamercolorbox}[wd=.1\paperwidth,ht=2.5ex,dp=1.125ex,leftskip=.3cm,rightskip=.3cm plus1fil]{mycolor}%
			\hfill\insertframenumber\,/\,\inserttotalframenumber
	\end{beamercolorbox}}%
	\vskip0pt%
}


%% LaTeX Definitions
%\newcounter{countup}

\newcommand{\rup}[1]{\lceil{#1}\rceil}
\newcommand{\rdown}[1]{\lfloor{#1}\rfloor}
\newcommand{\ilim}{\mathop{\varprojlim}\limits} % inverse limit
\newcommand{\dlim}{\mathop{\varinjlim}\limits}  % direct limit
\newcommand{\surj}{\twoheadrightarrow}
\newcommand{\inj}{\hookrightarrow}
\newcommand{\tensor}{\otimes}
\newcommand{\ext}{\bigwedge}
\newcommand{\Intersection}{\bigcap}
\newcommand{\Union}{\bigcup}
\newcommand{\intersection}{\cap}
\newcommand{\union}{\cup}

%%%%%%%%%%%%%%%%%%%%%%%%%%%%% new new commands :) %%%%%%%%%%%%%%%%
\newcommand{\supp}{{\rm Supp}}
\newcommand{\Exceptional}{{\rm Ex}}
\newcommand{\del}{\partial}
\newcommand{\delbar}{\overline{\partial}}
\newcommand{\boldphi}{\mbox{\boldmath $\phi$}}

%%%%%%%%%%%%%%%%%%%%%%%%%%%%%%%%%%%%%%%%%%%%%%%%%%%%%%%%%%%%%%%%%%%%%%%%%%%%%%

\newcommand{\udiv}{\underline{\Div}}

%%%%%%%%%%%%%%%%%

\newcommand{\Proj}{{\P roj}}
\newcommand{\sEnd}{{\sE nd}}
\newcommand{\mc}{\mathcal}
\newcommand{\mb}{\mathbb}
\newcommand{\an}{{\rm an}} 
\newcommand{\red}{{\rm red}}
\newcommand{\codim}{{\rm codim}}
\newcommand{\Dim}{{\rm dim}}
\newcommand{\rank}{{\rm rank}}
\newcommand{\Ker}{{\rm Ker  }}
\newcommand{\Pic}{{\rm Pic}}
\newcommand{\per}{{\rm per}}
\newcommand{\ind}{{\rm ind}}
\newcommand{\Div}{{\rm Div}}
\newcommand{\Hom}{{\rm Hom}}
\newcommand{\Aut}{{\rm Aut}}
\newcommand{\im}{{\rm im}}
\newcommand{\Spec}{{\rm Spec \,}}
\newcommand{\Sing}{{\rm Sing}}
\newcommand{\sing}{{\rm sing}}
\newcommand{\reg}{{\rm reg}}
\newcommand{\Char}{{\rm char}}
\newcommand{\Tr}{{\rm Tr}}
\newcommand{\Gal}{{\rm Gal}}
\newcommand{\Min}{{\rm Min \ }}
\newcommand{\Max}{{\rm Max \ }}
\newcommand{\Alb}{{\rm Alb}\,}
\newcommand{\Mat}{{\rm Mat}}
%\newcommand{\GL}{{\rm GL}\,}        % For the general linear group
\newcommand{\GL}{{\G\L}}
\newcommand{\Ho}{{\rm Ho}}
\newcommand{\ie}{{\it i.e.\/},\ }
\renewcommand{\iff}{\mbox{ $\Longleftrightarrow$ }}
\renewcommand{\tilde}{\widetilde}
% Skriptbuchstaben
\newcommand{\sA}{{\mathcal A}}
\newcommand{\sB}{{\mathcal B}}
\newcommand{\sC}{{\mathcal C}}
\newcommand{\sD}{{\mathcal D}}
\newcommand{\sE}{{\mathcal E}}
\newcommand{\sF}{{\mathcal F}}
\newcommand{\sG}{{\mathcal G}}
\newcommand{\sH}{{\mathcal H}}
\newcommand{\sI}{{\mathcal I}}
\newcommand{\sJ}{{\mathcal J}}
\newcommand{\sK}{{\mathcal K}}
\newcommand{\sL}{{\mathcal L}}
\newcommand{\sM}{{\mathcal M}}
\newcommand{\sN}{{\mathcal N}}
\newcommand{\sO}{{\mathcal O}}
\newcommand{\sP}{{\mathcal P}}
\newcommand{\sQ}{{\mathcal Q}}
\newcommand{\sR}{{\mathcal R}}
\newcommand{\sS}{{\mathcal S}}
\newcommand{\sT}{{\mathcal T}}
\newcommand{\sU}{{\mathcal U}}
\newcommand{\sV}{{\mathcal V}}
\newcommand{\sW}{{\mathcal W}}
\newcommand{\sX}{{\mathcal X}}
\newcommand{\sY}{{\mathcal Y}}
\newcommand{\sZ}{{\mathcal Z}}
% Sonderbuchstaben mit Doppellinie
\newcommand{\A}{{\mathbb A}}
\newcommand{\B}{{\mathbb B}}
\newcommand{\C}{{\mathbb C}}
\newcommand{\D}{{\mathbb D}}
\newcommand{\E}{{\mathbb E}}
\newcommand{\F}{{\mathbb F}}
\newcommand{\G}{{\mathbb G}}
\newcommand{\HH}{{\mathbb H}}
\newcommand{\I}{{\mathbb I}}
\newcommand{\J}{{\mathbb J}}
\newcommand{\M}{{\mathbb M}}
\newcommand{\N}{{\mathbb N}}
\renewcommand{\O}{{\mathbb O}}
\renewcommand{\P}{{\mathbb P}}
\newcommand{\Q}{{\mathbb Q}}
\newcommand{\R}{{\mathbb R}}
\newcommand{\T}{{\mathbb T}}
\newcommand{\U}{{\mathbb U}}
\newcommand{\V}{{\mathbb V}}
\newcommand{\W}{{\mathbb W}}
\newcommand{\X}{{\mathbb X}}
\newcommand{\Y}{{\mathbb Y}}
\newcommand{\Z}{{\mathbb Z}}
\newcommand{\Sh}{\sS h}
\newcommand{\deltaop}{\Delta^{op}(\sS h(Sm/\mathbf{k}))}
\newcommand{\pdeltaop}{\Delta^{op}(P\sS h(Sm/\mathbf{k}))}
%\newcommand{\psh}{\pi_0^{\text{\tiny pre}}}
\newcommand{\psh}{\pi_0}
\renewcommand{\k}{\mathbf{k}}

\newcommand{\colim}{{\rm colim \,}}
\newcommand{\DM}[2]{\mathbf{DM}_{#2}^{\mathit{eff}}(#1)}

\theoremstyle{definition}
\newtheorem{question}[theorem]{Question}




% Document information
\title[Moduli@IISERP]{Seminar on Moduli Theory}
\subtitle{Lecture 5}
\author{Neeraj Deshmukh}
\date{September 25, 2020}
%\address[IISERM]{Indian Institute of Science Education and Research, Mohali}

\begin{document}
	
	
\begin{frame}
\titlepage
\end{frame}

\begin{frame}{Last Week}
\begin{enumerate}
	\item ``Affine Communication" for morphisms.
	\item Various kinds of morphisms.
	\item Morphisms to $\P^n$.
\end{enumerate}
\end{frame}


\begin{frame}
	A \textit{presheaf} is a functor. Representable functors and Yoneda.
\end{frame}

\begin{frame}
	Three representable and one non-representable functors of $\mathit{Sch}$
\end{frame}


\begin{frame}
	Representable morphisms of functors. (Morphisms of representable functors are always representable!)
\end{frame}

\begin{frame}
	A useful lemma about the diagonal
	\begin{lemma}
		\label{lemma-representable-diagonal}
		Let $\mathcal{C}$ be a category.
		Let $F : \mathcal{C}^{opp} \to \textit{Sets}$ be a functor.
		Assume $\mathcal{C}$ has products of pairs of objects and fibre products.
		The following are equivalent:
		\begin{enumerate}
			\item the diagonal $\Delta : F \to F \times F$ is representable,
			\item for every $U$ in $\mathcal{C}$,
			and any $\xi \in F(U)$ the map $\xi : h_U \to F$ is representable,
			\item for every pair $U, V$ in $\mathcal{C}$
			and any $\xi \in F(U)$, $\xi' \in F(V)$ the fibre product
			$h_U \times_{\xi, F, \xi'} h_V$ is representable.
		\end{enumerate}
	\end{lemma}
\end{frame}



In this section, we will collect all the ``sheafy jargon"\footnote{This may or may not be used in the sequel.} As I have mentioned before, modern Moduli theory is set in this language. Hence, it is a good idea to review some key points. More details can be found in Vistoli's notes on Descent (Chapter 1 of FGA Explained, \cite{FGAExplained}) or the Stacks project \cite{stacks-project}.

\subsection{A Commutative Algebra Detour.}
We begin with the following lemma which we will use at the end of this section to prove Theorem \ref{fpqc-representable}.

\begin{lemma}[Amitsur's Lemma]\label{Amitsur}
	Let $f: A\rightarrow B$ be a faithfully flat ring map. Then the following sequence of $A$-modules is exact:
	\begin{equation}\label{fpqc-modules}
	0\rightarrow A\overset{f}{\rightarrow} B \overset{e_1-e_2}{\rightarrow} B\otimes_A B.
	\end{equation}
	Here, $e_1$and $e_2$ are the inclusions $b\otimes 1$ and $1\otimes b$, respectively.
\end{lemma}
\begin{proof}
	Note that since $f$ is faithfully flat, it is injective (why?), and that $(e_1-e_2)\circ f=0$. So, we only need to check that $Ker(e_1-e_2)\subseteq Im(f)$.
	
	Let us first consider the special case when $f:A\rightarrow B$ is a retract, i.e, there exists a $g:B\rightarrow A$ such that $g\circ f = id_A$. Now, take a $b\in Ker(e_1-e_2)$, so that  $b\otimes 1 = 1\otimes b \in B\otimes_A B$. The retraction $g$ gives us a map $B\otimes_A B\overset{g\otimes id}{\rightarrow} A\otimes_A B\simeq B$. Applying this to the previous equality gives, $g(b)\otimes 1=1\otimes b$, which is the same as $g(b)\cdot 1=b$. Now, note that $f(g(b)\cdot 1)=g(b)\cdot f(1)=b$, since $f$ is an also $A$-module homomorphism, showing exactness.
	
	Now, observe that tensoring (\ref{fpqc-modules}) with $\otimes_A B$ gives a sequence,
	\[0\longrightarrow A\otimes_A B\overset{f\otimes id}{\longrightarrow} B\otimes_A B \overset{(e_1-e_2)\otimes id}{\longrightarrow} B\otimes_A B\otimes_A B.\]
	The map $f\otimes id$ is also a faithfully flat ring map. Moreover, it has a retraction\footnote{All these statements about retractions are just a dual way for saying that the map $\Spec B\rightarrow \Spec A$ has a section.} $g: B\otimes_A B\rightarrow A\otimes_A B$ given by $b\otimes b'\mapsto 1\otimes bb'$. Thus, this sequence is exact. Now, faithful flatness implies the exactness of (\ref{fpqc-modules}).
\end{proof}


\begin{remark}
	Observe that the above lemma holds for any faithfully flat $A$-module $M$, i.e, the sequence of $A$-modules,
	\[0\longrightarrow M\overset{id\otimes f}{\longrightarrow} M\otimes_A B \overset{id\otimes (e_1-e_2)}{\longrightarrow} M\otimes_A B\otimes_A B,\]
	is exact. You may ask what is happening at the map $id\otimes (e_1-e_2)$. To answer this, note that we can extend this above sequence to higher tensor powers,
	\[0\longrightarrow M\overset{id\otimes f}{\longrightarrow} M\otimes_A B \overset{id\otimes (e_1-e_2)}{\longrightarrow} M\otimes_A B\otimes_A B\rightarrow M\otimes_A B\otimes_A B\otimes B\rightarrow\ldots.\]
	The map 
	\[M\otimes_A B\otimes_A B\rightarrow M\otimes_A B\otimes_A B\otimes B\]
	can be described on pure tensors as the alternating sum,
	\[m\otimes b\otimes b'\mapsto m\otimes b\otimes b'\otimes 1- m\otimes b\otimes 1\otimes b + m\otimes 1\otimes b\otimes b'.\]
	The higher tensor powers are described similarly as alternating sums\footnote{This is can be stated more cleanly in terms of (co)simplicial objects. All we doing is taking the chain complex associated to the cosimplicial diagram of $M$.}. Then, this extended chain complex is exact. Essentially, this characterises effective descent for fpqc morphisms (see \cite[Tag 023F]{stacks-project} for more on this).
\end{remark}

\begin{frame}
We will now discuss some ``sheafy jargon".

\begin{definition}
	\label{definition-family-morphisms-fixed-target}
	Let $\mathcal{C}$ be a category. A {\it family of morphisms with fixed target} in $\mathcal{C}$ is	given by an object $U \in \text{Ob}(\mathcal{C})$, a set $I$ and
	for each $i\in I$ a morphism $U_i \to U$ of $\mathcal{C}$ with target $U$.
	We use the notation $\{U_i \to U\}_{i\in I}$ to indicate this.
\end{definition}



\begin{definition}
	\label{definition-site}
	A {\it site} is given by a category $\mathcal{C}$ and a set
	$\text{Cov}(\mathcal{C})$ of families of morphisms with fixed target
	$\{U_i \to U\}_{i \in I}$, called {\it coverings of $\mathcal{C}$},
	satisfying the following axioms
	\begin{enumerate}
		\item If $V \to U$ is an isomorphism then $\{V \to U\} \in
		\text{Cov}(\mathcal{C})$.
		\item If $\{U_i \to U\}_{i\in I} \in \text{Cov}(\mathcal{C})$ and for each
		$i$ we have $\{V_{ij} \to U_i\}_{j\in J_i} \in \text{Cov}(\mathcal{C})$, then
		$\{V_{ij} \to U\}_{i \in I, j\in J_i} \in \text{Cov}(\mathcal{C})$.
		\item If $\{U_i \to U\}_{i\in I}\in \text{Cov}(\mathcal{C})$
		and $V \to U$ is a morphism of $\mathcal{C}$ then $U_i \times_U V$
		exists for all $i$ and
		$\{U_i \times_U V \to V \}_{i\in I} \in \text{Cov}(\mathcal{C})$.
	\end{enumerate}
\end{definition}

\end{frame}

\begin{frame}
	The sheaf condition and the category of sheaves
\end{frame}




A category  satisfying the above definition is said to be equipped with a \textit{Grothendieck topology}. Thus, a site is a category with a Grothendieck topology.

\subsection{Various Topologies on Affine Schemes.} Let $\mathit{Ring}^{op}$ or $\mathit{Aff}$ denote the cateogry of affine schemes. We can define the following Grothendieck topologies\footnote{This list is by no means exhaustive. Some notable exculsions are the \textit{Nisnevich} and \textit{smooth} topologies.} on it:

\begin{enumerate}
	\item[\textit{Zariski:}]  Coverings are families of ring maps $\{ R \overset{f_i}{\to} R_{r_i}\}_{i \in I}$, where $R_{r_i}$ is the localisation with respect to $r_i\in R$ such that $(\{r_i\}_{i\in I})= R$.
	\item[\textit{\'{E}tale}:] Covering are families of \'{e}tale ring maps $\{ R \overset{f_i}{\to} R_i\}_{i \in I}$ such that the product $\prod_i f_i$ is faithfully flat.
	\item[\textit{fppf}:] Coverings are families of ring maps $\{ R \overset{f_i}{\to} R_i\}_{i \in I}$ which are flat, finite presentation and such that the product $\prod_i f_i$ is faithfully flat. (fppf: fid\`{e}lement plat et de pr\'{e}sentation fini)
	\item[\textit{fpqc}:] Coverings are families of morphisms $\{ R \overset{f_i}{\to} R_i\}_{i \in I}$ such that each $f_i$ is faithfully flat. (fpqc: fid\`{e}lement plat et quasi-compacte)
\end{enumerate}

\begin{frame}{Various topologies on Affine schemes.}
	\begin{enumerate}
		\setlength{\itemsep}{30pt}
		\item[\textit{Zariski:}]  
		\item[\textit{\'{E}tale}:]
		\item[\textit{fppf}:] 
		\item[\textit{fpqc}:] 
	\end{enumerate}	
\end{frame}

\begin{frame}{Various toplogies on schemes}
Zariski, \'{e}tale and fppf are defined almost analogously for schemes.\\
\vspace{1in}
Kleiman's trick for fpqc morphisms:
\end{frame}
Note that all the rings maps in the topologies above are used to define local properties of morphisms of schemes which are also stable under base change. Hence, we can extend these toplogies to schemes.

\subsection{Various Topologies on Schemes.}
Let $(\mathit{Sch})$ be the category of schemes. We will now ``globalise" the topologies defined for affine schemes to schemes. Note, however, the important change that needs to be made to get the correct notion of \textit{fpqc} coverings:


\begin{enumerate}
	\item[\textit{Zariski:}]  Coverings are families of open immersions $\{ U_i \overset{f_i}{\to} U\}_{i \in I}$ such that $U= \bigcup_i f(U_i)$.
	\item[\textit{\'{E}tale:}] Covering are families of \'{e}tale morphisms $\{ U_i \overset{f_i}{\to} U\}_{i \in I}$ such that $U= \bigcup_i f(U_i)$.
	\item[\textit{fppf:}] Coverings are families of morphisms $\{ U_i \overset{f_i}{\to} U\}_{i \in I}$ such that each $f_i$ is flat, locally of finite presentation and $U= \bigcup_i f(U_i)$.
	\item[\textit{fpqc:}] Coverings are families of morphisms $\{ U_i \overset{f_i}{\to} U\}_{i \in I}$ such that each $f_i$ is flat and for every affine open $V \subset U$ there exist quasi-compact opens $V_i \subset U_i$ which are almost all empty, such that $V = \bigcup f_i(V_i)$.
\end{enumerate}

\begin{frame}

\begin{lemma}
	We have the follow inclusion of topologies:
	\[\textit{Zariski}\;\subset\; \textit{\'{E}tale}\;\subset\; \textit{fppf}\;\subset\; \textit{fpqc}.\]
\end{lemma}
\end{frame}

In light of the above remark, note that if we simply define fpqc coverings to be morphisms of schemes which are faithfully flat and quasi-compact, then Zariski covers would no longer be fpqc coverings.





Let $S$ be a scheme. Let $(\mathit{Sch}/S)$ be the category of schemes over $S$. $\Ob(\mathit{Sch}/S)$ are given by morphisms $X\rightarrow S$ and morphisms are $S$-morphisms, i.e, commuting diagrams,
\begin{center}
	\begin{tikzcd}
	X\arrow[rd]\arrow[rr]& & Y\arrow[ld]\\
	&S&
	\end{tikzcd}
\end{center}
Similarly, we will denote by $(\mathit{Aff}/S)$, the category of affine morphisms to $S$.\\

The next two definitions capture some general notions about sites that good to know but not really essential.

\noindent\textbf{Notation:} For the next definition we will adopt the following terminology: Let $\mathcal{P}$ be a property of morphisms of schemes which is stable under composition and base change, and includes isomorphisms. A \textit{family of $\mathcal{P}$-morphisms} is collection of morphisms $\{U_i \overset{f_i}{\to} U\}_{i\in I}$ such that each $f_i\in \mathcal{P}$ and is jointly surjective, i.e, $U = \bigcup f_i(U_i)$. Since $\mathcal{P}$ is stable under composition and base-change and includes isomorphisms, it is easy to see that these famililes will define a topology on $\mathit{Sch}$, which we will denote by $\tau$.


\begin{definition}[Big site, Small site]
	Let $(\mathit{Sch}/S)$ be the category of schemes over $S$ and let $\tau$ be a topology on schemes which comes from families of $\mathcal{P}$-morphisms. We have the following conventions for topologies on $(\mathit{Sch}/S)$:
	\begin{enumerate}
		\item[\textsc{Big site:}] For the topology $\tau$, we denote the site by $(\mathit{Sch}/S)_{\tau}$ and call it the \textit{Big $\tau$ site}.
		\item[\textsc{Small site:}] Consider the full subcategory of $(\mathit{Sch}/S)$ whose objects are $f: X\rightarrow S$ such that $f\in\mathcal{P}$. We define coverings for this subcategory as families of $\mathcal{P}$-morphisms which are jointly surjective. We denote this site by $S_{\tau}$ and call it the \textit{Small $\tau$ site}.
	\end{enumerate}
\end{definition}
For example, we have the big and small Zariski sites, the big and small \'{e}tale sites etc.\footnote{People mostly talk about small sites only in the Zariski and \'{e}tale topologies (and the Nisnevich topology if you are doing $\A^1$-homotopy theory)} The following is another notion which we will probably never use in this seminar\footnote{Unless we start discussing algebraic stacks at some point!}.


\begin{definition}[$\mathcal{P}$-$\mathcal{Q}$ site]
	Let $\mathcal{P}$ and $\mathcal{Q}$ be classes of morphisms that are stable under base change and composition, and contain isomorphisms. Then, the $\mathcal{P}$-$\mathcal{Q}$ site of $S$ is defined as follows: the objects are $S$-schemes, $f:X\rightarrow S$ such that $f\in \mathcal{P}$ and coverings are families of $\mathcal{Q}$-morphisms $\{U_i \overset{f_i}{\to} U\}_{i\in I}$ over $S$ which are jointly surjective.
\end{definition}

Notable examples are the smooth-\'{e}tale\footnote{also called the lisse-\'{e}tale site. Lisse is french for smooth.} and the flat-fppf sites.

\subsection{Presheaves and the Yoneda Embedding.}

\begin{definition}
	Let $\mathcal{C}$ be a category. A \textit{presheaf of sets} or simply a \textit{presheaf} is a functor
	\[F:\mathcal{C}^{opp}\rightarrow \mathit{Sets}.\]
	We denote the cateogory of all presheaves by $\mathit{Psh}(\mathcal{C})$.
\end{definition}

\begin{example}[Functor of points]
	\label{example-hom-functor}
	
	For any $U\in \Ob(\mathcal{C})$ there is a
	functor
	$$
	\begin{matrix}
	h_U & : & \mathcal{C}^{opp}
	&
	\longrightarrow
	&
	\textit{Sets} \\
	& &
	X
	&
	\longmapsto
	&
	\textit{Mor}_\mathcal{C}(X, U)
	\end{matrix}
	$$
	which takes an object $X$ to the set
	$\textit{Mor}_\mathcal{C}(X, U)$. In other words $h_U$ is a presheaf.
	Given a morphism $f : X\to Y$ the corresponding map
	$h_U(f) :  \textit{Mor}_\mathcal{C}(Y, U)\to \textit{Mor}_\mathcal{C}(X, U)$
	takes $\phi$ to $\phi\circ f$. It is called the {\it representable presheaf} associated to $U$.
	If $\mathcal{C}$ is the category of schemes this functor is
	sometimes referred to as the
	\emph{functor of points} of $U$.
\end{example}

Note that given a morphism $\phi : U \to V$ in $\mathcal{C}$ we get a
corresponding natural transformation of functors $h(\phi) : h_U \to h_V$
defined by composing with the morphism $U \to V$. This turns
composition of morphisms in $\mathcal{C}$ into composition of
transformations of functors. In other words we get a functor
$$
h :
\mathcal{C}
\longrightarrow
\textit{PSh}(\mathcal{C}).
$$

The following lemma says that $h$ is a fully faithful embedding.

\begin{lemma}[Yoneda lemma]
	\label{lemma-yoneda}
	Let $U, V \in \Ob(\mathcal{C})$.
	Given any morphism of functors $s : h_U \to h_V$
	there is a unique morphism $\phi : U \to V$
	such that $h(\phi) = s$. In other words the
	functor $h$ is fully faithful. More generally,
	given any contravariant functor $F$ and any object
	$U$ of $\mathcal{C}$ we have a natural bijection
	$$
	\mathit{Mor}_{\textit{PSh}(\mathcal{C})}(h_U, F) \longrightarrow F(U),
	\quad
	s \longmapsto s_U(\text{id}_U).
	$$
\end{lemma}

\begin{definition}
	\label{definition-representable-functor}
	A presheaf $F : \mathcal{C}^{opp}\to \textit{Sets}$ is said
	to be {\it representable} if it is isomorphic to the functor of
	points $h_U$ for some object $U$ of $\mathcal{C}$.
\end{definition}

Seen one way, the basic objective in moduli theory is to show representability of various presheaves. We will often confuse a scheme $X$ with its associated functor $h_X$.

\begin{example}[Three representable functors] We will now describe three important functors which are representable\footnote{Convince yourself that all these assignments define presheaves of sets.}.
	\begin{enumerate}
		\item $X \mapsto \Gamma (X,\sO_X)^{\oplus n}$. This is same as giving a ring map $\Z[x_1,\ldots,x_n]\rightarrow \Gamma (X,\sO_X)^{\oplus n}$. But such a ring map corresponds a morphism to $\A_{\Z}^n$.
		\item $X \mapsto \Gamma (X,\sO_X)^{\times}$. A similar argument as above tells us that these are morphisms to $\A_{\Z}^1\setminus \{0\}:= \Spec \Z[t,t^{-1}]$.
		\item $X \mapsto \{ (\sL,s)\;|\; \text{$\sL$ is a line bundle on $X$ with a surjection}\; p:\sO_X^{n+1}\twoheadrightarrow \sL\}$. We have already seen in \ref{functor-of-Pn} that this is $\mathit{Hom}(-,\P^n)$.
	\end{enumerate}
\end{example}


\begin{example}[A non-representable functor] Some moduli of curves. Let $k$ be an algebraically closed field. Consider the functor $F$ which assigns to any $k$-scheme a conic in $\P^2$ upto isomorphism. That is, 
	\begin{align*}
	F: (\mathit{Sch}/k)^{opp} & \rightarrow \mathit{Sets}\\
	S & \mapsto \left\{
	\begin{tikzcd}[cramped, row sep =2ex, column sep=4ex ,ampersand replacement=\&]
	Z\arrow[r,hook,"\phi"]\arrow[d,"\pi"] \& \P_S^2\\
	S
	\end{tikzcd}\right\} \Bigg/\!\sim
	\end{align*}
	where $\phi$ is a closed embedding, and $\pi$ is flat and its geometric fibres are conics, i.e, for any geometric point $\Spec L\rightarrow S$\footnote{and $L$ is (separably) algebraically closed.} the base change $Z_L$ is a conic in $\P_L^2$. We will show that this $F$ is not representable.
	
	For if it were representable, then there would exist a scheme $X$ such that $F\simeq h_X$. Moreover, for any $k$-scheme $S$, any element of $F(S)$ would then correspond to a morphism of schemes $S\rightarrow X$ (why?). Now, upto isomorphism there are only two conics in $\P_k^2$: the pair of lines $P$, and the smooth conic $Q$. This means that $X$ has only two $k$ points which we will also denote by $\{p,q\}$, respectively.
	
	Consider the family $Z$ over $\A_t^1$ described in Example \ref{family-nodal} (here the subscript $t$ is just to keep track of the variable used to define $\A^1$). This is a family of conics given by the equation $XY-tW^2$ in $\P_{\A_t^1}^2$. Its geometric fibres are given by,
	\begin{center}
		\begin{align*}
		Z_t = &\left\{
		\begin{aligned}
		& P,  && t=0\\
		& Q,  && t\neq 0
		\end{aligned}\right.
		\end{align*}
	\end{center}
	We have a map $f:\A_t^1\rightarrow X$ which pulls back the universal family\footnote{The \textit{universal family} is the element in $F(X)$ corresponding to the identity morphism $id: X\rightarrow X$.} to $Z$, i.e, the base change of the universal family to $\A_t^1$ is $Z$. Now, observe that for the closed point $q\in X$, $f^{-1}(q)$ is all non-zero $t$ in $\A_t^1$ (which is not a closed set!). Hence, such a map cannot be continuous. In particular, no such map exists. But clearly, $Z\in F(\A_t^1)$. Hence, $F$ is not representable.
\end{example}

\subsection{The Sheaf Condition.} Let $\mathcal{C}$ be a site, and $\{U_i \overset{f_i}{\to} U\}_{i\in I}$ be an element of $\text{Cov}(\mathcal{C})$. We have the fibre products for all $i,j$
\begin{center}
	\begin{tikzcd}
	U_i\times_U U_j\arrow[r,"\text{pr}_1"]\arrow[d,"\text{pr}_0"] & U_j\arrow[d,"f_j"]\\
	U_i\arrow[r,"f_i"] & U
	\end{tikzcd}
\end{center}

Let $\sF$ be a presheaf on $\mathcal{C}$. For the above covering we have a diagram,
\begin{equation}\label{equation-sheaf-condition}
\begin{tikzcd}
\sF(U)\arrow[r] &\prod_{i\in I} \sF(U_i)\arrow[r,shift left,"\text{pr}^*_0"]\arrow[r,shift right,"\text{pr}^*_1"'] & \prod_{i,j\in I}\sF(U_i\times_U U_j)
\end{tikzcd}
\end{equation}

\begin{definition}
	Let $\mathcal{C}$ be a site, and let $\mathcal{F}$ be a presheaf of sets
	on $\mathcal{C}$. We say $\mathcal{F}$ is a {\it sheaf} if
	for every covering $\{U_i \to U\}_{i \in I} \in \text{Cov}(\mathcal{C})$,
	the diagram (\ref{equation-sheaf-condition}) represents the first arrow as the equalizer of $\text{pr}_0^*$
	and $\text{pr}_1^*$.	
\end{definition}

Loosely speaking this means that given sections $s_i \in \mathcal{F}(U_i)$
such that
$$
s_i|_{U_i \times_U U_j} = s_j|_{U_i \times_U U_j}
$$
in $\mathcal{F}(U_i \times_U U_j)$ for all $i, j \in I$
then there exists a unique $s \in \mathcal{F}(U)$ such
that $s_i = s|_{U_i}$\footnote{It is instructive to write out all the details in diagram (\ref{equation-sheaf-condition}) when the covering family has $3$ elements.}. Note that the above definition implies that $\sF(\emptyset)=\{*\}$ is a singleton.

\begin{example}
	Any scheme $X$ is sheaf on the big Zariski site $(\mathit{Sch}/S)_{\textit{Zar}}$ (why?).
\end{example}


\begin{definition}
	\label{definition-category-sheaves-sets}
	The category {\it $\Sh(\mathcal{C})$}
	of sheaves of sets is the full subcategory of the category
	$\textit{PSh}(\mathcal{C})$ whose objects are the sheaves of sets.
\end{definition}


If the inclusion $\iota: \Sh(\mathcal{C})\hookrightarrow \textit{Psh}(\mathcal{C})$ admits a left adjoint, we call it the \textit{sheafification functor}. It may not exist even if fairy concrete situations. For example, it does not exist in the fpqc topology (see \cite[Theorem 5.5]{Waterhouse-fpqc-sheafification}). However, we have the following theorem due to Grothendieck.

\begin{frame}
	Why bother with the affine site?
\end{frame}


\begin{frame}{Separated Schemes}
	
\end{frame}

\begin{frame}{Schemes}
	
\end{frame}




We always want representable objects to be sheaves, as the sheaf property expresses the correct categorical notion of ``gluing" things. And representable functors being sheaves means that morphisms glue. Moreover, the statements in Definition \ref{locally-P-morphisms} can now be generalised to covering families of the various topologies on schemes.

\begin{definition}
	Let $\mathcal{C}$ be a category. We say that a topology $\tau$ on $\mathcal{C}$ is \textit{subcanonical} if every representable functor is a sheaf with respect to $\tau$. The \textit{canonical} topology is the finest on $\mathcal{C}$ such that every representable functor is a sheaf.
\end{definition}


\begin{example}[A non-subcanonical site]
	Consider the topology on $(\textit{Sch})$ with coverings given by $\{U_i \to U\}_{i \in I}$ which are jointly surjective families of flat morphisms. This ``wild" flat topology is not subcanonical.
	
	Take a smooth integral curve $U$ over an algebraically closed field $k$\footnote{For example, take $\P^1$.}. Let $K$ be its quotient field. For every closed point $p\in U(k)$, consider the spectrum of its local ring $V_p:=\Spec \sO_{U,p}$. Then $\{V_p \to U\}_{p \in U(k)}$ is a covering in this wild flat topology.
	
	Note that each of these local rings are discrete valuation rings with generic point $\Spec K$ and closed point $p$\footnote{Well, the maximal ideal in $\sO_{U,p}$ corresponding to $p$.}. Thus, $V_p\times_U V_q =V_p$ if $p=q$, and $\Spec K$ otherwise.
	
	We can now construct a scheme $X$ by gluing all the $V_p$'s along $\Spec K$\footnote{Note that this is just the construction in Example \ref{DVR-double-origin} iterated over all $p\in U(k)$}. The functor $\mathit{Hom}(-,X)$ is not a sheaf in the wild flat topology. In fact, we will show that it does not satisfy the sheaf condition for the covering $\{V_p \to U\}_{p \in U(k)}$. For this, consider the diagram
	\begin{center}
		\begin{tikzcd}
		\prod_{p\in U(k)} h_X(V_p)\arrow[r,shift left,"\text{pr}^*_0"]\arrow[r,shift right,"\text{pr}^*_1"'] & \prod_{p,q\in U(k)}h_X(V_p\times_U V_q)
		\end{tikzcd}
	\end{center}
	By construction of $X$, we have inclusions $i_p:V_p\hookrightarrow X$ for each $p\in U(k)$. This gives us an element $\{i_p\}_{p\in U(k)}$ of $\prod_{p\in U(k)} h_X(V_p)$ whose images in along $\text{pr}_0$ and $\text{pr}_1$ agree (why?). However, there is no morphism $U\rightarrow X$. This is because all subsets of $X$ formed by closed points are closed, but only finite set are closed in $U$.
	
\end{example}


To prove Theorem \ref{fpqc-representable}, we will use the following lemma.
\begin{frame}{Characterising fpqc sheaf property}
\begin{lemma}
	\label{lemma-sheaf-property-fpqc}
	Let $F:\text{Sch}\rightarrow\text{Sets}$ be a presheaf. Then $F$ satisfies the sheaf property for the fpqc topology if and only
	if it satisfies
	\begin{enumerate}
		\item the sheaf property for every Zariski covering, and
		\item the sheaf property for $\{V \to U\}$
		with $V$, $U$ affine and $V \to U$ faithfully flat.
	\end{enumerate}
\end{lemma}
\end{frame}

\begin{frame}{Characterising fpqc sheaf property}

\end{frame}

\begin{proof}
	Note that since every Zariki cover is an fpqc cover, the only if direction is obvious. So we only need to show the ($\Rightarrow$) direction.
	
	First observe that it is sufficient to prove this for fpqc covers $\{V\rightarrow U\}$ consisting of a single morphism (given an fpqc cover $\{U_i\rightarrow U\}_{i\in I}$, set $V:=\amalg_i U_i$). 
	
	The rest of the proof involves ``resolving" $F(U)$ in two different ways. First using the sheaf property for Zariski covers and second using the sheaf property the faithfully flat maps of affine. To do this, we will cover $U$ and $V$ by compatible affine opens.
	
	Take an fpqc cover $\{f: V\rightarrow U\}$. By definition, we have an open covering $V=\cup_i V_i$ such that each $V_i$ is quasi-compact and $f(V_i):=U_i$ is open and affine in $U$. Write $V_i$ as a union of finitely many open affines $V_{ia}$. Then we have the following diagram\footnote{or a ``resolution" of $F(U)$ in two different way.}
	\begin{center}
		\begin{tikzcd}
		F(U)\arrow[d]\arrow[r] &F(V)\arrow[d]\arrow[r,shift left]\arrow[r,shift right] & \prod_{i,j}F(V\times_U V)\arrow[d]\\
		\prod_{i} F(U_i)\arrow[d,shift left]\arrow[d,shift right]\arrow[r] &\prod_{i}\prod_a F(V_{ia})\arrow[d,shift left]\arrow[d,shift right]\arrow[r,shift left]\arrow[r,shift right] & \prod_i\prod_{a,b}F(V_{ia}\times_U V_{ib})\\
		\prod_{i,j I}F(U_i\cap U_j)\arrow[r] &\prod_i\prod_{a,b}F(V_{ia}\cap V_{ib})
		\end{tikzcd}
	\end{center}
	Since $F$ is a Zariski sheaf, the columns are equalizers. Further, note that since  $\{\amalg_a V_{ia}\rightarrow U_i\}$ is a faithfully flat ring map the diagram
	\begin{center}
		\begin{tikzcd}
		F(U_i)\arrow[r] &\prod_a F(V_{ia})\arrow[r,shift left]\arrow[r,shift right] & \prod_{a,b}F(V_{ia}\times_U V_{ib})
		\end{tikzcd}
	\end{center}
	is an equalizer. As equalizers commute with products, the second row is also an equalizer. Hemce, $F(U)\rightarrow F(V)$ is injective, and so the bottom row is also injective. Now, a diagram chase show that the top row is an equalizer.
\end{proof}

\begin{frame}
	\begin{theorem}[Grothendieck]\label{fpqc-representable}
		Every representable functor satisfies the sheaf property in the fpqc topology.
	\end{theorem}
\end{frame}


\begin{proof}[Proof of Theorem \ref{fpqc-representable}]
	We only have to check condition (2) of Lemma \ref{lemma-sheaf-property-fpqc}. Further, we can write $X=\cup_i X_i$ as a union of affines. So, we are reduced to the case of affine schemes. But, for affine this follows easily from Lemma \ref{Amitsur}.
\end{proof}


\begin{frame}{Amitsur's Lemma}
	Let $f: A\rightarrow B$ be a faithfully flat ring map. Then, the following sequence of $A$-modules is exact:
	\[0\rightarrow A\overset{f}{\rightarrow} B\overset{e_1 -e_2}{\rightarrow} B\otimes_A B\]
\end{frame}

\begin{frame}
	What happens at $B\otimes_A B$?
\end{frame}
\begin{frame}
	subcanonical sites and a non-canonical site
\end{frame}

\subsection{A Criteria for Representability.} Theorem \ref{fpqc-representable} implies that for a functor $F$ to be representable by a scheme, it must satisfy the sheaf property in the fpqc topology. Also, note that $(\mathit{Sch})$ is a full subcategory of $\mathit{Psh}(\mathit{Aff})$\footnote{This is straightforward to prove using the sheaf condition for the Zariski topology.}, a natural question is to identify when a functor $F$ lies in $\mathit{Sch}$. We will do this now.

Let $F,G$ be two functors on a category $\mathcal{C}$ with values in $\mathit{Sets}$. Note that by Lemma \ref{lemma-yoneda}, an element $\xi\in G(U)$ is the same as a morphism of functors $\xi: h_U\rightarrow G$.


\begin{definition}
	\label{definition-representable-map-presheaves}
	Let $\mathcal{C}$ be a category.
	Let $F, G : \mathcal{C}^{opp} \to \textit{Sets}$
	be functors. We say a morphism $a : F \to G$ is
	{\it representable}, or that {\it $F$ is relatively representable
		over $G$}, if for every $U \in \Ob(\mathcal{C})$
	and any $\xi \in G(U)$ the functor
	$h_U \times_G F$ is representable\footnote{Fibre products exist for functors to $\mathit{Sets}$ because fibre products of sets exists.}.
\end{definition}


\begin{remark}[Important reality check]
	Any morphism $h_U\rightarrow h_V$ between representable functors is representable.
\end{remark}
We have the following lemma about representable morphisms.

This is a straightforward exercise in definitions. The thing to keep in mind is the ``magic diagram"\footnote{as Ravi Vakil calls it.} (see \cite[Tag 0024]{stacks-project} for hints).

Representable morphisms of functors is a useful notion because any property of morphisms of schemes can be extended to representable morphisms.

\begin{definition}
	Let $\mathcal{P}$ be a property of morphisms of schemes that is stable under base change. Let $a: F\rightarrow G$ be a representable morphism of functors. We say that $a$ has the property $\mathcal{P}$, if for every scheme $U$ and every $\xi\in G(U)$ the morphism of schemes $a_U: h_U\times_G F\rightarrow h_U$ has $\mathcal{P}$. 
\end{definition}

Henceforth, we will confuse $U$ with its functor of points $h_U$.

\begin{definition}\label{separated-scheme}
	Let $F$ be a functor on the category of affine schemes $\mathit{Aff}$. We say that $F$ is a \textit{separated scheme} if the following conditions are satisfied:
	\begin{enumerate}
		\item $F$ is a sheaf on $(\mathit{Aff})_{\textit{Zar}}$.
		\item The diagonal $\Delta: F\rightarrow F\times F$ is a representable closed immersion.
		\item There exist affine schemes $\{U_i\}_{i\in I}$ and morphisms $\xi_i: U_i\rightarrow F$ which are representable open immersions, such that the map from the disjoint union $\amalg_i U_i\rightarrow F$ is an epimorphism of sheaves in the Zariski topology.
	\end{enumerate}
\end{definition}

We can tweak Definition \ref{definition-representable-map-presheaves} to the following: We will say $a: F\rightarrow G$ is \textit{representable by separated schemes} if the fibre product $h_U\times_G F$ is a separated scheme in the sense of Definition \ref{separated-scheme}.

\begin{definition}\label{scheme}
	Let $F$ be a functor on the category of affine schemes $\mathit{Aff}$. We say that $F$ is a \textit{scheme} if the following conditions are satisfied:
	\begin{enumerate}
		\item $F$ is a sheaf on $(\mathit{Aff})_{\textit{Zar}}$.
		\item The diagonal $\Delta: F\rightarrow F\times F$ is a representable by separated schemes.
		\item There exist affine schemes $\{U_i\}_{i\in I}$ and morphisms $\xi_i: U_i\rightarrow F$ which are representable open immersions, such that the map from the disjoint union $\amalg_i U_i\rightarrow F$ is an epimorphisms of sheaves in the Zariski topology.
	\end{enumerate}
\end{definition}

Be sure to check this gives this agrees with the usual description of a scheme.

\begin{remark}
	It is not possible to directly define the category of schemes without defining separated schemes first. The reason is that for a scheme the intersection of affine opens need not be an affine. This kind of functorial description was first thought of by Artin (and Grothendieck) to define quotients by \'{e}tale equivalence relations\footnote{For example, a $\Z/2$-action on a proper threefold. The details will show up in detail at some point in our seminar.} that do not exist as schemes. These are what are called \textit{algebraic spaces} (replace Zariski with \'{e}tale in Definition \ref{separated-scheme} above and you will have defined an (separated) algebraic space). You will find mentions of such a construction for schemes in any source discussing algebraic spaces. However, the first place where I found a complete construction as above was in Martin Olsson's book (see \cite{olsson16})\footnote{Definition \ref{scheme} is the same as the one given in \cite{olsson16}. But, one could also use schemes with affine diagonal instead of separated scheme for this construction.}.
\end{remark}




\bibliography{Mybib.bib}
\bibliographystyle{alpha}

\end{document}



