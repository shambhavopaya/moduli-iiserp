\documentclass[10pt]{amsart}
%\usepackage[margin=1.2in,marginparsep=0.1in,marginparwidth=1in]{geometry}
%\usepackage{amssymb,amsmath,amsthm,amstext,amscd,latexsym,graphics,graphicx,bbm,caption}
%\usepackage[usenames,dvipsnames,svgnames,table]{xcolor}
%\usepackage[plainpages=false,colorlinks=true, pagebackref]{hyperref}
\usepackage{enumitem}


\makeatletter
\def\@settitle{\begin{center}%
  \baselineskip14\p@\relax
  \bfseries
  \uppercasenonmath\@title
  \@title
  \ifx\@subtitle\@empty\else
     \\[1ex]\uppercasenonmath\@subtitle
     \footnotesize\mdseries\@subtitle
  \fi
  \end{center}%
}
\def\subtitle#1{\gdef\@subtitle{#1}}
\def\@subtitle{}
\makeatother


%% LaTeX Definitions
%\newcounter{countup}

\newcommand{\rup}[1]{\lceil{#1}\rceil}
\newcommand{\rdown}[1]{\lfloor{#1}\rfloor}
\newcommand{\ilim}{\mathop{\varprojlim}\limits} % inverse limit
\newcommand{\dlim}{\mathop{\varinjlim}\limits}  % direct limit
\newcommand{\surj}{\twoheadrightarrow}
\newcommand{\inj}{\hookrightarrow}
\newcommand{\tensor}{\otimes}
\newcommand{\ext}{\bigwedge}
\newcommand{\Intersection}{\bigcap}
\newcommand{\Union}{\bigcup}
\newcommand{\intersection}{\cap}
\newcommand{\union}{\cup}

%%%%%%%%%%%%%%%%%%%%%%%%%%%%% new new commands :) %%%%%%%%%%%%%%%%
\newcommand{\supp}{{\rm Supp}}
\newcommand{\Exceptional}{{\rm Ex}}
\newcommand{\del}{\partial}
\newcommand{\delbar}{\overline{\partial}}
\newcommand{\boldphi}{\mbox{\boldmath $\phi$}}

%%%%%%%%%%%%%%%%%%%%%%%%%%%%%%%%%%%%%%%%%%%%%%%%%%%%%%%%%%%%%%%%%%%%%%%%%%%%%%

\newcommand{\udiv}{\underline{\Div}}

%%%%%%%%%%%%%%%%%

\newcommand{\Proj}{{\P roj}}
\newcommand{\sEnd}{{\sE nd}}
\newcommand{\mc}{\mathcal}
\newcommand{\mb}{\mathbb}
\newcommand{\an}{{\rm an}} 
\newcommand{\red}{{\rm red}}
\newcommand{\codim}{{\rm codim}}
\newcommand{\Dim}{{\rm dim}}
\newcommand{\rank}{{\rm rank}}
\newcommand{\Ker}{{\rm Ker  }}
\newcommand{\Pic}{{\rm Pic}}
\newcommand{\per}{{\rm per}}
\newcommand{\ind}{{\rm ind}}
\newcommand{\Div}{{\rm Div}}
\newcommand{\Hom}{{\rm Hom}}
\newcommand{\Aut}{{\rm Aut}}
\newcommand{\im}{{\rm im}}
\newcommand{\Spec}{{\rm Spec \,}}
\newcommand{\Sing}{{\rm Sing}}
\newcommand{\sing}{{\rm sing}}
\newcommand{\reg}{{\rm reg}}
\newcommand{\Char}{{\rm char}}
\newcommand{\Tr}{{\rm Tr}}
\newcommand{\Gal}{{\rm Gal}}
\newcommand{\Min}{{\rm Min \ }}
\newcommand{\Max}{{\rm Max \ }}
\newcommand{\Alb}{{\rm Alb}\,}
\newcommand{\Mat}{{\rm Mat}}
%\newcommand{\GL}{{\rm GL}\,}        % For the general linear group
\newcommand{\GL}{{\G\L}}
\newcommand{\Ho}{{\rm Ho}}
\newcommand{\ie}{{\it i.e.\/},\ }
\renewcommand{\iff}{\mbox{ $\Longleftrightarrow$ }}
\renewcommand{\tilde}{\widetilde}
% Skriptbuchstaben
\newcommand{\sA}{{\mathcal A}}
\newcommand{\sB}{{\mathcal B}}
\newcommand{\sC}{{\mathcal C}}
\newcommand{\sD}{{\mathcal D}}
\newcommand{\sE}{{\mathcal E}}
\newcommand{\sF}{{\mathcal F}}
\newcommand{\sG}{{\mathcal G}}
\newcommand{\sH}{{\mathcal H}}
\newcommand{\sI}{{\mathcal I}}
\newcommand{\sJ}{{\mathcal J}}
\newcommand{\sK}{{\mathcal K}}
\newcommand{\sL}{{\mathcal L}}
\newcommand{\sM}{{\mathcal M}}
\newcommand{\sN}{{\mathcal N}}
\newcommand{\sO}{{\mathcal O}}
\newcommand{\sP}{{\mathcal P}}
\newcommand{\sQ}{{\mathcal Q}}
\newcommand{\sR}{{\mathcal R}}
\newcommand{\sS}{{\mathcal S}}
\newcommand{\sT}{{\mathcal T}}
\newcommand{\sU}{{\mathcal U}}
\newcommand{\sV}{{\mathcal V}}
\newcommand{\sW}{{\mathcal W}}
\newcommand{\sX}{{\mathcal X}}
\newcommand{\sY}{{\mathcal Y}}
\newcommand{\sZ}{{\mathcal Z}}
% Sonderbuchstaben mit Doppellinie
\newcommand{\A}{{\mathbb A}}
\newcommand{\B}{{\mathbb B}}
\newcommand{\C}{{\mathbb C}}
\newcommand{\D}{{\mathbb D}}
\newcommand{\E}{{\mathbb E}}
\newcommand{\F}{{\mathbb F}}
\newcommand{\G}{{\mathbb G}}
\newcommand{\HH}{{\mathbb H}}
\newcommand{\I}{{\mathbb I}}
\newcommand{\J}{{\mathbb J}}
\newcommand{\M}{{\mathbb M}}
\newcommand{\N}{{\mathbb N}}
\renewcommand{\O}{{\mathbb O}}
\renewcommand{\P}{{\mathbb P}}
\newcommand{\Q}{{\mathbb Q}}
\newcommand{\R}{{\mathbb R}}
\newcommand{\T}{{\mathbb T}}
\newcommand{\U}{{\mathbb U}}
\newcommand{\V}{{\mathbb V}}
\newcommand{\W}{{\mathbb W}}
\newcommand{\X}{{\mathbb X}}
\newcommand{\Y}{{\mathbb Y}}
\newcommand{\Z}{{\mathbb Z}}
\newcommand{\Sh}{\sS h}
\newcommand{\deltaop}{\Delta^{op}(\sS h(Sm/\mathbf{k}))}
\newcommand{\pdeltaop}{\Delta^{op}(P\sS h(Sm/\mathbf{k}))}
%\newcommand{\psh}{\pi_0^{\text{\tiny pre}}}
\newcommand{\psh}{\pi_0}
\renewcommand{\k}{\mathbf{k}}

\newcommand{\colim}{{\rm colim \,}}
\newcommand{\DM}[2]{\mathbf{DM}_{#2}^{\mathit{eff}}(#1)}


\theoremstyle{definition}
\newtheorem{theorem}{Theorem}[section]
\newtheorem{lemma}[theorem]{Lemma}
\newtheorem{question}[theorem]{Question}
\newtheorem{definition}[theorem]{Definition}
\newtheorem{remark}[theorem]{Remark}

\begin{document}
% Document information
\title{Gluing Stuff}

\author{Neeraj Deshmukh}
\date{\today}
%\address[IISERM]{Indian Institute of Science Education and Research, Mohali}

\maketitle
I'm going to outline some gluing problems here. Being able to do these will help you with ``descent" when we get to it.

\section{Fibre products in Topology}


\noindent\underline{Fact:} Let $U$ and $V$ be topological spaces. Assume that there exist open subsets $U'\subset U$ and $V'\subset V$ such that we have an isomorphism $\phi: U'\rightarrow V$. Then, we can construct the quotient $Z:= U\sqcup V/\sim$, where $u\in U$ is equivalent to $v\in V$ if $\phi(u)=v$ (in other words, we identify or ``glue" $U'$ and $V'$).

Now try the following exercise. The idea behind the exercise two-fold. First, to give you a taste of how ``abstract nonsense" arguments are made. Second, to show you how gluing techniques work. This is the essential technique used to construct all sorts of algebro-geometric objects. Ideally, you should be able to do this without explicitly using any ``points". Just basic facts about continuous maps and the universal properties of quotients and fibre products should suffice (or so I hope!).

\begin{enumerate}\addtocounter{enumi}{-1}
	\item Look-up the definition of a fibre product in a category (this is Tag 001U in the stacks project).
	\item What is the fibre product in the category of topological spaces? (tag 0020)
\end{enumerate}

Let $X, Y, Z$  be topological spaces and suppose we have maps $f:X\rightarrow Z$ and $g:Y\rightarrow Z$.
Assume that $X=X_1 \cup X_2$ and $W=X_1\cap X_2$.\\
This give us maps (by restriction) $f_1: X_1\rightarrow Z$ and $f_2:X_2\rightarrow Z$. Suppose that $X_i\times_{f_i,Z} Y$ exists for $i=1,2$.

\noindent\underline{Key assumption}: Further assume that for every open set $U\subset X_i$, the fibre product $U\times_{f_i,Z}Y$ exists and is an open set of $X_i\times_{f_i,Z}Y$.

We will construct the fibre product $X\times_Z Y$.


\begin{enumerate}[resume]
	\item Consider the compositions $W\hookrightarrow X_1\overset{f_1}{\rightarrow} Z$, and $W\hookrightarrow X_2\overset{f_2}{\rightarrow} Z$. By our assumptions, $W\times_{f_i,Z}Y$ exist for each $i$. Convince yourself, that these are isomorphic (and that the isomorphism is unique!).
	\item Let $\phi$ be the unique isomorphism. Using the ``Fact" above, construct a topological space $S=(X_1\times_{f_1,Z}Y\bigsqcup X_2\times_{f_2,Z}Y)/\sim$, where the identification `$\sim$' is along $\phi$. This is to say that we are ``gluing" $W\times_{f_1,Z}Y$ and $W\times_{f_2,Z}Y$ along $\phi$.
	\item By our assumptions, $W\times_{f_i,Z}Y$ is an open set of $X_i\times_{f_i,Z}Y$ for each $i$. Show that there exists a map $S\rightarrow Y$. Do this by checking that the projections $X_i\times_{f_i,Z}Y$ agree on $W\times_{f_i,Z}Y$ (upto composition by $\phi$). Now use, universal property of quotients of topological spaces to conclude.
	\item We have the compositions $X_i\times_{f_i,Z}Y\overset{pr}{\rightarrow} X_i\hookrightarrow X$, where the first map is the projection from the fibre product and the second one is inclusion. Check that they agree on $W\times_{f_i,Z}Y$ (upto composition by $\phi$). This gives us a map $S\rightarrow X$.
	\item Show that $S$ satisfies the universal property of the fibre product of $X$ and $Y$ over $Z$. (I have a feeling this make be a bit tricky. The idea is to decompose any map $h:T\rightarrow X$ in to parts $h:h^{-1}(X_i)\rightarrow X_i$, and then use properties of continuous maps to check that the maps on fibre products ``glue".)
\end{enumerate}

\section{Fibre Products of Schemes}

Essentially, the same kind of argument as above works when you try to construct fibre products for schemes. Because we know what it should be for affine schemes, we can glue using the universal property.

\begin{enumerate}\addtocounter{enumi}{-1}
	\item Let $A\rightarrow B$ and $A\rightarrow C$ be ring maps, show that $\Spec (B\otimes_A C)$ is the fibre product $\Spec B\times_{\Spec A} \Spec C$ (use universal property of the tensor product).
	\item  Let $X$ be a scheme. Let $X\rightarrow\Spec A$ and $\Spec B\rightarrow \Spec A$ be two morphisms of schemes. Show that the fibre product $X\times_{\Spec A}\Spec B$ exists (check that the key assumption holds and use the argument in section 1).
	\item Let $Y$ be a scheme with a morphism $Y\rightarrow \Spec A$. Show that $X\times_{\Spec A} Y$ exists.
	\item Let $Z$ be a scheme. Show that $X\times_Z Y$ exists. This might look tricky, but basically you have to cover $Z$ by affines, and glue (you also have to check to something similar to our key assumption that if $Z$ is an open subset of an affine scheme then $X\times_Z Y$ exists.)
	
\end{enumerate}

\underline{Note}: The way we have constructed the fibre product, it is clear that given affine cover $\lbrace X_i\rbrace$, $\lbrace Y_j\rbrace$ and $\lbrace Z_k\rbrace$ of $X$, $Y$ and $Z$, respectively,  $\bigcup_{i,j,k}X_i\times_{Z_k}Y_j$ is an affine open cover of $X\times_Z Y$.

\section{More Gluing}

\begin{enumerate}
	\item Normalisation.
	\item Closed Immersions.
	\item Affine morphisms (Relative Spec).
	\item Relative Proj
	\item sheaf of differentials (?)
\end{enumerate}


\end{document}



