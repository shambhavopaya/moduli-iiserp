\documentclass[ignorenonframetext,t]{beamer}
\usepackage{amscd}
\usepackage{verbatim}
\usepackage{tikz-cd}

\usetheme{Frankfurt}

\setbeamerfont{block body}{size=\small}

\setbeamercolor{mycolor}{fg=white,bg=black}
\defbeamertemplate*{footline}{shadow theme}{%
	\leavevmode%
	\hbox{\begin{beamercolorbox}[wd=.5\paperwidth,ht=2.5ex,dp=1.125ex,leftskip=.3cm plus1fil,rightskip=.3cm]{author in head/foot}%
			\usebeamerfont{author in head/foot}\hfill\insertshorttitle\, - \insertsubtitle
		\end{beamercolorbox}%
		\begin{beamercolorbox}[wd=.4\paperwidth,ht=2.5ex,dp=1.125ex,leftskip=.3cm,rightskip=.3cm plus1fil]{title in head/foot}%
			\usebeamerfont{title in head/foot}\insertshortauthor\hfill%
		\end{beamercolorbox}%
		\begin{beamercolorbox}[wd=.1\paperwidth,ht=2.5ex,dp=1.125ex,leftskip=.3cm,rightskip=.3cm plus1fil]{mycolor}%
			\hfill\insertframenumber\,/\,\inserttotalframenumber
	\end{beamercolorbox}}%
	\vskip0pt%
}


%% LaTeX Definitions
%\newcounter{countup}

\newcommand{\rup}[1]{\lceil{#1}\rceil}
\newcommand{\rdown}[1]{\lfloor{#1}\rfloor}
\newcommand{\ilim}{\mathop{\varprojlim}\limits} % inverse limit
\newcommand{\dlim}{\mathop{\varinjlim}\limits}  % direct limit
\newcommand{\surj}{\twoheadrightarrow}
\newcommand{\inj}{\hookrightarrow}
\newcommand{\tensor}{\otimes}
\newcommand{\ext}{\bigwedge}
\newcommand{\Intersection}{\bigcap}
\newcommand{\Union}{\bigcup}
\newcommand{\intersection}{\cap}
\newcommand{\union}{\cup}

%%%%%%%%%%%%%%%%%%%%%%%%%%%%% new new commands :) %%%%%%%%%%%%%%%%
\newcommand{\supp}{{\rm Supp}}
\newcommand{\Exceptional}{{\rm Ex}}
\newcommand{\del}{\partial}
\newcommand{\delbar}{\overline{\partial}}
\newcommand{\boldphi}{\mbox{\boldmath $\phi$}}

%%%%%%%%%%%%%%%%%%%%%%%%%%%%%%%%%%%%%%%%%%%%%%%%%%%%%%%%%%%%%%%%%%%%%%%%%%%%%%

\newcommand{\udiv}{\underline{\Div}}

%%%%%%%%%%%%%%%%%

\newcommand{\Proj}{{\P roj}}
\newcommand{\sEnd}{{\sE nd}}
\newcommand{\mc}{\mathcal}
\newcommand{\mb}{\mathbb}
\newcommand{\an}{{\rm an}} 
\newcommand{\red}{{\rm red}}
\newcommand{\codim}{{\rm codim}}
\newcommand{\Dim}{{\rm dim}}
\newcommand{\rank}{{\rm rank}}
\newcommand{\Ker}{{\rm Ker  }}
\newcommand{\Pic}{{\rm Pic}}
\newcommand{\per}{{\rm per}}
\newcommand{\ind}{{\rm ind}}
\newcommand{\Div}{{\rm Div}}
\newcommand{\Hom}{{\rm Hom}}
\newcommand{\Aut}{{\rm Aut}}
\newcommand{\im}{{\rm im}}
\newcommand{\Spec}{{\rm Spec \,}}
\newcommand{\Sing}{{\rm Sing}}
\newcommand{\sing}{{\rm sing}}
\newcommand{\reg}{{\rm reg}}
\newcommand{\Char}{{\rm char}}
\newcommand{\Tr}{{\rm Tr}}
\newcommand{\Gal}{{\rm Gal}}
\newcommand{\Min}{{\rm Min \ }}
\newcommand{\Max}{{\rm Max \ }}
\newcommand{\Alb}{{\rm Alb}\,}
\newcommand{\Mat}{{\rm Mat}}
%\newcommand{\GL}{{\rm GL}\,}        % For the general linear group
\newcommand{\GL}{{\G\L}}
\newcommand{\Ho}{{\rm Ho}}
\newcommand{\ie}{{\it i.e.\/},\ }
\renewcommand{\iff}{\mbox{ $\Longleftrightarrow$ }}
\renewcommand{\tilde}{\widetilde}
% Skriptbuchstaben
\newcommand{\sA}{{\mathcal A}}
\newcommand{\sB}{{\mathcal B}}
\newcommand{\sC}{{\mathcal C}}
\newcommand{\sD}{{\mathcal D}}
\newcommand{\sE}{{\mathcal E}}
\newcommand{\sF}{{\mathcal F}}
\newcommand{\sG}{{\mathcal G}}
\newcommand{\sH}{{\mathcal H}}
\newcommand{\sI}{{\mathcal I}}
\newcommand{\sJ}{{\mathcal J}}
\newcommand{\sK}{{\mathcal K}}
\newcommand{\sL}{{\mathcal L}}
\newcommand{\sM}{{\mathcal M}}
\newcommand{\sN}{{\mathcal N}}
\newcommand{\sO}{{\mathcal O}}
\newcommand{\sP}{{\mathcal P}}
\newcommand{\sQ}{{\mathcal Q}}
\newcommand{\sR}{{\mathcal R}}
\newcommand{\sS}{{\mathcal S}}
\newcommand{\sT}{{\mathcal T}}
\newcommand{\sU}{{\mathcal U}}
\newcommand{\sV}{{\mathcal V}}
\newcommand{\sW}{{\mathcal W}}
\newcommand{\sX}{{\mathcal X}}
\newcommand{\sY}{{\mathcal Y}}
\newcommand{\sZ}{{\mathcal Z}}
% Sonderbuchstaben mit Doppellinie
\newcommand{\A}{{\mathbb A}}
\newcommand{\B}{{\mathbb B}}
\newcommand{\C}{{\mathbb C}}
\newcommand{\D}{{\mathbb D}}
\newcommand{\E}{{\mathbb E}}
\newcommand{\F}{{\mathbb F}}
\newcommand{\G}{{\mathbb G}}
\newcommand{\HH}{{\mathbb H}}
\newcommand{\I}{{\mathbb I}}
\newcommand{\J}{{\mathbb J}}
\newcommand{\M}{{\mathbb M}}
\newcommand{\N}{{\mathbb N}}
\renewcommand{\O}{{\mathbb O}}
\renewcommand{\P}{{\mathbb P}}
\newcommand{\Q}{{\mathbb Q}}
\newcommand{\R}{{\mathbb R}}
\newcommand{\T}{{\mathbb T}}
\newcommand{\U}{{\mathbb U}}
\newcommand{\V}{{\mathbb V}}
\newcommand{\W}{{\mathbb W}}
\newcommand{\X}{{\mathbb X}}
\newcommand{\Y}{{\mathbb Y}}
\newcommand{\Z}{{\mathbb Z}}
\newcommand{\Sh}{\sS h}
\newcommand{\deltaop}{\Delta^{op}(\sS h(Sm/\mathbf{k}))}
\newcommand{\pdeltaop}{\Delta^{op}(P\sS h(Sm/\mathbf{k}))}
%\newcommand{\psh}{\pi_0^{\text{\tiny pre}}}
\newcommand{\psh}{\pi_0}
\renewcommand{\k}{\mathbf{k}}

\newcommand{\colim}{{\rm colim \,}}
\newcommand{\DM}[2]{\mathbf{DM}_{#2}^{\mathit{eff}}(#1)}

\theoremstyle{definition}
\newtheorem{question}[theorem]{Question}




% Document information
\title[Moduli@IISERP]{Seminar on Moduli Theory}
\subtitle{Lecture 17}
\author{Neeraj Deshmukh}
\date{January 8, 2021}
%\address[IISERP]{Indian Institute of Science Education and Research, Pune}

\begin{document}
	
	
\begin{frame}
\titlepage
\end{frame}

\begin{frame}{Last Time}
\begin{enumerate}
	\item Castelnuovo-Mumford regularity
\end{enumerate}
\end{frame}

\begin{frame}
\frametitle{Castelnuovo Mumford regularity}

	\begin{definition}
		Let $\sF$ be a coherent sheaf on $\P_k^n$ . Let $m$ be an integer. $\sF$ is said to be \textit{$m$-regular} if we have
		\[H^i(\P^n_k,\sF(m-i))=0 \;\;\text{for each}\;\; i\geq 1.\]
	\end{definition}
	
	\begin{theorem}[Mumford]
		Given any non-negative integers $p$ and $n$, there exists a polynomial $F_{p,n}$ in $n+1$-variables with the following property:\\
		If $\sF\subset \oplus^p\, \sO_{\P^n_k}$ is any coherent subsheaf  with Hilbert polynomial 
		\[\chi(\sF,r)=\overset{n}{\underset{i=0}{\sum}} a_i 
		\begin{pmatrix}
		r\\
		i	
		\end{pmatrix},\]
		then $\sF$ is $F_{p,n}(a_0,\ldots,a_n)$-regular.
	\end{theorem}
\end{frame}

\begin{frame}
	\begin{theorem}[Grothendieck]
		\label{theorem-quot-representable}
		Let $\pi: X\rightarrow S$ be a projective morphism with $S$ Noetherian. Then for any coherent sheaf $E$ on $X$ and any polynomial $\phi \in \Q[t]$, the functor $\mathfrak{Quot}^{\phi(t)}_{E/X/S}$ is representable by a projective $S$-scheme.
	\end{theorem}
	Three key ingredients:
	
	\quad
	
	(1)
	
	\quad
	
	(2)
	
	\quad
	
	(3)
\end{frame}



\begin{frame}
\frametitle{Flattening stratification}

\begin{theorem}
	Let $\sF$ be a coherent sheaf on $\P^n_S$ with $S$ a Noetherian scheme. Then the set $I$ of Hilbert polynomials of $\sF$ on the fibers of $\P^n_S\rightarrow S$ is a finite set. Moreover, for each $f\in I$, there exists a locally closed subscheme $S_f\subset S$ such that the following conditions hold:
	\begin{enumerate}
		\item $|S|=\coprod_f |S_f|$, set-theoretically;
		\item Fix an $f\in \Q[\lambda]$. For any morphism $\phi: T\rightarrow S$ the pullback $\phi^*\sF$ is flat on $\P^n_T$ with Hilbert polynomial $f$ if and only if $\phi:T\rightarrow S$ factors through $S_f$.
	\end{enumerate}
\end{theorem}

\end{frame}

\begin{frame}
	\textbf{Special case} when $n=0$
\end{frame}

\begin{frame}
	Behaviour of the fibres $\sF_s$ via generic flatness
\end{frame}

\begin{frame}
	Behaviour of the fibres $\sF_s$ via generic flatness
\end{frame}

\begin{frame}
	Applying special case to get stratification for pushforward sheaves 
\end{frame}

\begin{frame}
	Applying special case to get stratification for pushforward sheaves 
\end{frame}

\begin{frame}
	The correct subscheme structure on $W_{e_0,\ldots,e_n}$
\end{frame}


\bibliography{Mybib.bib}
\bibliographystyle{alpha}

\end{document}



