\documentclass[ignorenonframetext,t]{beamer}
\usepackage{amscd}
\usepackage{verbatim}

\usetheme{Frankfurt}

\setbeamerfont{block body}{size=\small}

\setbeamercolor{mycolor}{fg=white,bg=black}
\defbeamertemplate*{footline}{shadow theme}{%
	\leavevmode%
	\hbox{\begin{beamercolorbox}[wd=.5\paperwidth,ht=2.5ex,dp=1.125ex,leftskip=.3cm plus1fil,rightskip=.3cm]{author in head/foot}%
			\usebeamerfont{author in head/foot}\hfill\insertshorttitle\, - \insertsubtitle
		\end{beamercolorbox}%
		\begin{beamercolorbox}[wd=.4\paperwidth,ht=2.5ex,dp=1.125ex,leftskip=.3cm,rightskip=.3cm plus1fil]{title in head/foot}%
			\usebeamerfont{title in head/foot}\insertshortauthor\hfill%
		\end{beamercolorbox}%
		\begin{beamercolorbox}[wd=.1\paperwidth,ht=2.5ex,dp=1.125ex,leftskip=.3cm,rightskip=.3cm plus1fil]{mycolor}%
			\hfill\insertframenumber\,/\,\inserttotalframenumber
	\end{beamercolorbox}}%
	\vskip0pt%
}


%% LaTeX Definitions
%\newcounter{countup}

\newcommand{\rup}[1]{\lceil{#1}\rceil}
\newcommand{\rdown}[1]{\lfloor{#1}\rfloor}
\newcommand{\ilim}{\mathop{\varprojlim}\limits} % inverse limit
\newcommand{\dlim}{\mathop{\varinjlim}\limits}  % direct limit
\newcommand{\surj}{\twoheadrightarrow}
\newcommand{\inj}{\hookrightarrow}
\newcommand{\tensor}{\otimes}
\newcommand{\ext}{\bigwedge}
\newcommand{\Intersection}{\bigcap}
\newcommand{\Union}{\bigcup}
\newcommand{\intersection}{\cap}
\newcommand{\union}{\cup}

%%%%%%%%%%%%%%%%%%%%%%%%%%%%% new new commands :) %%%%%%%%%%%%%%%%
\newcommand{\supp}{{\rm Supp}}
\newcommand{\Exceptional}{{\rm Ex}}
\newcommand{\del}{\partial}
\newcommand{\delbar}{\overline{\partial}}
\newcommand{\boldphi}{\mbox{\boldmath $\phi$}}

%%%%%%%%%%%%%%%%%%%%%%%%%%%%%%%%%%%%%%%%%%%%%%%%%%%%%%%%%%%%%%%%%%%%%%%%%%%%%%

\newcommand{\udiv}{\underline{\Div}}

%%%%%%%%%%%%%%%%%

\newcommand{\Proj}{{\P roj}}
\newcommand{\sEnd}{{\sE nd}}
\newcommand{\mc}{\mathcal}
\newcommand{\mb}{\mathbb}
\newcommand{\an}{{\rm an}} 
\newcommand{\red}{{\rm red}}
\newcommand{\codim}{{\rm codim}}
\newcommand{\Dim}{{\rm dim}}
\newcommand{\rank}{{\rm rank}}
\newcommand{\Ker}{{\rm Ker  }}
\newcommand{\Pic}{{\rm Pic}}
\newcommand{\per}{{\rm per}}
\newcommand{\ind}{{\rm ind}}
\newcommand{\Div}{{\rm Div}}
\newcommand{\Hom}{{\rm Hom}}
\newcommand{\Aut}{{\rm Aut}}
\newcommand{\im}{{\rm im}}
\newcommand{\Spec}{{\rm Spec \,}}
\newcommand{\Sing}{{\rm Sing}}
\newcommand{\sing}{{\rm sing}}
\newcommand{\reg}{{\rm reg}}
\newcommand{\Char}{{\rm char}}
\newcommand{\Tr}{{\rm Tr}}
\newcommand{\Gal}{{\rm Gal}}
\newcommand{\Min}{{\rm Min \ }}
\newcommand{\Max}{{\rm Max \ }}
\newcommand{\Alb}{{\rm Alb}\,}
\newcommand{\Mat}{{\rm Mat}}
%\newcommand{\GL}{{\rm GL}\,}        % For the general linear group
\newcommand{\GL}{{\G\L}}
\newcommand{\Ho}{{\rm Ho}}
\newcommand{\ie}{{\it i.e.\/},\ }
\renewcommand{\iff}{\mbox{ $\Longleftrightarrow$ }}
\renewcommand{\tilde}{\widetilde}
% Skriptbuchstaben
\newcommand{\sA}{{\mathcal A}}
\newcommand{\sB}{{\mathcal B}}
\newcommand{\sC}{{\mathcal C}}
\newcommand{\sD}{{\mathcal D}}
\newcommand{\sE}{{\mathcal E}}
\newcommand{\sF}{{\mathcal F}}
\newcommand{\sG}{{\mathcal G}}
\newcommand{\sH}{{\mathcal H}}
\newcommand{\sI}{{\mathcal I}}
\newcommand{\sJ}{{\mathcal J}}
\newcommand{\sK}{{\mathcal K}}
\newcommand{\sL}{{\mathcal L}}
\newcommand{\sM}{{\mathcal M}}
\newcommand{\sN}{{\mathcal N}}
\newcommand{\sO}{{\mathcal O}}
\newcommand{\sP}{{\mathcal P}}
\newcommand{\sQ}{{\mathcal Q}}
\newcommand{\sR}{{\mathcal R}}
\newcommand{\sS}{{\mathcal S}}
\newcommand{\sT}{{\mathcal T}}
\newcommand{\sU}{{\mathcal U}}
\newcommand{\sV}{{\mathcal V}}
\newcommand{\sW}{{\mathcal W}}
\newcommand{\sX}{{\mathcal X}}
\newcommand{\sY}{{\mathcal Y}}
\newcommand{\sZ}{{\mathcal Z}}
% Sonderbuchstaben mit Doppellinie
\newcommand{\A}{{\mathbb A}}
\newcommand{\B}{{\mathbb B}}
\newcommand{\C}{{\mathbb C}}
\newcommand{\D}{{\mathbb D}}
\newcommand{\E}{{\mathbb E}}
\newcommand{\F}{{\mathbb F}}
\newcommand{\G}{{\mathbb G}}
\newcommand{\HH}{{\mathbb H}}
\newcommand{\I}{{\mathbb I}}
\newcommand{\J}{{\mathbb J}}
\newcommand{\M}{{\mathbb M}}
\newcommand{\N}{{\mathbb N}}
\renewcommand{\O}{{\mathbb O}}
\renewcommand{\P}{{\mathbb P}}
\newcommand{\Q}{{\mathbb Q}}
\newcommand{\R}{{\mathbb R}}
\newcommand{\T}{{\mathbb T}}
\newcommand{\U}{{\mathbb U}}
\newcommand{\V}{{\mathbb V}}
\newcommand{\W}{{\mathbb W}}
\newcommand{\X}{{\mathbb X}}
\newcommand{\Y}{{\mathbb Y}}
\newcommand{\Z}{{\mathbb Z}}
\newcommand{\Sh}{\sS h}
\newcommand{\deltaop}{\Delta^{op}(\sS h(Sm/\mathbf{k}))}
\newcommand{\pdeltaop}{\Delta^{op}(P\sS h(Sm/\mathbf{k}))}
%\newcommand{\psh}{\pi_0^{\text{\tiny pre}}}
\newcommand{\psh}{\pi_0}
\renewcommand{\k}{\mathbf{k}}

\newcommand{\colim}{{\rm colim \,}}
\newcommand{\DM}[2]{\mathbf{DM}_{#2}^{\mathit{eff}}(#1)}

\theoremstyle{definition}
\newtheorem{question}[theorem]{Question}




% Document information
\title[Moduli@IISERP]{Seminar on Moduli Theory}
\subtitle{Lecture 8}
\author{Neeraj Deshmukh}
\date{October 16, 2020}
%\address[IISERP]{Indian Institute of Science Education and Research, Pune}

\begin{document}
	
	
\begin{frame}
\titlepage
\end{frame}

\begin{frame}{Last Week}
\begin{enumerate}
	\item Characterisation of fpqc sheaf property.
	\item representable functors have fpqc sheaf property.
	\item Examples of representable morphisms.
\end{enumerate}
\end{frame}

\begin{frame}
	The Hilbert function: $\chi(M,n): n\mapsto  dim_k(M_n)$
\end{frame}

\begin{frame}
	A function $f:\Z_{\geq n_0}\rightarrow Z$ is said to be \textit{polynomial-like} if there exists a polynomial $P_f(X)$ such that $f(n)=P_f(n)$ for $n\gg0$.\\
	We will show that the Hilbert function is polynomial-like.
\end{frame}

\begin{frame}
	\begin{theorem}
		$\chi(M,n)$ is a polynomial-like function of $n$, of degree $\leq r$.
	\end{theorem}
\end{frame}

\begin{frame}
	Examples of Hilbert function
\end{frame}

\begin{frame}
	Examples of Hilbert function
\end{frame}

\begin{frame}
	Let $X\overset{i}{\hookrightarrow} Y\hookrightarrow \P_k^n$ be a sequence of closed embeddings.
\end{frame}

\begin{frame}
	Euler characteristic and Hilbert polynomial
	\begin{theorem}[Serre Vanishing]
		Let $\sF$ be a coherent sheaf on a projective $k$-scheme $X$, then for $m\gg 0$, $H^i(X,\sF(m))=0$ for all $i>0$.
	\end{theorem}
\end{frame}

In this section we will discuss theory of Hilbert polynomials of subschemes (and, more generally coherent sheaves) of $\P^n$. An important fact about Hilbert polynomials is that they are (locally) constant in flat families. This has some important consequences for the Hilbert scheme.

\subsection{Integer-valued polynomials and polynomial-like functions.}

This discussion has been taken from Serre's local algebra. For an integer $k$, we denote by $Q_k(X)$ the \textit{binomial polynomials}:
\[Q_k(X) = \begin{pmatrix}
X \\
k
\end{pmatrix} = \frac{X(X-1)\ldots(X-k+1)}{k!}\]
We set $Q_0(X)=1$. These form a basis of $\Q[X]$.

If $\Delta$ denotes the difference operator $\Delta f(n)=f(n+1)-f(n)$, observe that for $k> 0$:

\begin{align*}
\Delta Q_k(n) &= \frac{(n+1)n\ldots(n-k+2)}{k!} - \frac{n(n-1)\ldots(n-k+1)}{k!}\\
&= \frac{n(n-1)\ldots(n-k+2)}{k!}[(n+1)- (n-k+1)]\\
&= \frac{n(n-1)\ldots(n-k+2)}{(k-1)!}\\
&= Q_{k-1}(n)
\end{align*}

\begin{lemma}
	Let f be an element of $\Q[X]$. The following are equivalent:
	\begin{enumerate}
		\item $f$ is a $\Z$-linear combination of the binomial polynomials $Q_k$.
		\item $f(n)\in \Z$ for all $n\in\Z$.
		\item $f(n)\in \Z$ for all $n\in\Z$ large enough.
		\item $\Delta f$ has property (1), and $f(n)\in\Z$ for some integer $n$.
	\end{enumerate}
\end{lemma}
\begin{proof}
	The implication $(1)\Rightarrow (2)\Rightarrow (3)$ is obvious. For $(1)\Rightarrow (4)$, given an $f$ which satisfies $(1)$, we compute
	\begin{align*}
	\Delta f(n) &= f(n+1) - f(n)\\
	&= \sum_k e_k Q_k (n+1) - \sum_k e_k Q_k(n)\\
	\end{align*}
	Rearranging the terms, we see that
	\[\Delta f(n)= \sum_k e_k \Delta Q_k(n)=\sum_k e_k Q_{k-1}(n)\]
	Hence, $\Delta f$ satisfies property $(1)$.
	
	To prove $(4)\Rightarrow (1)$, write $\Delta f = \sum_k e_k Q_k$, with $e_k\in \Z$. Then, by the above computation, $f=\sum_k e_k Q_{k+1} + e_0$ with $e_0\in \Q$. But as $f$ take at least one integer value $e_0$ must be an integer. Hence, $(1)\Leftrightarrow (4)$.	
	To prove $(3)\Rightarrow (1)$, we use induction on the degree of $f$. If $f$ is a constant, then this is certainly true. If $f$ has degree $n$, then note that the degree of $\Delta f$ is strictly less than $n$. So, by the induction hypothesis, $\Delta f$ satisfies property $(1)$. Moreover, $f(n)\in \Z$ for large enough $n$. Thus, $(3)\Rightarrow (4)\Leftrightarrow (1)$.
\end{proof}

A polynomial $f$ satisfying the equivalent conditions of the above lemma is called an \textit{integer-valued polynomial}. Any such $f$ can be written as $\sum e_k Q_k$. We shall often write the coefficient of $Q_k$ as $e_k(f)$. A simple computation shows that $e_k(f)= e_{k-1}(\Delta f)$. If $deg(f)\leq k$, then $\Delta^k f$ is the constant polynomial $e_k(f)$.

We will now relate integer-valued polynomomials to certain special kind of integer valued functions. A function $f:\Z_{\geq n_0}\rightarrow Z$ is said to be \textit{polynomial-like} if there exists a polynomial $P_f(X)$ such that $f(n)=P_f(n)$ for $n\gg0$\footnote{This notation means for $n$ sufficiently large.}. $P_f$ is uniquely detemined by $f$. This is because if there exists another polynomial $Q_f$ such that $f(n)=Q_f(n)$ for $m\gg 0$, then $P_f-Q_f$ is zero for all $n$ large enough. But the degree of $P_f-Q_f$ is bounded. Hence, it is the zero polynomial. Note that this means $P_f$ is integer-valued.

\begin{lemma}
	\label{lemma-polynomial-like-function}
	The following are equivalent:
	\begin{enumerate}
		\item $f$ is polynomial-like;
		\item $\Delta f$ is polynomial-like;
		\item there exists $r\geq 0$ such that $\Delta^r f(n)=0$ for $n\gg 0$.
	\end{enumerate}
\end{lemma}
\begin{proof}
	$(1)\Rightarrow (2)\Rightarrow (3)$ are straight-forward.
	
	For $(2)\Rightarrow (1)$, let $P_{\Delta f}$ be the polynomial associated to $\Delta f$. Let $R$ be an integer valued polynomial with $\Delta R= P_{\Delta F}$. Then, the function $g: n\mapsto f(n)-R(n)$ satisfies $\Delta g(n)=0$, for $n\gg 0$. Hence, $g$ is constant for $n\gg 0$, so that $f(n)=R(n)+e_0$ for $n\gg 0$. This shows that $f$ is polynomial-like.
	
	Applying $(2)\Rightarrow (1)$ $r$-times gives $(3)\Rightarrow (1)$.
\end{proof}

\subsection{The Hilbert function.}
Let $k$ be a field and $\sF$ a coherent sheaf on $\P_k^n$. By similar arguments as in \ref{subsection-closed-in-proj}, we can show that $\sF$ corresponds to a graded module $M_{\bullet}$ over the graded ring $S:= k[X_0,X_1,\ldots, X_r]$. We will the hilbert function for $\sF$ in terms of the associated graded module.

Let $M_{\bullet}$ be a graded module over $S$. We define the \textit{Hilbert function} $\chi(M,n)$ of $M$ by the assignment $\chi(M,n): n\mapsto  dim_k(M_n)$\footnote{If $k$ was a Noetherian ring, then we can define the Hilbert function using length of $M_n$ over $k$}.

\begin{theorem}
	$\chi(M,n)$ is a polynomial-like function of $n$, of degree $\leq r$.
\end{theorem}
\begin{proof}
	We use induction on $r$. If $r=-1$\footnote{This is an oddity because of our choice of indexing.}, $M$ is a finitely generated module over $k$. Hence, $M_n=0$ for $n\gg 0$. Assume now that $r\geq 0$, and that the theorem has been proved for $r-1$. Let $\phi$ be the endomorphism of $M$ defined by $X_r$, i.e, $m\overset{\phi}{\mapsto} X_r m$. This is a graded endomosphism of degree $1$. We thus have a short exact sequence,
	\[0\rightarrow N_n\rightarrow M_n\rightarrow M_{n+1}\rightarrow R_{n+1}\rightarrow 0\]
	Hence:
	\[\Delta\chi(M,n)=\chi(M,n+1)-\chi(M,n)=\chi(R,n+1)-\chi(N,n).\]
	Since $X_r R=0=X_r N$, $R$ and $N$ can be viewed as graded modules over $k[X_0,X_1,\ldots,X_{r-1}]$. Therefore, by the induction hypothesis, $\chi(R,n)$ and $\chi(N,n)$ are polynomial-like functions of degree $\leq r-2$ and, so $\Delta\chi(M,n)$ has the same property. Thus, by Lemma \ref{lemma-polynomial-like-function}, $\chi(M,n)$ is polynomial-like of degree $\leq r-1$.
\end{proof}

We call the polynomial associated to $\chi(M, n)$ as the Hilbert polynomial of $M$. For a closed subscheme $X\subseteq \P_k^n$, the Hilbert polynomial $h_X$ of $X$ is defined to be the Hilbert polynomial of the graded module $\Gamma(X):=S/I$, where $I$ is the graded ideal of $X$.
We will now compute the Hilbert function in some concrete examples.

\begin{example}
	\label{example-hilbert-function}
	\begin{enumerate}
		\item ($d$ points) Let $V=(p_1,p_2,\ldots, p_d)$ be $d$ points in $\P_k^n$. For $m\geq d-1$, we can write a map
		\begin{align*}
		\phi: k[X_0,X_1,\ldots,X_n]_{m} &\rightarrow k^{d}\\
		F &\mapsto (F(p_1),F(p_2),\ldots,F(p_d))
		\end{align*}
		Note that this map is surjective. Moreover, $\phi(F)=0$ if and only if $F\in I(V)$. This tells us that the Hilbert polynomial of $d$ points is $h_V(m)=d$.
		\item ($d$-fold Veronese) For $\P_k^1$, the Veronese embedding is the map $\nu: \P_k^1\rightarrow \P_k^d$ given by $[X:Y]\mapsto [X^d:X^{d-1}Y:\ldots:XY^{d-1}:Y^d]$. The coordinate ring of the image can be identified with the subring $R:=\oplus_{m\geq 0} k[X_0,X_1]_{dm}\subset k[X_0,X_1]$. Then,
		\begin{align*}
		\chi(R,m)= \begin{pmatrix}
		dm+1\\
		1
		\end{pmatrix} = dm +1
		\end{align*}
		Similarly, for $\P_k^n$, the $d$-fold Veronese embedding $\nu: \P_k^n\rightarrow \P_k^N$ is given by the subring $R:=\oplus_{m\geq 0}k[X_0,X_1,\ldots,X_n]_{dm}\subset k[X_0,X_1,\ldots, X_n]$. And we see that the Hilbert function can be computed as,
		\begin{align*}
		\chi(R,m)= \begin{pmatrix}
		dm+n\\
		n
		\end{pmatrix}
		\end{align*}
		Thus, the Hilbert polynomial of $X:=\nu(\P_k^n)$ has the form \[h_X= \frac{d^n}{n!}m^n+\ldots.\]
		\item degree $d$ curve in $\P_k^2$. Let $X\subset \P_k^2$ be a curve given by the irreducible homogeneous polynomial $F$ of degree $d$. For $m\geq d$, the $m$-th graded piece $I(X)_m$ of the ideal of $X$ is given by elements of the form $F\cdot(\Gamma(X))_{m-d}$.
		Thus, we compute, for $m\geq d$,
		\begin{align*}
		\chi(\Gamma(X),m) & = \begin{pmatrix}
		m + 2 \\
		2
		\end{pmatrix}
		- \begin{pmatrix}
		m - d + 2 \\
		2
		\end{pmatrix}
		\end{align*}
		Thus, we see that for $m\geq d$ the hilbert polynomial $h_X$ is
		\[h_X(m) = d\cdot m - \frac{d(d-3)}{2}.\]
		One can extend this reasoning to any degree $d$ hypersurface in $\P_k^n$. Thus, for $m\geq d$, we have
		\begin{align*}
		\chi(\Gamma(X),m) & = \begin{pmatrix}
		m + n \\
		n
		\end{pmatrix}
		- \begin{pmatrix}
		m - d + n \\
		n
		\end{pmatrix}
		\end{align*}
		Expanding the terms tell us that for $m\geq d$, $h_X$ has the form,
		\[h_X = \frac{d}{(n-1)!}m^{n-1}+\ldots .\]
	\end{enumerate}
\end{example}

\begin{definition}
	Let $X\subset \P_k^n$ be a projective scheme fof dimension $r$, and $h_X$ be its Hilbert polynomial. We define the \textit{degree} of $X$ to be $r!$ times the leading coefficient of $h_X$. We define the \textit{arithmetic genus} as the constant term $h_X(0)$ of the Hilbert polynomial.
\end{definition}

From Example \ref{example-hilbert-function}(3), we can see that the degree of a hypersurface in $\P_k^n$ is precisely the degree of the irreducible homogeneous polynomial which defines it. Also, note that the degree is non-negative.

%remarks on degree and dimension, and arithmetic genus.

We want to prove the following theorem about Hilbert polynomials:

\begin{theorem}
	Let $X\rightarrow Y$ be a projective morphism to a locally Noetherian scheme $Y$. If $\sF$ is a coherent sheaf on $X$, then the Hilbert polynomial $\chi(\sF_{X_y},d)$ is locally constant for $y\in Y$.
\end{theorem}

The rest of the section will be dedicated to establishing this fact.

%\subsection{Flat base change.}


\bibliography{Mybib.bib}
\bibliographystyle{alpha}

\end{document}



