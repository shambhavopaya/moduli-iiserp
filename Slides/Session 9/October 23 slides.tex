\documentclass[ignorenonframetext,t]{beamer}
\usepackage{amscd}
\usepackage{verbatim}
\usepackage{tikz-cd}

\usetheme{Frankfurt}

\setbeamerfont{block body}{size=\small}

\setbeamercolor{mycolor}{fg=white,bg=black}
\defbeamertemplate*{footline}{shadow theme}{%
	\leavevmode%
	\hbox{\begin{beamercolorbox}[wd=.5\paperwidth,ht=2.5ex,dp=1.125ex,leftskip=.3cm plus1fil,rightskip=.3cm]{author in head/foot}%
			\usebeamerfont{author in head/foot}\hfill\insertshorttitle\, - \insertsubtitle
		\end{beamercolorbox}%
		\begin{beamercolorbox}[wd=.4\paperwidth,ht=2.5ex,dp=1.125ex,leftskip=.3cm,rightskip=.3cm plus1fil]{title in head/foot}%
			\usebeamerfont{title in head/foot}\insertshortauthor\hfill%
		\end{beamercolorbox}%
		\begin{beamercolorbox}[wd=.1\paperwidth,ht=2.5ex,dp=1.125ex,leftskip=.3cm,rightskip=.3cm plus1fil]{mycolor}%
			\hfill\insertframenumber\,/\,\inserttotalframenumber
	\end{beamercolorbox}}%
	\vskip0pt%
}


%% LaTeX Definitions
%\newcounter{countup}

\newcommand{\rup}[1]{\lceil{#1}\rceil}
\newcommand{\rdown}[1]{\lfloor{#1}\rfloor}
\newcommand{\ilim}{\mathop{\varprojlim}\limits} % inverse limit
\newcommand{\dlim}{\mathop{\varinjlim}\limits}  % direct limit
\newcommand{\surj}{\twoheadrightarrow}
\newcommand{\inj}{\hookrightarrow}
\newcommand{\tensor}{\otimes}
\newcommand{\ext}{\bigwedge}
\newcommand{\Intersection}{\bigcap}
\newcommand{\Union}{\bigcup}
\newcommand{\intersection}{\cap}
\newcommand{\union}{\cup}

%%%%%%%%%%%%%%%%%%%%%%%%%%%%% new new commands :) %%%%%%%%%%%%%%%%
\newcommand{\supp}{{\rm Supp}}
\newcommand{\Exceptional}{{\rm Ex}}
\newcommand{\del}{\partial}
\newcommand{\delbar}{\overline{\partial}}
\newcommand{\boldphi}{\mbox{\boldmath $\phi$}}

%%%%%%%%%%%%%%%%%%%%%%%%%%%%%%%%%%%%%%%%%%%%%%%%%%%%%%%%%%%%%%%%%%%%%%%%%%%%%%

\newcommand{\udiv}{\underline{\Div}}

%%%%%%%%%%%%%%%%%

\newcommand{\Proj}{{\P roj}}
\newcommand{\sEnd}{{\sE nd}}
\newcommand{\mc}{\mathcal}
\newcommand{\mb}{\mathbb}
\newcommand{\an}{{\rm an}} 
\newcommand{\red}{{\rm red}}
\newcommand{\codim}{{\rm codim}}
\newcommand{\Dim}{{\rm dim}}
\newcommand{\rank}{{\rm rank}}
\newcommand{\Ker}{{\rm Ker  }}
\newcommand{\Pic}{{\rm Pic}}
\newcommand{\per}{{\rm per}}
\newcommand{\ind}{{\rm ind}}
\newcommand{\Div}{{\rm Div}}
\newcommand{\Hom}{{\rm Hom}}
\newcommand{\Aut}{{\rm Aut}}
\newcommand{\im}{{\rm im}}
\newcommand{\Spec}{{\rm Spec \,}}
\newcommand{\Sing}{{\rm Sing}}
\newcommand{\sing}{{\rm sing}}
\newcommand{\reg}{{\rm reg}}
\newcommand{\Char}{{\rm char}}
\newcommand{\Tr}{{\rm Tr}}
\newcommand{\Gal}{{\rm Gal}}
\newcommand{\Min}{{\rm Min \ }}
\newcommand{\Max}{{\rm Max \ }}
\newcommand{\Alb}{{\rm Alb}\,}
\newcommand{\Mat}{{\rm Mat}}
%\newcommand{\GL}{{\rm GL}\,}        % For the general linear group
\newcommand{\GL}{{\G\L}}
\newcommand{\Ho}{{\rm Ho}}
\newcommand{\ie}{{\it i.e.\/},\ }
\renewcommand{\iff}{\mbox{ $\Longleftrightarrow$ }}
\renewcommand{\tilde}{\widetilde}
% Skriptbuchstaben
\newcommand{\sA}{{\mathcal A}}
\newcommand{\sB}{{\mathcal B}}
\newcommand{\sC}{{\mathcal C}}
\newcommand{\sD}{{\mathcal D}}
\newcommand{\sE}{{\mathcal E}}
\newcommand{\sF}{{\mathcal F}}
\newcommand{\sG}{{\mathcal G}}
\newcommand{\sH}{{\mathcal H}}
\newcommand{\sI}{{\mathcal I}}
\newcommand{\sJ}{{\mathcal J}}
\newcommand{\sK}{{\mathcal K}}
\newcommand{\sL}{{\mathcal L}}
\newcommand{\sM}{{\mathcal M}}
\newcommand{\sN}{{\mathcal N}}
\newcommand{\sO}{{\mathcal O}}
\newcommand{\sP}{{\mathcal P}}
\newcommand{\sQ}{{\mathcal Q}}
\newcommand{\sR}{{\mathcal R}}
\newcommand{\sS}{{\mathcal S}}
\newcommand{\sT}{{\mathcal T}}
\newcommand{\sU}{{\mathcal U}}
\newcommand{\sV}{{\mathcal V}}
\newcommand{\sW}{{\mathcal W}}
\newcommand{\sX}{{\mathcal X}}
\newcommand{\sY}{{\mathcal Y}}
\newcommand{\sZ}{{\mathcal Z}}
% Sonderbuchstaben mit Doppellinie
\newcommand{\A}{{\mathbb A}}
\newcommand{\B}{{\mathbb B}}
\newcommand{\C}{{\mathbb C}}
\newcommand{\D}{{\mathbb D}}
\newcommand{\E}{{\mathbb E}}
\newcommand{\F}{{\mathbb F}}
\newcommand{\G}{{\mathbb G}}
\newcommand{\HH}{{\mathbb H}}
\newcommand{\I}{{\mathbb I}}
\newcommand{\J}{{\mathbb J}}
\newcommand{\M}{{\mathbb M}}
\newcommand{\N}{{\mathbb N}}
\renewcommand{\O}{{\mathbb O}}
\renewcommand{\P}{{\mathbb P}}
\newcommand{\Q}{{\mathbb Q}}
\newcommand{\R}{{\mathbb R}}
\newcommand{\T}{{\mathbb T}}
\newcommand{\U}{{\mathbb U}}
\newcommand{\V}{{\mathbb V}}
\newcommand{\W}{{\mathbb W}}
\newcommand{\X}{{\mathbb X}}
\newcommand{\Y}{{\mathbb Y}}
\newcommand{\Z}{{\mathbb Z}}
\newcommand{\Sh}{\sS h}
\newcommand{\deltaop}{\Delta^{op}(\sS h(Sm/\mathbf{k}))}
\newcommand{\pdeltaop}{\Delta^{op}(P\sS h(Sm/\mathbf{k}))}
%\newcommand{\psh}{\pi_0^{\text{\tiny pre}}}
\newcommand{\psh}{\pi_0}
\renewcommand{\k}{\mathbf{k}}

\newcommand{\colim}{{\rm colim \,}}
\newcommand{\DM}[2]{\mathbf{DM}_{#2}^{\mathit{eff}}(#1)}

\theoremstyle{definition}
\newtheorem{question}[theorem]{Question}




% Document information
\title[Moduli@IISERP]{Seminar on Moduli Theory}
\subtitle{Lecture 9}
\author{Neeraj Deshmukh}
\date{October 23, 2020}
%\address[IISERP]{Indian Institute of Science Education and Research, Pune}

\begin{document}
	
	
\begin{frame}
\titlepage
\end{frame}

\begin{frame}{Last Week}
\begin{enumerate}
	\item Integer-valued polynomials and polynomial-like functions
	\item Hilbert function and examples
\end{enumerate}
\end{frame}


\begin{frame}
We want to prove the following theorem about Hilbert polynomials:

\begin{theorem}
	\label{theorem-hilbert-polynomial-in-flat-family}
	Let $X\rightarrow Y$ be a projective morphism with $Y$ locally Noetherian. If $\sF$ is a coherent sheaf on $X$ which is flat over $Y$, then the Hilbert polynomial $\chi(\sF_{X_y},d)$ is locally constant for $y\in Y$.
\end{theorem}

\end{frame}

The rest of the section is dedicated to establishing this fact.


%\subsection{Coherent sheaves on Projective space.}

\begin{frame}
\begin{theorem}[Serre Vanishing]
	\label{theorem-serre-vanishing}
	Let $A$ be a Noetherian ring. Let $\sF$ be a coherent sheaf on a projective $A$-scheme $X$. Then, for $m\gg 0$, $H^i(X,\sF(m))=0$ for all $i>0$.
\end{theorem}
\end{frame}

\begin{frame}
	Relative Serre Vanishing
\end{frame}
\begin{proof}
	Choose an embedding $i:X\hookrightarrow \P_A^n$. By taking the pushforward $i_*\sF$, we may assume that $X=\P_A^n$. In this situation, $\sF$ admits a surjection from a direct sum of line bundles $\sO(m)^{\oplus j}\twoheadrightarrow \sF$. Let $\sR$ denote the kernel. This gives us a short exact sequence of coherent sheaves. 
	\[0\rightarrow \sR \rightarrow\sO(m)^{\oplus j}\rightarrow \sF\rightarrow 0\]
	Note that the statement holds for line bundles $\sO_{\P_A^n}(m)$. Now, taking the associated long exact sequence in cohomology, and apply descending induction (see Vakil's notes or \cite[III, 5.2]{Hart1} for more details).
\end{proof}


\begin{remark}
	You can also define the Hilbert polynomial as the Euler characteristic. In fact, our notation is suggestive of this. For any coherent sheaf $\sF$, define
	\[h_{\sF}(m)= \sum_i (-1)^i dim_k H^i(X,\sF(m)).\]
	Checking that this is indeed a polynomial takes a bit of work. But note that this definition also works for finite type scheme with given a line bundle.\\
	Moreover, Theorem \ref{theorem-serre-vanishing} tell us that for $m\gg 0$, this agrees with our definition.
\end{remark}


%\subsection{Flat base change.}

In order to prove Theorem \ref{theorem-hilbert-polynomial-in-flat-family}, we need some result about cohomology and base change.

\begin{frame}
		
\begin{lemma}
	\label{lemma-base-change-map-flat-case}
	Suppose we have a commutative diagram of schemes,
	\begin{center}
		\begin{tikzcd}[ampersand replacement=\&, column sep=small, row sep=small]
		X' \arrow[r, "g'"] \arrow[d, "f'"] \&
		X \arrow[d, "f"] \\
		S' \arrow[r, "g"] \&
		S
		\end{tikzcd}
	\end{center}
	Let $\mathcal{F}$ be an	$\mathcal{O}_X$-modules.
	Assume both $g$ and $g'$ are flat.
	Then there exists a canonical base change map
	\[
	g^*R^if_*\mathcal{F}
	\longrightarrow
	R^if'_*(g')^*\mathcal{F}\]
\end{lemma}

\end{frame}

\begin{proof}
	Choose injective resolutions $\sF\rightarrow \sI^{\bullet}$, and $(g')^*\sF\rightarrow \sJ^{\bullet}$. Since $(g')^*$ is flat, it is also exact. This implies that the pushforward $(g')_*$ preserves injectives\footnote{Use adjunction formula with the fact that $(g')^*$ preserves injections.}. Thus, $(g')_*(g')^*\sF\rightarrow (g')_*\sJ^{\bullet}$ is an injective resolution of $(g')_*(g')^*\sF$. By adjunction we have a map, $\alpha: \sF \rightarrow (g')_*(g')^*\sF$. And since $(g')_*\sJ^{\bullet}$ and $\sI^{\bullet}$ are injective resolutions, we have a lift $\beta$ of $\alpha$,
	\begin{center}
		\begin{tikzcd}
		\sF \arrow[r]\arrow[d,"\alpha"] & \sI^{\bullet}\arrow[d, "\beta"]\\
		(g')_*(g')^*\sF\arrow[r]	& (g')_*\sJ^{\bullet}
		\end{tikzcd}
	\end{center}
	such that $\beta$ is unique upto homotopy. Pushing down to $S$, we get
	\[f_*\beta :
	f_*\mathcal{I}^\bullet
	\longrightarrow
	f_*(g')_*\mathcal{J}^\bullet
	=
	g_*(f')_*\mathcal{J}^\bullet
	\]
	By adjunction we get a map $g^*f_*\sI^\bullet \rightarrow (f')_*\sJ^\bullet$. Since $\sJ^\bullet$ computes the cohomology of $(g')^*\sF$, passing to cohomology gives us the required map.
\end{proof}

\begin{frame}
\begin{lemma}[Flat base change]
	\label{lemma-flat-base-change-cohomology}
	Consider a cartesian diagram of schemes
	\begin{center}
		\begin{tikzcd}[ampersand replacement=\&, column sep=small, row sep=small]
		X' \arrow[d,"f'"] \arrow[r, "g'"] \& X \arrow[d, "f"] \\
		S' \arrow[r,"g"] \& S
		\end{tikzcd}
	\end{center}
	Let $\mathcal{F}$ be a quasi-coherent $\mathcal{O}_X$-module
	with pullback $\mathcal{F}' = (g')^*\mathcal{F}$.
	Assume that $g$ is flat and that $f$ is quasi-compact and quasi-separated.
	For any $i \geq 0$
	\begin{enumerate}
		\item the base change map, $ g^*R^if_*\mathcal{F} \longrightarrow R^if'_*\mathcal{F}'$
		is an isomorphism
		\item if $S = \Spec(A)$ and $S' = \Spec(B)$, then
		$H^i(X, \mathcal{F}) \otimes_A B = H^i(X', \mathcal{F}')$.
	\end{enumerate}
\end{lemma}
\end{frame}

\begin{frame}
	\underline{Case 1:} $X$ is separated
\end{frame}


\begin{frame}
	\underline{Case 2:} $X$ is quasi-separated
\end{frame}
\begin{proof}
	Observe that being an isomorphism of sheaves is a local question, so (2) $\Rightarrow$ (1). Thus, it suffices to prove statement (2).
	
	Note that the higher direct images of $\sF$ along $f$ are quasi-coherent $\sO_S$-modules since  \[R^if_*\sF=\tilde{\Gamma (S,R^if_*\sF)}=\tilde{H^i(X,\sF)}.\]
	Similarly, $R^if'_*\sF'= \tilde{H^i(X',\sF')}$.
	
	Moreover, $g^*$ corresponds to $-\otimes_A B$. Let $A\rightarrow B$ be a flat ring homomorphism. Let $X$ be a qcqs scheme over $A$ and $\sF$ be a quasi-coherent $\sO_X$-module. Set $X_B:= X\times_A \Spec B$ and $\sF_B:= g'^*(\sF)$ be the pullback of $\sF$ to $X_B$.
	
	\noindent\underline{Case 1:} X is separated.\\
	Let $\sU:=\{U_i\}_i$ be an affine open cover of $X$. Then, we have
	\[ \check{H}^p(\sU,\sF)=H^p(X,\sF).\]
	By base changing the cover $\sU$ to $B$, we get
	\[\check{H}^p(\sU_B,\sF_B)=H^p(X_B,\sF),\]
	and that
	\[\check{C}^{\bullet}(\sU_B,\sF_B)=\check{C}^{\bullet}(\sU,\sF)\otimes_A B.\]
	As, $A\rightarrow B$ is flat, this is true after passing to cohomology as well.
	
	\noindent\underline{Case 2:} X is quasi-separated.\\
	Once again, take an affine open cover, $\sU:=\{U_i\}_i$. We can approximate this case from the case of $X$ separated using a spectral sequence argument. For this we use the \v{C}ech-to-cohomology spectral sequence. This spectral sequence has $E_2$-page,
	\[E^{p,q}_2=\check{H}^p(\sU,\underline{H}^q(\sF))\Rightarrow H^{p+q}(X,\sF)\]
	We have a similar spectral sequence for $X_B$,
	\[E^{p,q}_2=\check{H}^p(\sU_B,\underline{H}^q(\sF_B))\Rightarrow H^{p+q}(X_B,\sF_B).\]
	
	Since $U_{i_0,\ldots,i_p}$ are quasi-compact and separated, from the previous case, we have
	\[\check{H}^p(\sU_B,\underline{H}^q(\sF_B))=\check{H}^p(\sU,\underline{H}^q(\sF)) \otimes_A B.\]
	Since, $A\rightarrow B$ is flat, the right hand side is the \v{C}ech-to-cohomology spectral sequence tensored with $B$, and so, converges to $H^{p+q}(X,\sF)\otimes_A B$. Since these two spectral sequences agree on the $E_2$-page, they also agree in the limit. Thus for all $p,q$,
	\[H^{p+q}(X_B,\sF_B) \simeq H^{p+q}(X,\sF)\otimes_A B,\]
	as required.
\end{proof}

\begin{remark}
	If you wish to avoid using spectral sequences in Case 2, you can alternatively use Mayer-Vietoris sequence and argue as follows:
	Let $X=U_1\cup U_2 \cup \ldots \cup U_t$ be an affine cover of $X$. Since $X$ is quasi-compact, $t$ is finite. We will use induction minimal $t$ required to cover $X$. If $t=1$, then $X$ is affine, and we have the result by Case 1. \\
	If $t>1$, we may write $X=U\cup V$, where $U=U_1\cup U_2 \cup \ldots \cup U_{t-1}$, and $V$ is affine. Denote $a = f|_U : U \to S$, $b = f|_V : V \to S$, and $c = f|_{U \cap V} : U \cap V \to S$. Observe that $U$ can be covered by at least $t-1$ affines, and $V$, $U\cup V$ are separated.\\
	Then, (relative) Mayer-Vietoris \cite[Tag 01EC]{stacks-project} gives us a long exact sequence of cohomology sheaves,
	\[0 \rightarrow
	f_*\mathcal{F} \rightarrow
	a_*(\mathcal{F}|_U) \oplus b_*(\mathcal{F}|_V) \rightarrow
	c_*(\mathcal{F}|_{U \cap V}) \rightarrow
	R^1f_*\mathcal{F} \to \ldots
	\]
	From the base change map, we have a morphism of long exact sequences,
	\begin{center}
		\begin{tikzcd}
		0 \arrow[r]
		& g^*f_*\mathcal{F} \arrow[r]\arrow[d]
		& g^* a_*(\mathcal{F}|_U) \oplus g^*b_*(\mathcal{F}|_V) \arrow[r]\arrow[d, "\simeq"]
		& g^* c_*(\mathcal{F}|_{U \cap V}) \arrow[r]\arrow[d, "\simeq"]
		& g^* R^1f_*\mathcal{F} \arrow[r]\arrow[d] &\ldots \\
		0 \arrow[r]
		& g'_*f'^*\mathcal{F} \arrow[r]
		& g'_* a'^*(\mathcal{F}|_U) \oplus g'_*b'^*(\mathcal{F}|_V) \arrow[r]
		& g'_* c'^*(\mathcal{F}|_{U \cap V}) \arrow[r]
		& g_* R^1f'^*\mathcal{F} \arrow[r] &\ldots \\
		\end{tikzcd}
	\end{center}
	By induction, the base change map is an isomorphism for $U$, $V$ and $U\cup V$. A diagram chase now tells us that this is also true for $X$\footnote{The isomorphism of the left-most vertical arrow follows easily. You can use five lemma for the higher direct images.}.
\end{remark}


%\subsection{Hilbert polynomial in flat families.}

The following lemma is a special case of Theorem \ref{theorem-hilbert-polynomial-in-flat-family}. In fact, we will prove Theorem \ref{theorem-hilbert-polynomial-in-flat-family} by reducing to Lemma \ref{lemma-local-ring-hilbert-polynomial-in-flat-family}. The argument follows \cite[III, 9.8]{Hart1}.

\begin{frame}
	
	\begin{theorem}
		\label{theorem-hilbert-polynomial-in-flat-family}
		Let $X\rightarrow Y$ be a projective morphism with $Y$ locally Noetherian. If $\sF$ is a coherent sheaf on $X$ which is flat over $Y$, then the Hilbert polynomial $\chi(\sF_{X_y},d)$ is locally constant for $y\in Y$.
	\end{theorem}
	
\end{frame}

\begin{frame}

\begin{lemma}
	\label{lemma-local-ring-hilbert-polynomial-in-flat-family}
	Let $S=\Spec A$ be a Noetherian local ring. Let $\sF$ be a coherent sheaf on $X=\P_S^n$. Consider the following statements:
	\begin{enumerate}
		\item $\sF$ is flat over $S$;
		\item $H^0(X,\sF(m))$ is a free $A$-module of finite rank, for all $m\gg 0$;
		\item for any $t\in S$, the Hilbert polynomial $\chi(\sF_t,m)$ of $\sF_t$ on $X_t$ is independent of $t$.
	\end{enumerate}
	Then we have the implications, (1) $\Leftrightarrow$ (2) $\Rightarrow$ (3). Moreover, if $S$ is domain then they are all equivalent.
\end{lemma}

\end{frame}

\begin{proof}
	(1) $\Rightarrow$ (2). We can use \v{C}ech cohomology to compute the cohomology of $\sF(m)$. For the standard affine open cover $\sU$ of $X$, we have
	\[H^i(X,\sF(m))=h^i(C^{\bullet}(\sU,\sF(m))).\]
	Since $\sF$ is flat over $S$, each term $C^{i}(\sU,\sF(m))$ of the \v{C}ech complex is a flat $A$-module. On the other hand, by Theorem \ref{theorem-serre-vanishing}, for $m\gg 0$, $H^i(X,\sF(m))=0$ for all $i>0$. Thus, we have an exact sequence
	\[0\rightarrow H^0(X,\sF(m))\rightarrow C^{0}(\sU,\sF(m))\rightarrow C^{1}(\sU,\sF(m))\rightarrow \ldots \rightarrow C^{n}(\sU,\sF(m))\rightarrow 0,\]
	which gives a resolution of the $A$-module $H^0(X,\sF(m))$. We can now split this sequence into short exact sequences. Since, each $C^i$ is a flat $A$-module, we conclude that so is $H^0(X,\sF(m))$. But, since it also finitely generated, we see that it also has finite rank.\\
	(2) $\Rightarrow$ (1). Let $S=A[x_0,x_1,\ldots,x_n]$. Consider the graded $S$-module
	\[M=\underset{m\geq m_0}{\bigoplus}H^0(X,\sF(m))\]
	where $m_0\gg 0$ so that $H^0(X,\sF(m))$ is finite free for all $m\geq m_0$. Then, $\sF=\tilde{M}$. Since, $M$ is a free $A$-module, we see that $\sF$ is flat over $A$.\\
	(2) $\Rightarrow$ (3). Since $H^0(X,\sF(m))$ is a free module of finite rank for $m\gg 0$, it is enough to show that
	\[\chi(\sF_t,m)=\text{rank}_A\; H^0(X,\sF(m)),\]
	for $m\gg 0$.	For this, we will show that for any $t\in S$ and $m\gg 0$,
	\[H^0(X_t,\sF_t(m))=H^0(X,\sF(m)) \otimes_A k(t)\]
	Let $\mathfrak{p}$ be the prime ideal in $A$ corresponding to the point $t$, and let $S'=\Spec A_{\mathfrak{p}}$. Base change along the flat morphism $S'\rightarrow S$. Then, by Lemma \ref{lemma-flat-base-change-cohomology}, we can reduce to the situation when $t$ is a closed point of $S$. Denote the fibre over $t$ by $X_0$ and $\sF_0$, and let $k$ the residue field. Take a presentation of $k$ over $A$,
	\[A^q\rightarrow A\rightarrow k\rightarrow 0.\]
	Then we get an exact sequence of sheaves on $X$,
	\[\sF^q\rightarrow \sF \rightarrow \sF_0\rightarrow 0.\]
	Then for $m\gg 0$ we have an exact sequence,
	\[H^0(X,\sF(m)^q)\rightarrow H^0(X,\sF(m))\rightarrow H^0(X_0,\sF_0(m)) \rightarrow 0\]
	On the other hand, we can tensor the first sequence with $H^0(X,\sF(m))$. Comparing the two gives us an isomorphism,
	\[H^0(X_0,\sF_0(m))=H^0(X,\sF(m)) \otimes_A k(t)\]
	for all $m\gg 0$.
\end{proof}

\begin{frame}
	(1) $\Leftrightarrow$ (2)
\end{frame}

\begin{frame}
	(2) $\Rightarrow$ (3)
\end{frame}

We now finish the proof of Theorem \ref{theorem-hilbert-polynomial-in-flat-family}. In fact, the theorem follows effortlessly since all the heavy-lifting has already  been done.

\begin{proof}[Proof of Theorem \ref{theorem-hilbert-polynomial-in-flat-family}]
	Since $X\rightarrow Y$ is projective, choose an embedding $i:X\hookrightarrow \P_Y^n$. By taking the pushforward $i_*\sF$, we may assume that $X=\P_Y^n$. To check that Hilbert polynomial is locally constant, we can assume that $Y=\Spec \sO_{y,Y}$. Then we are reduced to the situation of Lemma \ref{lemma-local-ring-hilbert-polynomial-in-flat-family}.
\end{proof}



\bibliography{Mybib.bib}
\bibliographystyle{alpha}

\end{document}



