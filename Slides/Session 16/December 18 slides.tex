\documentclass[ignorenonframetext,t]{beamer}
\usepackage{amscd}
\usepackage{verbatim}
\usepackage{tikz-cd}

\usetheme{Frankfurt}

\setbeamerfont{block body}{size=\small}

\setbeamercolor{mycolor}{fg=white,bg=black}
\defbeamertemplate*{footline}{shadow theme}{%
	\leavevmode%
	\hbox{\begin{beamercolorbox}[wd=.5\paperwidth,ht=2.5ex,dp=1.125ex,leftskip=.3cm plus1fil,rightskip=.3cm]{author in head/foot}%
			\usebeamerfont{author in head/foot}\hfill\insertshorttitle\, - \insertsubtitle
		\end{beamercolorbox}%
		\begin{beamercolorbox}[wd=.4\paperwidth,ht=2.5ex,dp=1.125ex,leftskip=.3cm,rightskip=.3cm plus1fil]{title in head/foot}%
			\usebeamerfont{title in head/foot}\insertshortauthor\hfill%
		\end{beamercolorbox}%
		\begin{beamercolorbox}[wd=.1\paperwidth,ht=2.5ex,dp=1.125ex,leftskip=.3cm,rightskip=.3cm plus1fil]{mycolor}%
			\hfill\insertframenumber\,/\,\inserttotalframenumber
	\end{beamercolorbox}}%
	\vskip0pt%
}


%% LaTeX Definitions
%\newcounter{countup}

\newcommand{\rup}[1]{\lceil{#1}\rceil}
\newcommand{\rdown}[1]{\lfloor{#1}\rfloor}
\newcommand{\ilim}{\mathop{\varprojlim}\limits} % inverse limit
\newcommand{\dlim}{\mathop{\varinjlim}\limits}  % direct limit
\newcommand{\surj}{\twoheadrightarrow}
\newcommand{\inj}{\hookrightarrow}
\newcommand{\tensor}{\otimes}
\newcommand{\ext}{\bigwedge}
\newcommand{\Intersection}{\bigcap}
\newcommand{\Union}{\bigcup}
\newcommand{\intersection}{\cap}
\newcommand{\union}{\cup}

%%%%%%%%%%%%%%%%%%%%%%%%%%%%% new new commands :) %%%%%%%%%%%%%%%%
\newcommand{\supp}{{\rm Supp}}
\newcommand{\Exceptional}{{\rm Ex}}
\newcommand{\del}{\partial}
\newcommand{\delbar}{\overline{\partial}}
\newcommand{\boldphi}{\mbox{\boldmath $\phi$}}

%%%%%%%%%%%%%%%%%%%%%%%%%%%%%%%%%%%%%%%%%%%%%%%%%%%%%%%%%%%%%%%%%%%%%%%%%%%%%%

\newcommand{\udiv}{\underline{\Div}}

%%%%%%%%%%%%%%%%%

\newcommand{\Proj}{{\P roj}}
\newcommand{\sEnd}{{\sE nd}}
\newcommand{\mc}{\mathcal}
\newcommand{\mb}{\mathbb}
\newcommand{\an}{{\rm an}} 
\newcommand{\red}{{\rm red}}
\newcommand{\codim}{{\rm codim}}
\newcommand{\Dim}{{\rm dim}}
\newcommand{\rank}{{\rm rank}}
\newcommand{\Ker}{{\rm Ker  }}
\newcommand{\Pic}{{\rm Pic}}
\newcommand{\per}{{\rm per}}
\newcommand{\ind}{{\rm ind}}
\newcommand{\Div}{{\rm Div}}
\newcommand{\Hom}{{\rm Hom}}
\newcommand{\Aut}{{\rm Aut}}
\newcommand{\im}{{\rm im}}
\newcommand{\Spec}{{\rm Spec \,}}
\newcommand{\Sing}{{\rm Sing}}
\newcommand{\sing}{{\rm sing}}
\newcommand{\reg}{{\rm reg}}
\newcommand{\Char}{{\rm char}}
\newcommand{\Tr}{{\rm Tr}}
\newcommand{\Gal}{{\rm Gal}}
\newcommand{\Min}{{\rm Min \ }}
\newcommand{\Max}{{\rm Max \ }}
\newcommand{\Alb}{{\rm Alb}\,}
\newcommand{\Mat}{{\rm Mat}}
%\newcommand{\GL}{{\rm GL}\,}        % For the general linear group
\newcommand{\GL}{{\G\L}}
\newcommand{\Ho}{{\rm Ho}}
\newcommand{\ie}{{\it i.e.\/},\ }
\renewcommand{\iff}{\mbox{ $\Longleftrightarrow$ }}
\renewcommand{\tilde}{\widetilde}
% Skriptbuchstaben
\newcommand{\sA}{{\mathcal A}}
\newcommand{\sB}{{\mathcal B}}
\newcommand{\sC}{{\mathcal C}}
\newcommand{\sD}{{\mathcal D}}
\newcommand{\sE}{{\mathcal E}}
\newcommand{\sF}{{\mathcal F}}
\newcommand{\sG}{{\mathcal G}}
\newcommand{\sH}{{\mathcal H}}
\newcommand{\sI}{{\mathcal I}}
\newcommand{\sJ}{{\mathcal J}}
\newcommand{\sK}{{\mathcal K}}
\newcommand{\sL}{{\mathcal L}}
\newcommand{\sM}{{\mathcal M}}
\newcommand{\sN}{{\mathcal N}}
\newcommand{\sO}{{\mathcal O}}
\newcommand{\sP}{{\mathcal P}}
\newcommand{\sQ}{{\mathcal Q}}
\newcommand{\sR}{{\mathcal R}}
\newcommand{\sS}{{\mathcal S}}
\newcommand{\sT}{{\mathcal T}}
\newcommand{\sU}{{\mathcal U}}
\newcommand{\sV}{{\mathcal V}}
\newcommand{\sW}{{\mathcal W}}
\newcommand{\sX}{{\mathcal X}}
\newcommand{\sY}{{\mathcal Y}}
\newcommand{\sZ}{{\mathcal Z}}
% Sonderbuchstaben mit Doppellinie
\newcommand{\A}{{\mathbb A}}
\newcommand{\B}{{\mathbb B}}
\newcommand{\C}{{\mathbb C}}
\newcommand{\D}{{\mathbb D}}
\newcommand{\E}{{\mathbb E}}
\newcommand{\F}{{\mathbb F}}
\newcommand{\G}{{\mathbb G}}
\newcommand{\HH}{{\mathbb H}}
\newcommand{\I}{{\mathbb I}}
\newcommand{\J}{{\mathbb J}}
\newcommand{\M}{{\mathbb M}}
\newcommand{\N}{{\mathbb N}}
\renewcommand{\O}{{\mathbb O}}
\renewcommand{\P}{{\mathbb P}}
\newcommand{\Q}{{\mathbb Q}}
\newcommand{\R}{{\mathbb R}}
\newcommand{\T}{{\mathbb T}}
\newcommand{\U}{{\mathbb U}}
\newcommand{\V}{{\mathbb V}}
\newcommand{\W}{{\mathbb W}}
\newcommand{\X}{{\mathbb X}}
\newcommand{\Y}{{\mathbb Y}}
\newcommand{\Z}{{\mathbb Z}}
\newcommand{\Sh}{\sS h}
\newcommand{\deltaop}{\Delta^{op}(\sS h(Sm/\mathbf{k}))}
\newcommand{\pdeltaop}{\Delta^{op}(P\sS h(Sm/\mathbf{k}))}
%\newcommand{\psh}{\pi_0^{\text{\tiny pre}}}
\newcommand{\psh}{\pi_0}
\renewcommand{\k}{\mathbf{k}}

\newcommand{\colim}{{\rm colim \,}}
\newcommand{\DM}[2]{\mathbf{DM}_{#2}^{\mathit{eff}}(#1)}

\theoremstyle{definition}
\newtheorem{question}[theorem]{Question}




% Document information
\title[Moduli@IISERP]{Seminar on Moduli Theory}
\subtitle{Lecture 16}
\author{Neeraj Deshmukh}
\date{December 18, 2020}
%\address[IISERP]{Indian Institute of Science Education and Research, Pune}

\begin{document}
	
	
\begin{frame}
\titlepage
\end{frame}

\begin{frame}{Last Week}
\begin{enumerate}
	\item Hironaka's examples
	\item Hilbert scheme of a proper scheme need not be representable
\end{enumerate}
\end{frame}

\begin{frame}

\end{frame}

\begin{frame}
	Castelnuovo-Mumford regularity
\begin{definition}
	Let $\sF$ be a coherent sheaf on $\P_k^n$ . Let $m$ be an integer. $\sF$ is said to be \textit{$m$-regular} if we have
	\[H^i(\P^n_k,\sF(m-i))=0 \;\;\text{for each}\;\; i\geq 1.\]
\end{definition}

\end{frame}


As Nitsure points out in \cite{FGAExplained}, this definition is very strange-looking. However, it is quite useful for making inductive arguments with respect to suitable hyperplanes.

\begin{frame}
\begin{lemma}[Castelnuovo]
	Let $\sF$ be a $m$-regular on $\P^n_k$. Then the following statements hold:
	\begin{enumerate}
		\item The canonical map $H^0(\P^n_k,\sO(1)) \otimes H^0(\P^n_k,\sF(r)) \rightarrow H^0(\P^n_k,\sF(r+1))$ is surjective whenever $r\geq m$.
		\item $H^i(\P^n_k,\sF(r))=0$ whenever $i\geq 1$ and $r\geq m-i$. That is, if $\sF$ is $m$-regular then it also $m'$-regular for all $m'\geq m$.
		\item The sheaf $\sF(r)$ is generated by global sections, and all its higher cohomologies vanish, whenever $r\geq m$.
	\end{enumerate}
\end{lemma}
\end{frame}

\begin{frame}
	(2) $H^i(\P^n_k,\sF(r))=0$ whenever $i\geq 1$ and $r\geq m-i$. That is, if $\sF$ is $m$-regular then it also $m'$-regular for all $m'\geq m$.
\end{frame}

\begin{frame}
	(1) The canonical map $H^0(\P^n_k,\sO(1)) \otimes H^0(\P^n_k,\sF(r)) \rightarrow H^0(\P^n_k,\sF(r+1))$ is surjective whenever $r\geq m$.
\end{frame}

\begin{frame}
	\item The sheaf $\sF(r)$ is generated by global sections, and all its higher cohomologies vanish, whenever $r\geq m$.
\end{frame}


\begin{proof}
	Since cohomology behaves well with respect to field extension, we can assume that $k$ is infinite. We will now proceed by induction $n$.
	
	When $n=0$, all the statements are clearly true. So assume $n\geq 1$.
	
	Since $k$ is infinite, we can find a hyperplane $H\subset \P^n_k$ which does not contain any of the finitely many associated points of $\sF$, so that $\sF_H$ is again $m$-regular. By induction hypothesis, the assertions of the lemma hold for $\sF_H$.
	
	\noindent\underline{Proof of (2):} As $\sF$ is $m$-regular, $H^i(\sF(m-i))=0$. For $r> m-i$, we use induction on $r$. Consider the exact sequence
	\[H^i(\sF(r-1))\rightarrow H^i(\sF(r))\rightarrow H^i(\sF_H(r)).\]
	By induction for $r-1$, the first term is zero. Meanwhile, the lemma holds for $\sF_H$, so the last term is zero. This completes the proof of (2).
	
	\noindent\underline{Proof of (1):} We have a commutative diargram,
	\begin{center}
		\begin{tikzcd}
		& H^0(\sF(r))\otimes H^0(\sO(1))\arrow[r,"\sigma"]\arrow[d,"\mu"] & H^0(\sF_H(r))\otimes H^0(\sO_H(1))\arrow[d, "\tau"]\\
		H^0(\sF(r))\arrow[r,"\alpha"] & H^0(\sF(r+1))\arrow[r,"\nu_{r+1}"] & H^0(\sF_H(r+1))
		\end{tikzcd}
	\end{center}
	We need to show that $\mu$ is surjective.
	
	By $m$-regularity of $\sF$ and statement (2), $H^1(\sF(r-1))=0$ for $r\geq m$. Thus, the restriction map $\nu_r: H^0(\sF(r))\rightarrow H^0(\sF_H(r))$ is surjective. Also, the restriction map $\rho: H^0(\sO(1))\rightarrow H^0(\sO_H(1))$ is surjective. This shows that their tensor product $\sigma$ is surjective.
	
	The map $\tau$ is surjective, by induction.
	
	By commutativity of the above diagram, we have shown that $\nu_{r+1}\circ\mu$ is surjective. This means that $H^0(\sF(r+1))=im(\mu)+ker(\nu_{r+1})$. As the bottom row is exact, $H^0(\sF(r+1))=im(\mu)+ im(\alpha)$. But since $\alpha$ factors through $\mu$ (since it is defined using the exact sequence associated to $H$), $H^0(\sF(r+1))=im(\mu)$. This shows (1).
	
	\noindent\underline{Proof of (3):} Note that by repeatedly applying statement (1), we have surjections
	\[H^0(\sF(r))\otimes H^0(\sO(p))\twoheadrightarrow H^0(\sF(r+p))\]
	for $r\geq m$ and $p\geq 0$. 
	
	For $p\gg 0$, $\sF(r+p)$ is generated by global sections. It follows that $\sF(r)$ is generated by global sections for $r\geq m$. By statement (2), when $r\geq m$, $H^i(\sF(r))=0$ for all $i\geq 1$. This proves (3).
\end{proof}


The following technical result how $m$-regularity is used in the proof of Theorem \ref{theorem-quot-representable}.

\begin{frame}
\begin{theorem}[Mumford]
	Given any non-negative integers $p$ and $n$, there exists a polynomial $F_{p,n}$ in $n+1$-variables with the following property:\\
	If $\sF\subset \oplus^p\, \sO_{\P^n_k}$ is any coherent subsheaf  with Hilbert polynomial 
	\[\chi(\sF,r)=\overset{n}{\underset{i=0}{\sum}} a_i 
	\begin{pmatrix}
	r\\
	i	
	\end{pmatrix},\]
	then $\sF$ is $F_{p,n}(a_0,\ldots,a_n)$-regular.
\end{theorem}
\end{frame}

\begin{frame}
	Idea of Proof
\end{frame}

\begin{proof}(See \cite[Theorem 5.3]{FGAExplained}).
	Since cohomology behaves well with respect to extension of fields, we can assume that $k$ is infinite.
	
	The proof is by induction on $n$. When $n=0$, we can take $F_{p,0}$ to be any polynomial (what are coherent sheaves on $\P^0$?).
	
	Let $n\geq 1$. Since $k$ is infinite, we can find a hyperplane $H\subset \P^n_k$ which does not contain any of the finitely many associated points of $\oplus^p\,\sO_{\P^n_k}/\sF$. Thus, tensoring with $\sO_H$ gives us an exact sequence,
	\[0\rightarrow \sF_H \rightarrow \oplus^p\,\sO_H \rightarrow \oplus^p\,\sO_H/\sF \rightarrow 0.\]
	This shows that $\sF_H$ is isomorphic to a subsheaf of $\oplus^p\,\sO_{\P^{n-1}_k}/\sF$.
	
	Further, we have an exact sequence,
	\[0 \rightarrow \sF(-1)\rightarrow \sF \rightarrow \sF_H \rightarrow 0,\]
	which implies that $\chi(\sF_H(r))=\chi(\sF(r))-\chi(\sF(r-1))$. Expanding this out we see that
	\begin{align*}
	\chi(\sF_H(r))
	&=\overset{n}{\underset{i=0}{\sum}} a_i 
	\begin{psmallmatrix}
	r\\
	i	
	\end{psmallmatrix} -
	\overset{n}{\underset{i=0}{\sum}} a_i 
	\begin{psmallmatrix}
	r-1\\
	i	
	\end{psmallmatrix}\\
	&=\overset{n}{\underset{i=0}{\sum}} a_i 
	\begin{psmallmatrix}
	r-1\\
	i-1	
	\end{psmallmatrix}\\
	&=\overset{n-1}{\underset{j=0}{\sum}} b_j 
	\begin{psmallmatrix}
	r\\
	j	
	\end{psmallmatrix}
	\end{align*}
	where the coefficients $b_j$'s have the expressions $b_j=g_j(a_0,\ldots,a_n)$ where each $g_j$ is polynomial with integer coefficients (independent of $k$ and $\sF$).
	
	By induction hypothesis, there exists a polynomial $F_{p,n-1}$ such that $\sF$ is $m_0$-regular for $m_0=F_{p,n-1}(b_0,\ldots,b_{n-1})$. Substituting the expression of $b_j$'s in terms of the $a_i$'s, we get that $m_0=G(a_0,\ldots,a_n)$, where $G$ is a polynomial with integer coefficients inndependent of $k$ and $\sF$. Now, for any $m\geq m_0-1$, we have long exact sequence,
	\[0\rightarrow H^0(\sF(m-1))\rightarrow H^0(\sF(m))\rightarrow H^0(\sF_H(m))\rightarrow H^1(\sF(m-1))\rightarrow H^1(\sF(m)) \rightarrow 0 \ldots\]
	For $i\geq 2$, $m_0$-regularity of $\sF_H$, gives us isomorphisms $H^i(\sF(m-1))\simeq H^i(\sF(m))$. Morover, we have surjections $H^1(\sF(m-1))\rightarrow H^1(\sF(m))$ showing that $h^1(\sF(m))$ is a monotonically decreasing function in $m$ for $m\geq m_0 -2$.
	
	In fact for $m\geq m_0$, the function $h^1(\sF(m))$ is strictly decreasing meaning that 
	\[H^1(\sF(m))=0 \;\;\text{for}\;\; m\geq m_0+h^1(\sF(m)).\]
	To see this, note that $h^1(\sF(m-1))\geq h^1(\sF(m))$ for $m\geq m_0$. As $\sF$ is $m$-regular, the restriction map $\nu_j: H^0(\sF(j))\rightarrow H^0(\sF_H(j))$ is surjective for all $j\geq m$, so that $h^1(\sF(j-1))=h^1(\sF(j))$ for all $j\geq m$. As $h^1(\sF(j))=0$ for $j\gg 0$, $H^1(\sF(m))$ eventually becomes zero for $m\geq m_0$.
	
	Now, as $\sF\subset \oplus^p \sO_{\P^n}$, we must have $h^0(\sF(r))\leq p h^0(\sO_{\P^n}(r))= p 
	\begin{psmallmatrix}
	n+r\\
	n
	\end{psmallmatrix}
	$. Since $h^i(\sF(m))=0$ for $i\geq 2$ and $m\geq m_0 -2$, we now get,
	
	\begin{align*}
	h^1(\sF(m_0)) & = h^0(\sF(m_0)) - \chi(\sF(r))\\
	&\leq p
	\begin{pmatrix}
	n+ m_0\\
	n
	\end{pmatrix} - \sum_{i=0}^{n}a_i 
	\begin{pmatrix}
	m_0\\
	i
	\end{pmatrix}\\
	&= P(a_0,\ldots,a_n)
	\end{align*}
	where $P(a_0,\ldots,a_n)$ is a polynomial expression in $a_0,\ldots,a_n$, obtained by substituting $m_0=G(a_0,\ldots,a_n)$. Thus, the coefficients of the polynomial $P(x_0,\ldots,x_n)$ are independent of $k$ and $\sF$. As $h^1(\sF(m_0))\geq 0$, we must have $P(a_0,\ldots,a_n)\geq 0$. Thus, we get
	\[ H^!(\sF(m))=0 \;\; \text{for}\;\; m\geq G(a_0,\ldots,a_n) + P(a_0,\ldots,a_n).\]
	Taking $F_{p,n}(x_0,\ldots,x_n):= G(x_0,\ldots,x_n)+ P(x_0,\ldots, x_n)$, we see that $\sF$ is $F_{p,n}(a_0,\ldots,a_n)$-regular.
\end{proof}


\bibliography{Mybib.bib}
\bibliographystyle{alpha}

\end{document}



