\documentclass[ignorenonframetext,t]{beamer}
\usepackage{amscd}
\usepackage{verbatim}

\usetheme{Frankfurt}

\setbeamerfont{block body}{size=\small}

\setbeamercolor{mycolor}{fg=white,bg=black}
\defbeamertemplate*{footline}{shadow theme}{%
	\leavevmode%
	\hbox{\begin{beamercolorbox}[wd=.5\paperwidth,ht=2.5ex,dp=1.125ex,leftskip=.3cm plus1fil,rightskip=.3cm]{author in head/foot}%
			\usebeamerfont{author in head/foot}\hfill\insertshorttitle\, - \insertsubtitle
		\end{beamercolorbox}%
		\begin{beamercolorbox}[wd=.4\paperwidth,ht=2.5ex,dp=1.125ex,leftskip=.3cm,rightskip=.3cm plus1fil]{title in head/foot}%
			\usebeamerfont{title in head/foot}\insertshortauthor\hfill%
		\end{beamercolorbox}%
		\begin{beamercolorbox}[wd=.1\paperwidth,ht=2.5ex,dp=1.125ex,leftskip=.3cm,rightskip=.3cm plus1fil]{mycolor}%
			\hfill\insertframenumber\,/\,\inserttotalframenumber
	\end{beamercolorbox}}%
	\vskip0pt%
}


%% LaTeX Definitions
%\newcounter{countup}

\newcommand{\rup}[1]{\lceil{#1}\rceil}
\newcommand{\rdown}[1]{\lfloor{#1}\rfloor}
\newcommand{\ilim}{\mathop{\varprojlim}\limits} % inverse limit
\newcommand{\dlim}{\mathop{\varinjlim}\limits}  % direct limit
\newcommand{\surj}{\twoheadrightarrow}
\newcommand{\inj}{\hookrightarrow}
\newcommand{\tensor}{\otimes}
\newcommand{\ext}{\bigwedge}
\newcommand{\Intersection}{\bigcap}
\newcommand{\Union}{\bigcup}
\newcommand{\intersection}{\cap}
\newcommand{\union}{\cup}

%%%%%%%%%%%%%%%%%%%%%%%%%%%%% new new commands :) %%%%%%%%%%%%%%%%
\newcommand{\supp}{{\rm Supp}}
\newcommand{\Exceptional}{{\rm Ex}}
\newcommand{\del}{\partial}
\newcommand{\delbar}{\overline{\partial}}
\newcommand{\boldphi}{\mbox{\boldmath $\phi$}}

%%%%%%%%%%%%%%%%%%%%%%%%%%%%%%%%%%%%%%%%%%%%%%%%%%%%%%%%%%%%%%%%%%%%%%%%%%%%%%

\newcommand{\udiv}{\underline{\Div}}

%%%%%%%%%%%%%%%%%

\newcommand{\Proj}{{\P roj}}
\newcommand{\sEnd}{{\sE nd}}
\newcommand{\mc}{\mathcal}
\newcommand{\mb}{\mathbb}
\newcommand{\an}{{\rm an}} 
\newcommand{\red}{{\rm red}}
\newcommand{\codim}{{\rm codim}}
\newcommand{\Dim}{{\rm dim}}
\newcommand{\rank}{{\rm rank}}
\newcommand{\Ker}{{\rm Ker  }}
\newcommand{\Pic}{{\rm Pic}}
\newcommand{\per}{{\rm per}}
\newcommand{\ind}{{\rm ind}}
\newcommand{\Div}{{\rm Div}}
\newcommand{\Hom}{{\rm Hom}}
\newcommand{\Aut}{{\rm Aut}}
\newcommand{\im}{{\rm im}}
\newcommand{\Spec}{{\rm Spec \,}}
\newcommand{\Sing}{{\rm Sing}}
\newcommand{\sing}{{\rm sing}}
\newcommand{\reg}{{\rm reg}}
\newcommand{\Char}{{\rm char}}
\newcommand{\Tr}{{\rm Tr}}
\newcommand{\Gal}{{\rm Gal}}
\newcommand{\Min}{{\rm Min \ }}
\newcommand{\Max}{{\rm Max \ }}
\newcommand{\Alb}{{\rm Alb}\,}
\newcommand{\Mat}{{\rm Mat}}
%\newcommand{\GL}{{\rm GL}\,}        % For the general linear group
\newcommand{\GL}{{\G\L}}
\newcommand{\Ho}{{\rm Ho}}
\newcommand{\ie}{{\it i.e.\/},\ }
\renewcommand{\iff}{\mbox{ $\Longleftrightarrow$ }}
\renewcommand{\tilde}{\widetilde}
% Skriptbuchstaben
\newcommand{\sA}{{\mathcal A}}
\newcommand{\sB}{{\mathcal B}}
\newcommand{\sC}{{\mathcal C}}
\newcommand{\sD}{{\mathcal D}}
\newcommand{\sE}{{\mathcal E}}
\newcommand{\sF}{{\mathcal F}}
\newcommand{\sG}{{\mathcal G}}
\newcommand{\sH}{{\mathcal H}}
\newcommand{\sI}{{\mathcal I}}
\newcommand{\sJ}{{\mathcal J}}
\newcommand{\sK}{{\mathcal K}}
\newcommand{\sL}{{\mathcal L}}
\newcommand{\sM}{{\mathcal M}}
\newcommand{\sN}{{\mathcal N}}
\newcommand{\sO}{{\mathcal O}}
\newcommand{\sP}{{\mathcal P}}
\newcommand{\sQ}{{\mathcal Q}}
\newcommand{\sR}{{\mathcal R}}
\newcommand{\sS}{{\mathcal S}}
\newcommand{\sT}{{\mathcal T}}
\newcommand{\sU}{{\mathcal U}}
\newcommand{\sV}{{\mathcal V}}
\newcommand{\sW}{{\mathcal W}}
\newcommand{\sX}{{\mathcal X}}
\newcommand{\sY}{{\mathcal Y}}
\newcommand{\sZ}{{\mathcal Z}}
% Sonderbuchstaben mit Doppellinie
\newcommand{\A}{{\mathbb A}}
\newcommand{\B}{{\mathbb B}}
\newcommand{\C}{{\mathbb C}}
\newcommand{\D}{{\mathbb D}}
\newcommand{\E}{{\mathbb E}}
\newcommand{\F}{{\mathbb F}}
\newcommand{\G}{{\mathbb G}}
\newcommand{\HH}{{\mathbb H}}
\newcommand{\I}{{\mathbb I}}
\newcommand{\J}{{\mathbb J}}
\newcommand{\M}{{\mathbb M}}
\newcommand{\N}{{\mathbb N}}
\renewcommand{\O}{{\mathbb O}}
\renewcommand{\P}{{\mathbb P}}
\newcommand{\Q}{{\mathbb Q}}
\newcommand{\R}{{\mathbb R}}
\newcommand{\T}{{\mathbb T}}
\newcommand{\U}{{\mathbb U}}
\newcommand{\V}{{\mathbb V}}
\newcommand{\W}{{\mathbb W}}
\newcommand{\X}{{\mathbb X}}
\newcommand{\Y}{{\mathbb Y}}
\newcommand{\Z}{{\mathbb Z}}
\newcommand{\Sh}{\sS h}
\newcommand{\deltaop}{\Delta^{op}(\sS h(Sm/\mathbf{k}))}
\newcommand{\pdeltaop}{\Delta^{op}(P\sS h(Sm/\mathbf{k}))}
%\newcommand{\psh}{\pi_0^{\text{\tiny pre}}}
\newcommand{\psh}{\pi_0}
\renewcommand{\k}{\mathbf{k}}

\newcommand{\colim}{{\rm colim \,}}
\newcommand{\DM}[2]{\mathbf{DM}_{#2}^{\mathit{eff}}(#1)}

\theoremstyle{definition}
\newtheorem{question}[theorem]{Question}




% Document information
\title[Moduli@IISERP]{Seminar on Moduli Theory}
\subtitle{Lecture 2}
\author{Neeraj Deshmukh}
\date{\today}
%\address[IISERM]{Indian Institute of Science Education and Research, Mohali}

\begin{document}
	
	
\begin{frame}
\titlepage
\end{frame}

\begin{frame}{Last Week}
\begin{enumerate}
	\item Affine Communication Lemma.
	\item Two ways of gluing $\A^1$ outside the origin.
	\item DVR with a double origin, and sheaves on it.
	\item Line bundles on $\P^1$.
\end{enumerate}
\end{frame}

%\subsection{A slightly more involved scheme: \texorpdfstring{$\P^n$}{Pn}.}

\begin{frame}{A slightly more involved scheme: $\P^n$}
Let $D(x_i):= \Spec k[x_{0/i},x_{1/i},\ldots,\ldots,x_{n/i}]/(x_{i/i}-1)$.
If we invert one of the variables, say $x_{j/i}$, we can write an isomorphism $D(x_i)_{x_{j/i}}\cong D(x_j)_{x_{i/j}}$ given by the maps 

\[\phi_{ij}:x_{k/i}\mapsto x_{k/j}/x_{i/j}\;\; \&\;\; \phi_{ji}:x_{k/j}\mapsto x_{k/i}/x_{j/i}\]
\end{frame}
This is basically the same as $\A^n$, but we write it like this for reasons that will become evident soon. If we invert one of the variables, say $x_{j/i}$, we can write an isomorphism $D(x_i)_{x_{j/i}}\cong D(x_j)_{x_{i/j}}$ given by the maps 

\[\phi_{ij}:x_{k/i}\mapsto x_{k/j}/x_{i/j}\;\; \&\;\; \phi_{ji}:x_{k/j}\mapsto x_{k/i}/x_{j/i}\]
Now we just have to check that this agrees on triples. For this, you just have to check that $\phi_{ij}\circ\phi_{jk}=\phi_{ik}$ (what is the (co)domain of these maps?). Note that this construction doesn't really utilise the fact that $k$ is a field.

%\subsection{A classical interlude.} 
\begin{frame}{A classical interlude}
Let $k$ be a field. Consider $k^{n+1}\setminus{(0,0,\ldots,0)}$. We define $\P^n$ to be:
\[\P^n:=\lbrace (x_i)\; |\; (x_i)\simeq (y_i)\; \text{if there is a $\lambda\in k^{\times}$ such that}\; x_i=\lambda y_i\rbrace\]

\end{frame}


Here's a classical definition of $\P^n$ in terms of \textit{homogeneous coordinates}. Let $k$ be a field. Consider $k^{n+1}\setminus{(0,0,\ldots,0)}$. We define $\P^n$ to be:
\[\P^n:=\lbrace (x_0,x_1,\ldots,x_n)\; |\; (x_i)\simeq (y_i)\; \text{if there is a $\lambda\in k^{\times}$ such that for all $i$,}\; x_i=\lambda y_i\rbrace\]

We denote the equivalence class of the tuple $(x_i)$ by $[x_0:x_1:\ldots:x_n]$. These are called homogeneous coordinates. If we assume one of the coordinates to be non-zero, say $x_i$, then we can divide the entire tuple by it. This gives us a set, $D(x_i):=\lbrace [x_0/x_i:x_1/x_i:\ldots:1:\ldots:x_n/x_i]\rbrace$. It is easy to see that this set in bijection with $k^n$. Set $x_k/x_i:=x_{k/i}$\footnote{This notation is meant to be suggestive.}, then tuples in $D(x_i)$ look like $[x_{0/i}:x_{1/i}:\ldots:1:\ldots:x_{n/i}]$. This $D(x_i)$ should be thought of as the complement of the hyperplane defined by $x_i$ (which is actually a projective space of one dimension less). 

If an $x_j$ is non-zero, for a $j$ distinct from $i$, then we can divide by it. This gives us the relation, $x_{k/i}/x_{j/i}=x_{k/j}$. This is the origin of the morphisms $\phi_{ij}$ above\footnote{If you have seen the construction of Grassmannians as smooth manifolds, the same construction also goes through in algebraic geometry.}. Homogenisation and de-homogenisation(?).


\begin{frame}{Motivating $\Proj$}
	Consider $k[x_0,x_1,\ldots,x_n]$, now thought of as a graded ring with the grading given by degrees of monomials.\\
	$\Proj(k[x_0,x_1,\ldots,x_n])$ is the set of those \text{homogeneous} prime ideals which do not contain the ideal $(x_0,x_1,\ldots,x_n)$. The resulting scheme is $\P^n$.
\end{frame}

%\subsection{Proj of a graded ring.} Now that we have defined $\P^n$, I want to motivate the proj construction for graded rings using the example of $\P^n$. In a sense, this construction is not very different from the gluing construction above. However, it gives us more control ove the algebra and sheaf theory of $\P^n$ and its subschemes. For example, every closed subscheme of $\P^n$ comes from a graded ideal (this is a neat analogue of the affine case). Another advantage is that it lets us talk about affine open covers given by the complements of non-linear hypersurfaces. This can be useful when dealing with closed subschemes of $\P^n$, but then we won't need to make any choices about waht affine covers should be. 

Consider the ring $S_{\bullet}:=k[x_0,x_1,\ldots,x_n]$, now thought of as a graded ring with the grading given by degrees of monomials. The degree of a monomial $x^{r_1}_{i_1}\ldots x^{r_m}_{i_m}$ is the integer $r_1+\ldots +r_m$. The constants have degree zero. We can write our ring as $S_{\bullet}=\oplus_{i\geq 0}S_i$, where each $S_i$ is the homogeneous component of degree $i$. Note that $S_{\bullet}$ is generated by the elements $x_i$'s as an $S_0$-algebra (here, $S_0$ is $k$). The ideal generated by the $x_i$'s is just $S_{+}=\oplus_{i>0}S_i$. This is just the ideal $(x_0,\ldots,x_n)$ written by keeping track of the grading. We will call this the irrelevant ideal. Then, proj of the graded ring $S_{\bullet}$, $\Proj(S_{\bullet})$ is the set of those \text{homogeneous} prime ideals which do not contain $S_+$. This inherits a Zariski topology from $\Spec S_{\bullet}$. We can then to check that this is a scheme by producing affine open covers using the homogeneous elements. The resulting scheme is $\P^n$.

\begin{frame}
Here's an alternative description of $\P^1$ using degree $2$ hypersufaces
\end{frame}

\begin{frame}
Relation to our original construction of $\P^1$
\end{frame}


\begin{frame}
Note that if you just invert $x^2$ and $xy$, then this does not give a cover $\P^1$, since the the radical of $(x^2,xy)$ does not contain the irrelevant ideal. Geometrically speaking, this is because inverting $xy$ corresponds to the affine open of $\P^1$ obtained by knocking off $0$ and $\infty$.\\
\end{frame}

This can be a bit tricky to write down when the homogeneous elements have large degrees. But to give you a flavour of what is going on, let's examine what this is in for $\P^1$. We have already seen a description of $\P^1$ above, by gluing $\A^1\setminus 0$ in an ``inverse fashion".  By the above discussion, $\P^1:=\Proj k[x,y]$. Here's a slightly different description of $\P^1$ using degree $2$ hypersurfaces. Consider the homogeneous ideal $(x^2,xy,y^2)$. Localise $k[x,y]$ with respect to $x^2$. This gives us a graded ring $(S_{\bullet})_{x^2}=k[x,y]_{x^2}$, where $1/x^2$ has degree $-2$. Thus, elements here are of the form $f(x,y)/(x^{2})^N$ and can have negative degrees. Look at the zero graded piece of this ring which we will denote by the horrible notation, $((S_{\bullet})_{x^2})_0$. This ring can be described as,
\[((S_{\bullet})_{x^2})_0=k[xy/x^2,y^2/x^2].\]
This can be identified with $k[y/x]$. So, $\Spec ((S_{\bullet})_{x^2})_0=\A^1$. At this point, we have to check that this does give an affine open in $\Proj k[x,y]$. This is true because there is a bijection between prime ideal of $((S_{\bullet})_{x^2})_0$ and the homogeneous prime ideals of $(S_{\bullet})_{x^2}$ (One way is easy. For the other direction, take a prime $\mathfrak{p}$ in $((S_{\bullet})_{x^2})_0$ and show that the \textit{radical} of the homogeneous ideal generated by $\mathfrak{p}$ in $(S_{\bullet})_{x^2}$ in prime\footnote{SThere is nothing particularly enlightening in doing this exercise for $x^2$, the exact same proof works for any element of positive degree.}). Similarly, inverting the elements $xy$ and $y^2$ in $S_{\bullet}$ and looking at the zeroth graded pieces gives us the polynomial rings,
\begin{align*}
((S_{\bullet})_{y^2})_0 &=k[x^2/y^2,xy/y^2]\longleftrightarrow \A^1\\
((S_{\bullet})_{xy})_0 &=k[x^2/xy,y^2/xy]\longleftrightarrow \A^1\setminus 0.
\end{align*}
The radical of the ideal $(x^2,xy,y^2)$ contains the irrelevant ideal (why?). This implies that every homogeneous prime ideal is contained in one of the above three affine pieces. Thus, this gives us a covering of $\P^1$.

If instead we invert degree one element $x$ and $y$, the zeroth graded pieces of these localisations look like,
\begin{align*}
((S_{\bullet})_{x})_0 &=k[x/y]\longleftrightarrow\A^1\\
((S_{\bullet})_{y})_0 &=k[y/x]\longleftrightarrow\A^1.
\end{align*}
This will then recover our original construction of $\P^1$.

\underline{\textbf{Caution:}} While using affine opens corresponding to degree $2$ elements, it may seem like it should be sufficient to just invert $x^2$, and $y^2$ to get an open cover of $\P^1$. This is false! For example, the homogeneous ideal $(x-y)$ is not contained in the radical of $(x^2,y^2)$.


%\subsection{Some more examples.}
\begin{frame}{More Examples}
$V_{+}(x^2+y^2+z^2)$ over $\R$ and $\C$.
\end{frame}

\begin{frame}{More Examples}
Blow-up of $\A^2$ at the origin.	
\end{frame}


\begin{frame}{More Examples}
An example of a scheme without a closed point.
	
\end{frame}





%\section{Morphism}
As mentioned before, many of the properties of morphisms that we are interested in are ``globalised" versions of properties of ring maps. However, we have to first say what it means for morphism of schemes to be a local property. There are three kinds of local properties: local on the source, local on the target, local on the source and target. We will say what this means now:

\begin{frame}{Morphisms}
\begin{definition}
	Let $\mathcal{P}$ be a property of morphisms of schemes. Let $f:X\rightarrow Y$ be a morphism which satisfies $\mathcal{P}$. Then,
	\begin{enumerate}
		\item We say that $\mathcal{P}$ is \textit{affine-local on the target} if given any affine open cover $\lbrace V_i\rbrace$ of $Y$, $f:X\rightarrow Y$ has $\mathcal{P}$ if and only if the restriction $f: f^{-1}(V_i)\rightarrow V_i$ has $\mathcal{P}$ for each $i$.
		\item We say that $\mathcal{P}$ is \textit{affine-local on the source} if given any affine open cover $\lbrace U_i\rbrace$ of $X$, $X\rightarrow Y$ has $\mathcal{P}$ if and only if the composite $U_i\rightarrow Y$ has $\mathcal{P}$ for each $i$.
	\end{enumerate}
\end{definition}

Using \textit{affine communication lemma} one can then show that it suffices to check the above statements on single affine open cover.

\end{frame}

An important maxim of Grothendieck was that instead of considering schemes in isolation, we should look at things relative to each other, i.e, everything should be seen as a propery of morphisms. This is mostly true: every property of schemes can be turned into a property of morphisms of schemes.

Examples:

\begin{frame}
	Something flat, something finitely presented/finite type, something finite.
\end{frame}

\begin{frame}
	Not all properties are like this. For example, separatedness, properness, quasi-compactness, etc.
\end{frame}

\begin{frame}
	Something ramified, something smooth, something singular.
\end{frame}

\begin{enumerate}
	\item $x\mapsto x^2$ (more, generally $x^n$). This covers ramified, finitely presented, flat.
	\item A non-quasi-compact, open-immersion. Polynomial ring in infinitely many variables and knock off the origin.
	\item A finte morphism. 
	\item A smooth morphism. A non-smooth morphism (nodal curve over $\A^1$).
\end{enumerate}

Open embeddings are locally finite presentation.\footnote{This is not true in perfectoid geometry, which is quite sad.}

Open embedding is \'{e}tale is fppf is fpqc


\bibliography{Mybib.bib}
\bibliographystyle{alpha}

\end{document}



